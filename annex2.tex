\chapter{Questionnaire} \label{app:questionnaire}
%\refstepcounter{annex}

This second addition contains the \enquote{explainability evaluation questionnaire} that was prepared in order to \enquote{formally} test the \textit{explanatory powers} of the prototype system developed during this thesis.
The questionnaire is references in the relevant Subsection \ref{subsec:explainability-validation}.

\begin{mdframed}
	\begin{center}
		{\huge Explainability evaluation questionnaire}
	\end{center}
	{\Large Confidence}
	\begin{enumerate} 
		\item Did the tool increase the confidence in diagnosis when diagnostic screening results were missing for a patient?  Why? \\
		O Yes O No
		\item Did the tool help in characterising a particular patient's profile? \\
		O Not at all O Somewhat O Absolutely
		\item Did the tool help in your confidence of understanding the cohort characteristics?  How? \\
		O Not at all O Somewhat O Absolutely
		\item Did the tool improve your confidence in your clinical decision-making?  How? \\ 
		O Not at all O Somewhat O Absolutely
		\item Did having the tool at your disposal improve your confidence when making time-constrained decisions?  How? (for example, did it improve confidence in prioritising some tests over others?) \\
		O Not at all O Somewhat O Absolutely
	\end{enumerate}
	{\Large Features}
	\begin{enumerate}[resume]
		\item Given the modes of interaction with the system labelled as \enquote{dialogues}, do you think you would have had more difficulty in interpreting the data without the these modalities? \\
		O No O Maybe O Yes
		\item Was natural language useful during the interaction?  Why? \\
		O No O Maybe O Yes
		\item Which type of \enquote{dialogue} did you feel was most useful? Why? \\
		O Exhaustive O Separations O Thresholded O A combination of the previous O All O None
		\item Did you feel that the dialogue helped you in cases of uncertainty?  If yes, how?  If no, why? \\
		O No O Somewhat O Yes
		\item Did you feel that the \enquote{dialogue} helped your clinical decision-making?  If yes, how?  If no, why? \\
		O No O Somewhat O Yes
		\item Did the generation of \enquote{counterfactual branches} help in your understanding of the data?  Why? \\
		O No O Somewhat O Yes
		\item Given the interaction mode labelled \enquote{pseudo-MPE query}, how would you rate the solutions it proposed from a point of view of their understandability? (1 poor, 5 good) \\
		O 1 O 2 O 3 O 4 O 5
		\item How would you rate the \enquote{pseudo-MPE} solutions from a point of view of their clinical usefulness? \\
		O 1 O 2 O 3 O 4 O 5
		\item Do you feel that the interaction mode labelled as \enquote{MPE query} gave better solutions than that labelled \enquote{pseudo-MPE query}?  Why? \\
		O No O Maybe O Yes
		\item Did you find the \enquote{pseudo-MPE} or \enquote{MPE} interaction mode the most useful?  Why? \\
		O \enquote{pseudo-MPE} O MPE O Both O None
		\item How important was the highlighting of the independencies between variables? \\
		O 1 O 2 O 3 O 4 O 5
		\item Do you think you would have had more difficulty in interpreting the data without the correlation strength displayed? \\
		O No O Maybe O Yes
		\item Do you think you would have had more difficulty in interpreting the data without visualisations? \\
		O No O Maybe O Yes
		\item Do you think you would have had more difficulty in interpreting the data without natural language output? \\
		O No O Maybe O Yes\\
	\end{enumerate}
	{\Large Time}
	\begin{enumerate}[resume]
		\item How would you rate the time it took to understand the dialogues' outputs?  Which of the three was best? (1 bad, 5 good) \\
		O 1 O 2 O 3 O 4 O 5
		\item How would you rate the time it took to understand the conditional probability query's outputs \\
		O 1 O 2 O 3 O 4 O 5
		\item How would you rate the time it took to understand the MPE and \enquote{pseudo-MPE} query's outputs? \\
		O 1 O 2 O 3 O 4 O 5
		\item Did natural language help in reducing the time needed to understand the outputs? \\
		O No O Somewhat O Yes
		\item Did visualisations help in reducing the time needed to understand the outputs? \\
		O No O Somewhat O Yes
	\end{enumerate}
	{\Large Tool}
	\begin{enumerate}[resume]
		\item Which interaction modes did you feel could be the most useful?  Why? \\
		O Plot model O Independencies O Conditional Probability Query O \enquote{pseudo-MPE} and MPE O Dialogues
		\item Which interaction modes did you use the most?  Why? \\
		O Plot model O Independencies O Conditional Probability Query O \enquote{pseudo-MPE} and MPE O Dialogues
		\item How did you use the tool in your day-to-day work?
		\item Is the tool missing any functionality that would address your needs?  If yes, which ones? \\
		O No O Yes
		\item Did you have any difficulties in understanding which functionalities to use to address your needs?  If yes, when? \\
		O No O Yes
		\item Did you have any difficulties in understanding the functionalities during usage?  If yes, when? \\
		O No O Yes
		\item If you answered Yes to the previous question, how do you think this could be addressed?
		\item Could you suggest any functionalities you would like to be implemented?\\\\
	\end{enumerate}
	{\Large Clinical}
	\begin{enumerate}[resume]
		\item Did the tool help in recovering missing features of patients thus supporting diagnostic profile creation and decision making? If yes, which is/are the feature/s that benefited the most? \\
		O No O Yes
		\item Did any of the tool's predictions have clinical confirmation later on?  If yes, how? \\
		O No O Yes
		\item Did the tool help in highlighting new relationships between variables? \\
		O No O Yes
		\item Did the tool help in highlighting new patient subgroups? \\
		O No O Yes
	\end{enumerate}
	{\Large Satisfaction}
	\begin{enumerate}[resume]
		\item What is your general satisfaction with the tool? For what reasons? \\
		O Completely dissatisfied O Somewhat dissatisfied O Neutral O Somewhat satisfied O Completely satisfied
	\end{enumerate}
	%\refstepcounter{subannex}
\end{mdframed}