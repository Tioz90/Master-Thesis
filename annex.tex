\newcounter{annexcounter}

\begin{sidewaystable*}[h]
  \centering
  \captionsetup{name=Annex}
  \caption{Natural language questions that can be answered by conditional probability queries}
    \begin{tabularx}{\textwidth}{lXllllX}
	    \toprule
	    \textbf{\#} & Natural language question & Type  & Target variable & Target value & Evidence variable & Evidence value \\
		\midrule
		\textbf{1} & At diagnosis, if estrogen receptors are negative, is tumor proliferative index high? & validation & ki67  & >30\% & estrogeni & 0-10\% \\
		\addlinespace
		\textbf{2} & At diagnosis, if estrogen receptors are negative, is the risk of metastases low? & validation & pM sub & pM=0  & estrogeni & 0-10\% \\
		\addlinespace
		\multirow{2}[0]{*}{\textbf{3}} & \multirow{2}[0]{4cm}{If estrogen receptors are negative and tumor proliferative index is  high at diagnosis, is the risk of metastases low?} & \multirow{2}[0]{*}{validation} & \multirow{2}[0]{*}{pM sub} & \multirow{2}[0]{*}{pM=0} & estrogeni & 0-10\% \\
		      &       &       &       &       & ki67  & >30\% \\
		\addlinespace[9ex]
	      \textbf{4} & If the diagnosis of mammary carcinoma happened at a young age, is tumour proliferative index high? & validation/research & ki67  & >30\% & eta arrotondata & <40 \\
      	\addlinespace
		\textbf{5} & If the histologic diagnosis is of lobular carcinoma, is the expression of the c-erbB2 marker absent? & validation & cerb  & 0 \& 1 & morfologia & Lobular carcinoma, NOS \\
		\addlinespace
		\textbf{6} & If the tumour is large, is lymph node involvement more probable? & validation & pN sub & pN!=0 & pT sub & pT>=2  \\  
		\addlinespace[2ex]
		\multirow{2}[0]{*}{\textbf{7}} & \multirow{2}[0]{4cm}{If the tumour is large and lymph nodes are involved, is the risk of metastases low at diagnosis?} & \multirow{2}[0]{*}{validation} & \multirow{2}[0]{*}{pM sub} & \multirow{2}[0]{*}{pM=0} & pT sub & pT>=2  \\ 
		&       &       &       &       & pN sub & pN!=0 \\  
\end{tabularx}
	\refstepcounter{annexcounter}
	\label{ann:conditionalprobability1}
\end{sidewaystable*}

\begin{sidewaystable*}[h]
  \centering
  \captionsetup{name=Annex}
  \caption{Natural language questions that can be answered by conditional probability queries}
    \begin{tabularx}{\textwidth}{lXllllX}
	    \toprule
	    \textbf{\#} & Natural language question & Type  & Target variable & Target value & Evidence variable & Evidence value \\
		\midrule      
		\textbf{8} & If the tumour is of high grade at diagnosis, is the risk of metastases low? & validation & pM sub & pM=0  & differenziazione & poco differenziato \\
		\multirow{2}[0]{*}{\textbf{9}} & \multirow{2}[0]{4cm}{In young patients, if estrogen receptors are negative, is tumor proliferative index high?} & \multirow{2}[0]{*}{validation} & \multirow{2}[0]{*}{ki67} & \multirow{2}[0]{*}{>30\%} & estrogeni & 0-10\% \\
		      &       &       &       &       & eta arrotondata & <40 \\	  
	    \addlinespace[6ex]
		\multirow{2}[0]{*}{\textbf{10}} & \multirow{2}[0]{4cm}{In young patients, if estrogen receptors are negative, is the risk of metastases low?} & \multirow{2}[0]{*}{validation} & \multirow{2}[0]{*}{pM sub} & \multirow{2}[0]{4cm}{pM=0} & estrogeni & 0-10\% \\
		      &       &       &       &       & eta arrotondata & <40 \\
      \addlinespace[6ex]
		\multirow{3}[0]{*}{\textbf{11}} & \multirow{3}[0]{4cm}{In young patients, if estrogen receptors are negative and tumor proliferative index is  high at diagnosis, is the risk of metastases low?} & \multirow{3}[0]{*}{validation} & \multirow{3}[0]{*}{pM sub} & \multirow{3}[0]{*}{pM=0} & estrogeni & 0-10\% \\
		      &       &       &       &       & ki67  & >30\% \\
		      &       &       &       &       & eta arrotondata & <40 \\
      \addlinespace[9ex]
	      \multirow{3}[0]{*}{\textbf{12}} & \multirow{3}[0]{4cm}{How does the tumoural grade change if I know the oestrogen expression?} & \multirow{3}[0]{*}{research} & \multirow{3}[0]{*}{differenziazione} &  ben differenziato & \multirow{3}[0]{*}{estrogeni} & negativi (0-10\%) \\
		      &       &       &       &  moderatamente differenziato &       & debolmente positivo (10-50\%) \\
		      &       &       &       &  poco differenziato &       & fortemente positivo (>50\%) \\
	      \addlinespace[1ex]
		\multirow{4}[0]{*}{\textbf{13}} & \multirow{4}[0]{4cm}{How does the oestrogen expression change if I know the proliferation index?} & \multirow{4}[0]{*}{research} & \multirow{4}[0]{*}{estrogeni} &  negativi (0-10\%) & \multirow{4}[0]{*}{ki67} & negativo (0-14\%) \\
		      &       &       &       &  debolmente positivi (10-50\%) &       & 14-20\% \\
		      &       &       &       & \multirow{2}[0]{*}{fortemente positivi (>50\%)} &       & 20-30\% \\
		      &       &       &       &       &       & positivi (>30\%) \\
      \addlinespace[1ex]
	      \textbf{14} & Does the negative expression of progestinic receptors influence lymph nodes' state? & validation & pN sub & pN!=0  & progestinici & 0-10\% \\
      \end{tabularx}
	\refstepcounter{annexcounter}
	\label{ann:conditionalprobability2}
\end{sidewaystable*}

\begin{sidewaystable}[h]
	\centering
	\captionsetup{name=Annex}
	\caption{Natural language questions that can be answered by d-separation queries}
	\begin{tabularx}{\textwidth}{lXllllX}
		\toprule
		\textbf{\#} & Natural language question & Type  & Target variable & Target value & Evidence variable & Evidence value \\
		\midrule
		\textbf{14} & Which clinical-pathological variables influence the lymph nodes' state at diagnosis? & research & pN    & -     & -     & - \\
		\textbf{15} & Which clinical-pathological variables influence tumoural proliferation index? & research & ki67  & -     & -     & - \\
		\textbf{16} & Which clinical-pathological variables influence the expression of the c-ERBB2 marker? & research & cerb  & \multicolumn{1}{r}{} & -     & - \\
		\textbf{17} & Which clinical-pathological variables influence the oestrogen expression? & research & recettori estrogeni & \multicolumn{1}{r}{} & -     & - \\
		\textbf{18} & Which clinical-pathological variables influence the tumoural grade? & research & differenziazione & \multicolumn{1}{r}{} & -     & - \\
		\textbf{19} & Which clinical-pathological variables influence the presence of metastases at diagnosis? & research & pM    & \multicolumn{1}{r}{} & -     & - \\
		\textbf{20} & Which clinical-pathological variables influence the tumoural dimension? & research & pT    & \multicolumn{1}{r}{} & -     & - \\
		\textbf{21} & Which clinical-pathological variables influence the age of the tumour onset? & research & eta   & -     & -     & - \\
		\end{tabularx}
	\refstepcounter{annexcounter}
	\label{ann:dseparation}
\end{sidewaystable}

\begin{sidewaystable}[h]
	\centering
	\captionsetup{name=Annex}
	\caption{Natural language questions that can be answered by conditional probability queries or, at a higher level, by d-separation queries}
	\begin{tabularx}{\textwidth}{lXllllX}
		\toprule
		\textbf{\#} & Natural language question & Type  & Target variable & Target value & Evidence variable & Evidence value \\
		\midrule
		\multirow{2}[0]{*}{\textbf{22}} & \multirow{2}[0]{4cm}{In young patients, does a negative expression of the progestinic receptors influence the lymph nodes' state?} & \multirow{2}[0]{*}{research} & \multirow{2}[0]{*}{pN sub} & \multirow{2}[0]{*}{pN!=0} & progestinici & 0-10\% \\
	      &       &       &       &       & eta   & <40 \\
	      \addlinespace[9ex]
		\multicolumn{1}{r}{\textbf{23}} & Does a negative expression of progestinic receptors influence the tumoural proliferation index? & research & ki67  & >30\% & progestinici & 0-10\% \\
		\addlinespace
		\multirow{2}[0]{*}{\textbf{24}} & \multirow{2}[0]{4cm}{In young patients, does a negative expression of the progestinic receptors influence the tumoural proliferation index?} & \multirow{2}[0]{*}{research} & \multirow{2}[0]{*}{ki67} & \multirow{2}[0]{*}{>30\%} & progestinici & 0-10\% \\
		      &       &       &       &       & eta   & <40 \\
		      \addlinespace[10ex]
		\multirow{3}[0]{*}{\textbf{25}} & \multirow{3}[0]{4cm}{Does a negative expression of progestinic receptors influence the expression of the c-ERBB2 marker?} & \multirow{3}[0]{*}{research} & \multirow{3}[0]{*}{cerb} & 0 \& 1 & \multirow{3}[0]{*}{progestinici} & \multirow{3}[0]{*}{0-10\%} \\
		      &       &       &       & 2     &       &  \\
		      &       &       &       & 3     &       &  \\
		      \addlinespace[4ex]
		\multirow{4}[0]{*}{\textbf{26}} & \multirow{4}[0]{4cm}{In young patients, does a negative expression of the progestinic receptors influence the expression of the c-ERBB2 marker?} & \multirow{4}[0]{*}{research} & \multirow{4}[0]{*}{cerb} & 0 \& 1 & \multirow{3}[0]{*}{progestinici} & \multirow{3}[0]{*}{0-10\%} \\
		      &       &       &       & 2     &       &  \\
		      &       &       &       & \multirow{2}[0]{*}{3} &       &  \\
		      &       &       &       &       & eta   & <40 \\
		\end{tabularx}
	\refstepcounter{annexcounter}
	\label{ann:conditionalanddseparation}
\end{sidewaystable}

\begin{sidewaystable}[h]
	\centering
	\captionsetup{name=Annex}
	\caption{Natural language questions that can be answered by MPE queries}
	\begin{tabularx}{\textwidth}{lXllllX}
		\toprule
		\textbf{\#} & Natural language question & Type  & Target variable & Target value & Evidence variable & Evidence value \\
		\midrule
		\multirow{3}[0]{*}{\textbf{27}} & \multirow{3}[0]{4cm}{How are tumours characterised by a triple negative profile from the point of view of the other clinical-pathological variables?} & \multirow{3}[0]{*}{research} & \multirow{3}[0]{*}{-} & \multirow{3}[0]{*}{-} & cerb  & 0 \\
		\addlinespace[9ex]
	      &       &       &       &       & recettori estrogeni & negativo \\
	      &       &       &       &       & recettori progestinici & negativo \\
		\textbf{28} & How are tumours characterised by high ki67 from the point of view of the other clinical-pathological variables? & research & -     & -     & ki67  & >30 \\
		\addlinespace[2ex]
		\textbf{29} & How are tumours characterised by nodes involvement from the point of view of the other clinical-pathological variables? & research & -     & -     & pN    & !0 \\
		\end{tabularx}
	\refstepcounter{annexcounter}
	\label{ann:mpe}
\end{sidewaystable}

\begin{framed}
	\begin{center}
		{\huge Explainability evaluation questionnaire}
	\end{center}
	{\Large Confidence}
	\begin{enumerate} 
		\item Did the tool increase the confidence in diagnosis when diagnostic screening results were missing for a patient?  Why? \\
		O Yes O No
		\item Did the tool help in characterising a particular patient's profile? \\
		O Not at all O Somewhat O Absolutely
		\item Did the tool help in your confidence of understanding the cohort characteristics?  How? \\
		O Not at all O Somewhat O Absolutely
		\item Did the tool improve your confidence in your clinical decision-making?  How? \\ 
		O Not at all O Somewhat O Absolutely
		\item Did having the tool at your disposal improve your confidence when making time-constrained decisions?  How? (for example, did it improve confidence in prioritising some tests over others?) \\
		O Not at all O Somewhat O Absolutely
	\end{enumerate}
	{\Large Features}
	\begin{enumerate}[resume]
		\item Given the modes of interaction with the system labelled as \enquote{dialogues}, do you think you would have had more difficulty in interpreting the data without the these modalities? \\
		O No O Maybe O Yes
		\item Was natural language useful during the interaction?  Why? \\
		O No O Maybe O Yes
		\item Which type of \enquote{dialogue} did you feel was most useful? Why? \\
		O Exhaustive O Separations O Thresholded O A combination of the previous O All O None
		\item Did you feel that the dialogue helped you in cases of uncertainty?  If yes, how?  If no, why? \\
		O No O Somewhat O Yes
		\item Did you feel that the \enquote{dialogue} helped your clinical decision-making?  If yes, how?  If no, why? \\
		O No O Somewhat O Yes
		\item Did the generation of \enquote{counterfactual branches} help in your understanding of the data?  Why? \\
		O No O Somewhat O Yes
		\item Given the interaction mode labelled \enquote{Pseudo-MPE query}, how would you rate the solutions it proposed from a point of view of their understandability? (1 poor, 5 good) \\
		O 1 O 2 O 3 O 4 O 5
		\item How would you rate the Pseudo-MPE solutions from a point of view of their clinical usefulness? \\
		O 1 O 2 O 3 O 4 O 5
		\item Do you feel that the interaction mode labelled as \enquote{MPE query} gave better solutions than that labelled \enquote{Pseudo-MPE query}?  Why? \\
		O No O Maybe O Yes
		\item Did you find the \enquote{Pseudo-MPE} or \enquote{MPE} interaction mode the most useful?  Why? \\
		O Pseudo-MPE O MPE O Both O None
		\item How important was the highlighting of the independencies between variables? \\
		O 1 O 2 O 3 O 4 O 5
		\item Do you think you would have had more difficulty in interpreting the data without the correlation strength displayed? \\
		O No O Maybe O Yes
		\item Do you think you would have had more difficulty in interpreting the data without visualisations? \\
		O No O Maybe O Yes
		\item Do you think you would have had more difficulty in interpreting the data without natural language output? \\
		O No O Maybe O Yes
	\end{enumerate}
	{\Large Time}
	\begin{enumerate}[resume]
		\item How would you rate the time it took to understand the dialogues' outputs?  Which of the three was best? (1 bad, 5 good) \\
		O 1 O 2 O 3 O 4 O 5
		\item How would you rate the time it took to understand the conditional probability query's outputs \\
		O 1 O 2 O 3 O 4 O 5
		\item How would you rate the time it took to understand the MPE and Pseudo-MPE query's outputs? \\
		O 1 O 2 O 3 O 4 O 5
		\item Did natural language help in reducing the time needed to understand the outputs? \\
		O No O Somewhat O Yes
		\item Did visualisations help in reducing the time needed to understand the outputs? \\
		O No O Somewhat O Yes
	\end{enumerate}
	{\Large Tool}
	\begin{enumerate}[resume]
		\item Which interaction modes did you feel could be the most useful?  Why? \\
		O Plot model O Independencies O Conditional Probability Query O Pseudo-MPE and MPE O Dialogues
		\item Which interaction modes did you use the most?  Why? \\
		O Plot model O Independencies O Conditional Probability Query O Pseudo-MPE and MPE O Dialogues
		\item How did you use the tool in your day-to-day work?
		\item Is the tool missing any functionality that would address your needs?  If yes, which ones? \\
		O No O Yes
		\item Did you have any difficulties in understanding which functionalities to use to address your needs?  If yes, when? \\
		O No O Yes
		\item Did you have any difficulties in understanding the functionalities during usage?  If yes, when? \\
		O No O Yes
		\item If you answered Yes to the previous question, how do you think this could be addressed?
		\item Could you suggest any functionalities you would like to be implemented?
	\end{enumerate}
	{\Large Clinical}
	\begin{enumerate}[resume]
		\item Did the tool help in recovering missing features of patients thus supporting diagnostic profile creation and decision making? If yes, which is/are the feature/s that benefited the most? \\
		O No O Yes
		\item Did any of the tool's predictions have clinical confirmation later on?  If yes, how? \\
		O No O Yes
		\item Did the tool help in highlighting new relationships between variables? \\
		O No O Yes
		\item Did the tool help in highlighting new patient subgroups? \\
		O No O Yes
	\end{enumerate}
	{\Large Satisfaction}
	\begin{enumerate}[resume]
		\item What is your general satisfaction with the tool? For what reasons? \\
		O Completely dissatisfied O Somewhat dissatisfied O Neutral O Somewhat satisfied O Completely satisfied
	\end{enumerate}
	\refstepcounter{annexcounter}
	\label{ann:questionnaire}
\end{framed}


