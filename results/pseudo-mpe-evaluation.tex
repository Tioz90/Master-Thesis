\section{Pseudo-MPE Evaluation} \label{sec:pseudo-mpe-evaluation}
This is not a user-facing feature \textit{per se}, but a way of running a test to compare the outputs of the Pseudo-MPE algorithm with the exact solution (this is described in detail in Subsection \ref{subsec:algorithms-novel} under the \textbf{MPE Algorithms Comparison} header).

The Hamming and Jaccard distances between the MPE calculated with Pgmpy's \texttt{map\_query} and with DAOOPT should have been zero, as both use exact methods to solve the MPE problem.
It was seen that this was not the case and this lead to the discovery of the problems described in Subsection \ref{subsec:results-mpe-calculation-issues}.
As is explained in Section \ref{sec:issues}, the benchmark against which to compare the Pseudo-MPE was thus chosen to be Pgmpy's \texttt{map\_query} function.

The tests were run over 1000 iterations; that is, 1000 initial random evidences $\boldsymbol{U_e}$ were generated and the Pseudo-MPE solution was found using Algorithm \ref{alg:pseudo-mpe-initial-evidence} and was compared with the true MPE solution.
Given that there are twelve variables in the BN, the maximum distance possible for both Hamming and Jaccard distances is 12.
Thus, the average distances shown in Table \ref{tab:mpe-vs-pseudo-results}, are a very promising result.

\begin{table}[h]
	\centering
	\caption{Distance of Pseudo-MPE from true MPE solution}
	\begin{tabularx}{0.5\textwidth}{Xl}
		\toprule
		Distance measure & Average distance  \\
		\midrule	
		Hamming & 0.062 \\
		Jaccard & 0.048 \\
		\bottomrule
		\end{tabularx}
	\label{tab:mpe-vs-pseudo-results}
\end{table}

As discussed by \citet{gamez2013advances}, the MPE problem is \textit{NP-hard} and remains such even in restricted network topologies, networks with restricted probabilities and for constant-bound approximations.
MAP, as anticipated in Subsection \ref{subsec:bnupdating} is a harder problem than MPE and is in fact NP\textsuperscript{PP}-complete and, unlike MPE, remains NP-complete when restricted to polytrees.
\todo[inline]{numero quadratico di chiamate, si riduce MAP ad updating. entrambi sono hard ma di classe inferiore}
The proposed \enquote{Pseudo-MPE} algorithm, whose pseudocode is shown in Algorithm \ref{alg:pseudo-mpe-initial-evidence}, reduces the MAP problem to the updating problem that is polynomial and thus has a complexity $O(kV^2)$, in the number of nodes $V$ with $k$ a parameter that depends on the algorithm used for the calculation of posterior probabilities.

The quadratic term $V^2$ comes from the fact that every \textit{(state,value)} pair is greedily chosen based solely on its efficiency and added to the existing evidence set so at every step the number of available \textit{(state,value)} pairs decreases by $1$.
So, at every step one fewer updating call is done and the resulting complexity is that shown in Equation \ref{eq:pseudo-mpe-complexity}.
\begin{equation} \label{eq:pseudo-mpe-complexity}
	\sum_{x=1}^{V} x=\frac{1}{2} V(V+1) = O(V^2)
\end{equation}
It is trivial to see that there are at most only $V$ \textit{(state,value)} pairs because every \textit{state} is paired at each step with only one of its \textit{values} - the most probable.
The single most efficient pair is chosen at each step and then the corresponding \textit{state} is not selectable anymore.

The extra term $k$ depends on the algorithm used to compute the posterior probabilities, as at each step these are recalculated based on the updated evidence.
The Pomegranate-based implementation does this inference utilising the \texttt{predict\_proba} built-in function, that is based on \textit{loopy belief propagation}.
Loopy belief propagation is an \textit{approximate inference algorithm} that estimates a solution to the exact \textit{belief propagation algorithm} introduced by \citet{pearl1982reverend}, that is unfortunately exponential in complexity $O(M^V)$, with $M$ the maximum number of states that a variable in the network may have.
Loopy belief propagation is instead quadratic in the maximum number of values for a state $O(M^2)$.
\todo[inline]{giusto?}

Thus, the current Pseudo-MPE implementation has complexity $O(M^2V)$ with $M$ the maximum number of \textit{values} in any state and $V$ the number of states.