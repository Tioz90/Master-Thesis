\documentclass[mscthesis]{usiinfthesis}
\usepackage{lipsum}
\input{macros}

\usepackage{tikz}
\usetikzlibrary{graphs}

\usepackage{natbib}

\usepackage{listings}
\newcommand{\Eta}{H}

\usepackage{tabularx}
\usepackage{ragged2e}
\newcolumntype{Y}{>{\RaggedRight\arraybackslash}X} 
\usepackage{booktabs}
\renewcommand\tabularxcolumn[1]{m{#1}}

\usepackage{algorithm, algorithmicx, algpseudocode}

\newtheorem{theorem}{Theorem}[chapter]
\newtheorem{definition}[theorem]{Definition} 
\newtheorem{corollary}[theorem]{Corollary} 

\lstdefinelanguage{algebra}
{morekeywords={import,sort,constructors,observers,transformers,axioms,if,
else,end},
sensitive=false,
morecomment=[l]{//s},
}

\usepackage{multirow}



\title{Expert-driven approximation of MPE} %compulsory
\specialization{Artificial Intelligence}%optional
\subtitle{Subtitle: Reinventing the World} %optional 
\author{Thomas Francesco Tiotto} %compulsory
\begin{committee}
\advisor{Prof.}{Alessandro Facchini}{} %compulsory
\coadvisor{Prof.}{Alessandro Antonucci}{}{} %optional
\end{committee}
\Day{xx} %compulsory
\Month{September} %compulsory
\Year{2019} %compulsory, put only the year
\place{Lugano} %compulsory

\dedication{To my beloved} %optional
\openepigraph{Someone said \dots}{Someone} %optional

%\makeindex %optional, also comment out \theindex at the end

\begin{document}

\maketitle %generates the titlepage, this is FIXED

\frontmatter %generates the frontmatter, this is FIXED

\begin{abstract}
This is a very abstract abstract. 

\end{abstract}

\begin{acknowledgements}
hello
\end{acknowledgements}

\tableofcontents 
\listoffigures %optional
\listoftables %optional

\mainmatter

%%%%
%%%% INTRODUCTION
%%%%
\chapter{Introduction}\label{chap:introduction}

% !TEX root = thesis-thomas-tiotto.tex

\section{Context}
\textbf{state the general topic and give background of what your reader needs to know to understand the problem outline the current situation evaluate the current situation (advantages/ disadvantages) and identify the gap}

\todo[inline]{In genere mancano referenze}

While neural networks and Artificial Intelligence (AI) - as a field - have existed for nearly seventy years, the concept of artificial intelligence dates back to at least Ancient Greece.  In ancient times, artificial intelligence embodied in mechanical men was part of the domain of myth; in the twentieth century, of that of science.  During this last decade, Artificial Intelligence can't anymore \todo{What do you mean? Why ``reductive''?} be described by a limited set of terms, as it has materialised out of Man's imagination, broken out of laboratories and has been given lease to act in the world at large.


No sector of our economy has been left untouched by the recent and rapid rise of machine learning that has been enabled by the rediscovery of deep neural networks, the availability of Big Data and cheap parallel computing power.  Fields as diverse and as critical as are government, healthcare, finance and bioinformatics have been revolutionised and the possibility has been set for new ones - such as self-driving vehicles - to be born.  
The ever increasing reliance of our society on ever more complex machine learning-driven algorithms can only make us worry ever more about the ethical dilemmas posed by such a situation.  
Our society has only very recently been confronted with the dilemma of assigning blame when a driverless car causes the death of a person but this moral problem is only the tip of the iceberg, even when focusing only on the automotive industry.  For example, how should a self-driving car behave when confronted with a real-world analogous of the classic Trolley Problem - a situation where each course of action is liable to cause harm?  
On what basis should a person be denied a mortgage, access to university or a job interview?  How can we be sure that there is no bias in the system?  How do we even define if the system is behaving morally?  Would it currently be feasible for a person that feels they have been harmed by such a decision to appeal it, as prescribed by the recent EU General Data Protection Regulation (GDPR)? 
As more and more decisions are made in an automated way, with many of them significantly impacting both individuals and society at large, it comes natural to stop and wonder what are the characteristics we would want the systems making these decisions to have.   

\textit{Explainable AI }(xAI) is the sub-field of AI that rests at the intersection between Computer Science, Social Sciences and Philosophy and whose aim is to define our desiderata of artificially intelligent systems and machine learning algorithms from the point of view of their explainability.  The basic idea is that the prerequisite for the evaluation of the ethical and moral implications of a machine's decision is for the system to be \enquote{interpretable} or \enquote{explainable}.  
Within the xAI community, \todo{references to articles ?} there is currently no unanimously agreed upon definition of which these desiderata should be or of the best way to implement them in real systems.
There is also no common, agreed-upon, definition of what is meant by the phrase \enquote{understanding a system}: some authors equate \todo{explain better what follows here, since it is important} it to having a \todo{what does functional mean?} \textit{functional understanding}, void of the \todo{what are low level details} low-level details, while others decline it into the  concepts of \textit{interpretation} and \textit{explanation}, the former indicating \todo{??? explain better. gives examples, help the reader} the output of a format that a human user can comprehend and the latter a set of features that have contributed to generating the system's decision. 

The difficulties start even in trying to define what interpretability really is.  \todo{does not sound well expressed here. also what has trust to do with interpretability? explain, give intuitions, motivations} Does it mean to gain the trust the system's user?  Of type of user in particular?  Does trust stem from some property of the decisions the system makes or from some other inherent characteristic of the machine?
A common approach to solving the difficulty in defining interpretability is to try and define it post-hoc by categorising systems into \todo{why are you speaking about ontologies here (and not simply of categories etc?} ontologies, based on their perceived interpretablility; unfortunately this seems like a circular way of approaching the problem: the classification of system models is being done utilising the same criterion that is trying to be uncovered by doing so.
For reference, a commonly used classification is the following: 
\begin{itemize}
	\item \textbf{Opaque systems}: these are systems that offer no insight into the mapping between \todo{in general. explain what a system, input and output are} inputs and outputs; all closed-source algorithms fall under this definition;
	\item \textbf{Interpretable systems}: this is the vastest category, as the characteristic of these systems is \textit{transparency} i.e. their inner workings are accessible but the onus of comprehensibility falls completely onto the user.  The classical example is that of neural networks where the mapping from inputs to outputs (the \textit{weights}) is inspectable by the user who can, theoretically and depending on her skill, interpret them;
	\item \textbf{Comprehensible systems}: systems falling into this category emit additional symbols together with their outputs with the explicit intent of giving the user the means to interpret and understand the automated decisions; the additional symbols may be visualisations, natural-language text or any other means of demystifying the output.  These extra symbols would need to be graded based on the user's expertise, as comprehension is a property that involves both man and machine but materialises on the human side.
\end{itemize}
\todo{References!!!} Some authors propose to classify systems as \textit{non-interpretable}, \textit{ante-hoc interpretable/transparent} and \textit{post-hoc interpretable}; this roughly corresponds to the ontology presented above.

What I hope can be gleamed from this brief introduction to the field of Explainable Artificial Intelligence, is that many of the problems it aims to tackle are hard \textit{per-se} and may not have a unique optimal solution.  This is because these issues are not only engineering problems, but exist at the intersection between man and machine and as such can't be tackled using only the methods of Computer Science.  There is no way to satisfactorily investigate the human element of the situation without resorting to the \todo{for instance? which methods? example?} well-established methods of the Social Sciences.  There is little hope to know in which direction to procede without the guiding force that can only come from philosophy, because of its millennia-long tradition in thinking about ethical and high-level issues.  
It should be clear that when the human - and particularly the ethical - domain are part of the equation, it is impossible \textit{by definition} to find an optimal and unique solution.

\todo[inline]{you have  to link what you just said with what you are gonna do later. for instance, illustrate what you are saying with examples. use one close to what you are going to do. also, you speak about NN but never about graphical methods, such as BN. you have to speak about them here too. }

% !TEX root = thesis-thomas-tiotto.tex

\section{Problem and Significance}
\textbf{identify the importance of the proposed research - how does it address the gap? state the research problem/ questions state the research aims and/or research objectives state the hypotheses
}

\todo[inline]{again, references!}

AI has a trust problem.  The bigger problem with AI is not anymore its utility, as that has mostly been solved by \todo{ok but come on, there is not only NN in this world} deep neural networks, but its capacity to elicit the trust of the users.
To be truly useful, an automated system should be able to make itself be trusted in a manner proportional to the criticality of its application.  Unfortunately, the explainability and, by extension, the \enquote{trustability} of machine learning models are inversely proportional.  There are many examples of modern methods - such as boosted trees, random forests, bagged trees, kernelized-SVMs - that show this tendency, but it is best exemplified by\textit{ deep neural networks} (DNN). Deep Neural Networks are machine learning models constructed by stacking many layers of artificial neurons, these systems are currently state of the art on a variety of tasks but are among the least easily interpretable systems due to the fact that they represent information in an implicit and distributed manner among their network weights.  Some older methods, like decision trees or rule-based methods, are inherently more interpretable due to their simplicity and the fact that they can explicit state their reasoning steps, but are less accurate and flexible than more modern techniques. 

The runaway success obtained by modern Machine Learning in a variety of domains, on a spectrum that goes from engineering to social work, has created the desire to also start applying these methods to mission-critical and traditionally more entrenched fields.  A perfect example of a field exhibiting both these characteristics is that of medicine.  The first successful artificially intelligent systems date back to the 1970s and '80s and were based on \textit{symbolic methods} integrated with \textit{knowledge-bases}.  These systems were by design capable of providing an explanation for their reasoning and were thus accepted by the medical community in an implementation known as \textit{expert systems} that aimed to perform functions similar to those of a human expert.  The deficiency of modern AI methods in being able to provide causal links for their reasoning process has held back their acceptance in the field of medicine, regardless of their superior performance and accuracy.

In a high-stakes domain such as the medical one, it would be unthinkable for a doctor to trust the predictions of an AI system a priori; any decision with profound moral implications - such as prescribing or interrupting the treatment of a patient - would have to first be validated by a human.  The possibility of carrying out this validation and its quality are dependent on the degree of interpretability of the model that made the decision.  Unfortunately, as has been repeated many times, the best performing models are often also the most opaque to inspection.

Explainability is not a necessary condition only for the verification of the system which, \todo{not clear. you were speaking about "validation" before. is this taken as a synonymous of "verification"? if yes, why two words for the same concept? if not, then explain better} as we have just discussed, is a presupposition for it to be applied in mission-critical domains, but also \todo{why?} for the extraction of knowledge from data.  The amount of information that a machine learning model can process is many orders of magnitude greater than that inspectable by any human; this may let a computer spot new patterns in the data that aren't immediately apparent or are latent given only a moderate amount of samples.  Being able to turn this information into \todo{so, what is knowledge? why explainability is necessary to create knowledge? you do not explain this here, imho} new knowledge implies the system having the ability to output human-interpretable symbols that are capable of communicating it in a comprehensible and effective way.

There has recently been much research carried out on trying to explain and extract knowledge from deep neural networks together with attempts to marry the connectionist and symbolic approaches to artificial intelligence - a subfield known as \textit{neuro-symbolic computation} while also reconsidering mixed approaches such as \textit{Bayesian Networks}.
A Bayesian Network is a graphical and computationally efficient way of representing dependencies between random variables.  The graphical component is immediate as in the model each random variable is represented by a node of a Directed Acyclic Graph (DAG), with the edges connecting them standing for their dependencies.  The efficiency stems from the fact that the graph structure imposes a factorisation of the joint probability space and thus lets each variable be calculated using only the values of its parents.

\todo[inline]{again, a lot of nice talk about NN, but then little about BN and thus what you are gonna do}
% !TEX root = thesis-thomas-tiotto.tex

\section{Response}
\textbf{outline the methodology used - 
outline the order of information in the thesis - a roadmap - Maximum 2500 words.}

The work carried out in this thesis concentrates on explainability in the medical domain and presents both a practical part, with the implementation of a Bayesian network-based system focused on \textit{knowledge-extraction}, as defined at the end of the previous section, and a theoretical one, regarding the definition and validation of desiderata for an artificially intelligent system using the aforementioned system.

The implemented system aims at supporting medical decision making through the instauration of a dialogue with the user/domain expert.  To this end, the information implicit in the data is used as basis for a constructive dialogue with the user; this starts with the expert informing the system of which knowledge is certain i.e. a variable's value that has been observed in a specific patient, and continues via a process where the next most probable \textit{(variable, state)} pair is proposed, with the expert having the choice of accepting it or refusing it, if she believes that the variable under examination doesn't adequately explain the accumulated evidence.  Each accepted variable is added to the evidence set, as the system gives priority to the domain expert's judgement.
The result of the dialogue is an \textit{explanation tree} whose nodes represent \textit{(variable, state)} pairs and are organised into branches, depending on the flow of the dialogue; more specifically, there will always be a \textit{main branch} corresponding to the choices of the user and none or more \textit{alternative branches} whose role is to inform the expert of the possible alternative outcomes to his decisions.

This software system was developed and tested tested in collaboration with \textit{Istituto Cantonale di Patologia}, a medical institute in Locarno, Ticino, Switzerland that specialises in the analysis of tissue samples received from hospitals, clinics and private doctors.  Its main activity is to characterise the samples by using \textit{histo-cytopathologic techniques}, with particular focus on the diagnosis of cancer and tumoural diseases in general.

The theoretical part of this thesis aims to understand how an  
 
%%%%
%%%% LITERATURE REVIEW
%%%%  
\chapter{Literature review}\label{chap:literature-review}

\section{Introduction}
Through a review of recent relevant literature, this chapter will clarify the concept of \textit{explainability}, that is central to the field of Explainable AI.
It will then move on to a discussion where a justification will be given for the importance assigned to explainability in our contemporary societies.
The evaluations methods that have been proposed to measure explainability will next be assessed.
The analysis will then focus on an assessment of the previous concepts as applied specifically to Bayesian Networks.
Finally, the recent paper \enquote{Explaining the Most Probable Explanation} by \cite{Butz2018} will be reviewed and connected to the previously analysed notions.

Throughout the chapter, various gaps present in the literature will be identified and assessed; these will be summarised in a coherent fashion in Sec. \ref{sec:literature-review-summary}.

\section{Explainability} \label{sec:explainability}
\cite{Doran2018}, based on a frequency analysis of explanation terms within documents from relevant research communities, highlight how different circles have different approaches to the concept of explainability and how, even within the same group, terms are used interchangeably.
In particular, they note the overloading of the notion of \enquote{explainability} with that of \enquote{interpretation}; a concept that is often defined within the xAI community as necessary for, but distinct from, explainability.
The use of \enquote{interpretable} as signifying the property belonging to a system whose inner workings are accessible can be found, for example, in the recent paper by \cite{gilpin2018explaining}.
In other recent works the two terms are conflated, for example by \cite{mittelstadt2019explaining}, \cite{guidotti2018survey} and in the influential work by \cite{doshi2017towards}.
This seems to prove the point that \cite{Lipton2016} makes in the widely-cited paper \enquote{The Mythos of Model Interpretability}, that \enquote{the task of interpretation appears underspecified. Papers provide diverse and sometimes non-overlapping motivations for interpretability, and offer myriad notions of what attributes render models interpretable.}

Most works, even those that blend the notions of interpretability and explainability, seem to agree on the end-goal that implementing such a concept should have; that is, to \enquote{summarize the reasons for neural network behavior, gain the trust of users, or produce insights about the causes of their decisions} (\cite{gilpin2018explaining}) by being able to \enquote{explain or to present in understandable terms to a human} (\cite{doshi2017towards}).
Where the consensus diverges, in defining what constitutes an explanation and the desiderata that it may have.
\cite{mittelstadt2019explaining} identify, within the literature, two broad classes of interpretability/explainability: \textit{transparency} (also called \textit{ante-hoc} explainablility) and \textit{post-hoc} interpretability.
The former type deals with the internal workings of a system while the latter applies to its external behaviour.
\cite{Lipton2016} identifies three explanations that can make a model transparent: a mechanistic understanding of the workings of the system in its entirety, of the individual components or of the algorithm.
A system may be made post-hoc interpretable by way of, among others, natural language explanations, visualisations or interactive interfaces.
These methods often do not precisely clarify the exact working of a model, but \enquote{they may nonetheless confer useful information for practitioners and end-users of machine learning.}
\cite{Biran2017} notes how the transparent or \textit{white-box} paradigm was sufficient for classic rule-based models but - with the advent of contemporary machine learning models - is no longer useful.
They argue that it is nowadays unreasonable to expect that any domain expert be able to understand a prediction if they are not also a machine learning specialist.
To address this issue, they propose a Natural Language Generation system; that is, a post-hoc explanation in \cite{Lipton2016}'s categorisation.

A widely-recognised feeling, tightly connected with the already recognised lack of shared working definitions, seems to be that researches of explainable AI are ignoring the enormous corpus of existing models in philosophy, psychology, cognitive and social sciences and human-computer interaction.
This feeling of disconnect is echoed by \cite{gilpin2018explaining} who points out how philosophical texts have long debated what constitutes an explanation, by \cite{mittelstadt2019explaining} who explicitly says how \enquote{many different people, be they lawyers, regulators, machine learning specialists, philosophers, or futurologists, are all prepared to agree on the importance of explainable AI [...] very few stop to check what they are agreeing to, and to find out what explainable AI means to other people involved in the discussion}
The fact that explainable AI researchers seem to be intent on \enquote{reinventing the wheel} is stated most strongly by \cite{miller2018explanation}, whose paper \enquote{Explanation in Artificial Intelligence: Insights from the Social Sciences} is based on the premiss that \enquote{most of the research and practice in this area seems to use the researchers' intuitions of what constitutes a `good' explanation} and argues for the adoption of the existing research in the social sciences.
The author's views are well summarised by the position xAI is set to occupy in Fig. \ref{fig:xai-position}.
The feeling is that explainable AI researchers are terming their methods \enquote{explanation} based on purely personal views and are thus building explanations that only work for themselves; in other words, \enquote{the inmates are running the asylum} (\cite{Miller2017}).
The following quote by \cite{guidotti2018survey}, in my view, perfectly sums up the state of the research in the field: \enquote{It is evident that the research activity in this field completely ignored the importance of studying a general and standard formalism for defining an explanation, identifying which are the properties that an explanation should guarantee, e.g., soundness, completeness, compactness and comprehensibility. Concerning this last property, there is no work that seriously addresses the problem of quantifying the grade of comprehensibility of an explanation for humans, although it is of fundamental importance.}

\begin{figure}[htbp]
\centerline{\includegraphics[width=\columnwidth/2]{literature-review/images/xai-position}}
\caption{\cite{miller2018explanation}}
\label{fig:xai-position}
\end{figure}

To remedy to this state of affairs, there have been a number of works, such as \cite{doshi2017towards}'s, that attempted to define what an explanation means and to reach consensus on it.
The most compelling attempt is in the paper \enquote{What Does Explainable AI Really Mean? A New Conceptualization of Perspectives} where \cite{Doran2018} try to synthesise the current state of affairs into a taxonomy of models:
\begin{itemize}
  \item Opaque systems: systems where the mapping from inputs to outputs is entirely inaccessible to the user.
  \item Interpretable systems: systems whose inner workings are inspectable from the outside, but that makes no effort to clarify them.
  \item Comprehensible systems: systems that emit extra information together with their output.
  \item Explainable systems: systems that explicitly output a human-understandable line of reasoning i.e., an explanation.
\end{itemize}
It is recognised, thus implicitly accepting the view that xAI should learn from the social sciences, that comprensibility depends not only on the system's characteristics, but also on the user's ability and knowledge.
Comprensibility and Interpretability are seen as separate concepts as comprehension requires transparency but interpretation does not, as the user may reason over only the emitted extra symbols.
This notion of comprensibility is expanded into that of \textit{real} explainability, that is based on a notion of \enquote{ability to formulate, for the user, a line of \textit{reasoning} that explains the decision making process of a model using \textit{human-understandable features of the input data}.} (italics by the authors).

\section{Importance of Explainability} \label{sec:importance-of-explainability}
As noted by \cite{edwards2018enslaving}, \enquote{businesses and governments are increasingly deploying machine learning (ML) systems to make and support decisions that have a crucial impact on everyday life} so, as \cite{gilpin2018explaining} say, \enquote{it becomes necessary for these mechanisms to explain themselves}.
This feeling of urgency and purpose is echoed throughout the reviewed literature; it seems, that even if researchers and the field as a whole cannot agree on a definition of explainability (as discussed in Sec. \ref{sec:explainability}), there is a keen awareness on the need for models to be explainable.
This intense urge to define explainability and, at the same time, try and create models exhibiting this property, may be counterproductive as the field risks fragmenting into a series of diverging strands, as noted by \cite{abdul2018trends} and visualised in Fig. \ref{fig:xai-citation-network}.

\begin{figure}[htbp]
\centerline{\includegraphics[width=\columnwidth]{literature-review/images/xai-citation-network}}
\caption{\cite{abdul2018trends}}
\label{fig:xai-citation-network}
\end{figure}

There are a myriad of reasons that are brought forth as a justification for the development of explainable AI, and we will review these in the following paragraphs, but it seems timely to start with one in particular: the need introduced by the European Union's broad General Data Protection Regulation (GDPR).
The GDPR was approved by the European Parliament in 2016 and came into effect in 2018.
More than one author cites an urgency to conform to this regulation (for example \cite{doshi2017towards}, \cite{gilpin2018explaining}), most likely referring to Article 22 of the regulation that, supposedly, mandates for a \enquote{right to an explanation} of algorithms.
While algorithmic explainability is undoubtedly a commendable end-goal, it may be the case that this reason, in particular, to strive for it be a false one.
\cite{edwards2018enslaving} posit that Article 22 of the GDPR actually does not contain the publicised right to an explanation but is \enquote{merely a right to stop processing unless a human is introduced to review the decision on challenge} and as the authors point out, there are, nowadays at least, very few systems without a human in the loop.
Secondly, there is no mandate for the \enquote{explanation} to be human-understandable, so the obtained result may actually be no explanation at all.
If the analysis of \cite{edwards2018enslaving} were correct, then the urgency advocated by many researchers on the grounds of conforming to the GDPR would turn out to be based on no valid reason.

A second motive brought forth for the necessity of explainability, for example by \cite{gilpin2018explaining} and \cite{abdul2018trends}, is that comprehensible models are much more likely, or even necessary, to engender users' trust.
While this may very probably be the case, no motives are given for why this should be the purpose of explainable AI and not just a desirable by-product of obtaining explainability. 

Other authors, for example, \cite{doshi2017towards} and \cite{guidotti2018survey}, frame the issue as one of moral necessity.
One need not look far to find examples of ML models displaying covert bias or making decisions we would regard as unethical; a more in-depth investigation would reveal that this was the case even before the popularisation of \enquote{black box} models as the Deep Neural Networks.
\cite{guidotti2018survey} give a reasonably comprehensive list of classic cases that show the risks of not having comprehensible AI.
The oldest of these dates back to the 1970s and 1980s and tells of a system used to screen job applicants that, even though programmed to ignore people's ethnicity, was still seen to discriminate against minorities.
In the same vein but much more recent, the American Military discovered that their computer vision system, that was developed to differentiate between enemy and friendly tanks automatically, had poor accuracy because it had learned to use the background information of the test set photos instead of the pixels representing the actual tanks.
Both these cases exemplify how an algorithm may make \enquote{wrong} inferences based on spurious or latent information that was already in the data set but that no human could have imagined being relevant.
Other failures epitomise how a model may learn our own social biases; for example, a recent Princeton study (\cite{caliskan2017semantics}) proved how models trained on web text corpora showed marked biases (towards race, gender ...) that reflected our the ones present in our own society.
Because of their findings, the authors went as far as to suggest that transparency would not be enough to uproot biases, as the very \textit{semantics} reflects biases latent in our culture. 

The driving motivation and sense of urgency present in all of the reviewed literature is quite certainly tied to the renewed interest and applicability of AI.
\cite{Preece2018} claim that the interest for explainability is naturally linked to the interest in AI itself; if this were true, then it would confirm that a need for transparency is implicit in the field itself.
This would validate the assertion made by \cite{doshi2017towards}, that the need for an explanation stems from an incompleteness in the formalisation.
\cite{Lipton2016} also acknowledges this and states that \enquote{the demand for interpretability arises when there is a mismatch between the formal objectives of supervised learning and the real world costs in a deployment setting}.
In the presence of such uncertainty, rendering the resulting model open to inspection would make the \enquote{gaps in problem formalization visible to us} and thus enable us to apply our best human judgement to evaluate them and their consequences (\cite{doshi2017towards}).

\section{Evaluation of Explainability}
As concluded in Sec. \ref{sec:importance-of-explainability}, the fact that ML models operate on incomplete assumptions makes it a necessity to have some form of evaluation of their performance.
As \cite{Lipton2016} states, \enquote{it turns out that many situations arise when our real world objectives are difficult to encode as simple real-valued functions} and this could lead to evident difficulties in optimising with respect to soft, but of paramount importance, concepts such as ethics and legality.
Being able to evaluate an automated explanation lets us \enquote{serve those objectives that we deem necessary but struggle to model formally}.

Unfortunately, the finding outlined in Sec. \ref{sec:explainability} that there is no consensus on the definition of explainability also necessarily entails that there is no agreed-upon methodology to evaluate such a property.
\cite{doshi2017towards} note as much when they comment \enquote{unfortunately, there is little consensus on what interpretability in machine learning is and how to evaluate it for benchmarking}.
This makes perfect sense because trying to evaluate something without first having defined it, is worse than trying to hit a moving target.
Once again, the feeling among many authors is that \enquote{inmates are running the asylum}.

Some authors have tried to put some order in the barrage of methods; \cite{doshi2017towards} provide one of the most compelling attempts.
 \cite{doshi2017towards} set out to outline a taxonomy, having noted a \enquote{lack of rigour} and how current interpretability approaches usually fall into two categories: interpretability in the context of an application and interpretability via a quantifiable proxy.
 The former approach assumes that \enquote{if the system is useful in either a practical application or a simplified version of it, then it must be somehow interpretable}; the latter sees researchers claim that a model class is interpretable and then present algorithms to optimise within that class.
 In their words, both classes rely on a notion of \enquote{you'll know it when you see it}.
 The taxonomy the authors lay out is shown in Fig. \ref{fig:xai-taxonomy} and borrows from methods already standard in human-computer interaction and visualisation; the guiding ideal is that \enquote{evaluation of applied work should demonstrate success in the application} and thus the best sort of evaluation is the one that involves humans the most.
 Thus, at the lowest level, they pose \enquote{Functionally-grounded Evaluations} that require no human in the loop and evaluate the quality of an explanation using some proxy measure; the advantage is the low cost, but the tradeoff is a lack of specificity.
 A proxy measure that has already been human-validated, for example a decision tree, a set of rules or a linear model (\cite{guidotti2018survey}) as regards its explainability, may be an appropriate measure for more exotic systems.
 The second level of evaluation, \enquote{Human-grounded Evaluation}, involves humans, albeit not domain expert ones; this kind of setup enables the testing of more general notions of explainability.
Finally, \enquote{Application-grounded Evaluation} is taken as the gold standard to demonstrate success in the context of an application; the authors claim that there is no better way to evaluate the quality of an explanation that having a domain expert test it in the context of a real task.
In their words, \enquote{the best way to show that the model works is to evaluate it with respect to the task}: \enquote{for example, a visualization for correcting segmentations from microscopy data would be evaluated via user studies on segmentation on the target image task; a homework-hint system is evaluated on whether the student achieves better post-test performance.  Specifically, we evaluate the quality of an explanation in the context of its end-task, such as whether it results in better identification of errors, new facts, or less discrimination}.

\begin{figure}[htbp]
\centerline{\includegraphics[width=\columnwidth]{literature-review/images/xai-taxonomy}}
\caption{\cite{doshi2017towards}}
\label{fig:xai-taxonomy}
\end{figure}

\cite{guidotti2018survey} outline a taxonomy developed along a different axis; specifically they identify: methods to explain black box models, methods to explain black box outcomes, methods to inspect black boxes and methods to design transparent boxes.
An essential factor that the authors identify is that evaluation is a graded notion, as also noted by \cite{gilpin2018explaining}: different users, with different expertise and background, may rate the quality of explanations differently.
Another point they make that is quite novel is that time should also be part of an evaluation; depending on the available time a user may prefer a more straightfoward or more elaborate explanation.
They end by noting that very few works take user expertise and the time taken to understand the proposed explanation into account.

The neglect of the human side of explanations is also lamented by \cite{abdul2018trends} who see researchers focusing on creating mathematically-explainable models at the expense of ones that are usable and practical in real-world situations.
Again, the underlying issue seems to be that xAI researchers are unaware or ignoring the sizeable corpus of research available in cognitive psychology, human-computer interfaces and philosophy.
The author sees the opportunity for HCI to bridge the gap between models and users by way of an interactive approach, as opposed to the mainstream static explanations being proposed in the literature.
An interactive explanation may take the form of a dialogue or of various visualisation techniques; the defining characteristic of such a mode is that it lets the user freely explore the system's behaviour.
\cite{guidotti2018survey}, while discussing the types of data used in ML models, lend credence to the goodness of this output modality with the statement that \enquote{other forms of data which are very common in daily human life are images and texts. They are perhaps for human brain even more easily understandable than tables}.

\cite{guidotti2018survey}, though, take the view that evaluation of model \textit{complexity} should be equated to its \textit{comprehensibility}, that is not something that often appears in the relevant literature.
Basically, the authors are advocating for the use of complexity as a proxy model for explainability, if we are framing the issue using the taxonomy proposed by \cite{doshi2017towards} (Fig. \ref{fig:xai-taxonomy}).
This may very well be a valid approach, but there is no supporting evidence for it in the paper itself.

A useful reference for how to set up an experiment falling into the class of either Human-grounded or Application-grounded Evaluation can be found in \cite{stumpf2009interacting}'s \enquote{Interacting meaningfully with machine learning systems: Three experiments} 2009 paper.
In this work the authors set up \enquote{three experiments to understand the potential for rich interactions between users and machine learning systems}; the first, and most relevant, was a think-aloud study that investigated \enquote{how machine learning systems should explain themselves to end users, and what kinds of improvement feedback end users might give to the machine learning systems}.
These studies are interesting as a blueprint for future human-centred evaluations, of the type whose absence is being lamented by many authors.

\cite{mittelstadt2019explaining} summarise the existing critiques to offer a clear and direct evaluation of the field of Explainable AI as a whole, when they state that \enquote{no matter the approach taken in xAI, reflexivity \textit{[taking account of itself or of the effect of the personality or presence of the researcher on what is being investigated], my note} is needed to ensure the community actually works towards its normative and practical goals to render models holistically transparent or provide high-quality post-hoc interpretations of model behaviour. Critical questions must be repeatedly asked and answered. For example, will the methods developed make machine learning models more interpretable? More trustworthy to users? More accountable? And to whom will explanations be accessible, comprehensible, and useful? Answering such questions requires considering the methods developed in xAI in the context of prior work in fields addressing such normative and social questions. Local and approximation models may in fact resemble existing, well-known approaches to explanations in the `explanation sciences', which would provide insight}.
They then conclude by stating that \enquote{xAI generally avoids the challenges of testing and validating approximation models, or fully characterising their domain}.
From the review of the current state-of-the-art carried out in this section and Sec. \ref{sec:explainability} and \ref{sec:importance-of-explainability}, these could both be seen as entirely valid criticisms.
It really seems that the field of xAI as a whole should try and reposition itself as suggested in Fig. \ref{fig:xai-position} and not try to build methods from first principles, many of which may be outside the domain of expertise of the researchers in question.

\section{Explainability of Bayesian Networks} \label{sec:explainability-in-bayesian-networks}
Bayesian Networks have enjoyed widespread appeal as a Machine Learning method, especially inside the medical domain.
This may be because the formalism (introduced in detail in Sec. \ref{sec:bayesiannetworks}) \enquote{offers a natural way to represent the uncertainties involved in medicine when dealing
with diagnosis, treatment selection, planning, and prediction of prognosis. This is due to the fact that the influences and probabilistic interactions among variables can be described readily in a BN} (\cite{Lucas2001}); that is, even if the BN model is \textit{complete}, in the sense that every possible probabilistic statement can be computed in it, it also is easy to combine multiple variables of interest into composite questions.
Another attractive feature of BNs is their relatedness to the class of \textit{Causal Networks} that were popularised by the groundbreaking work of \cite{Pearl1988}; for all intents and purposes, a Causal Network is simply a Bayesian Network where all the relationships represent a causal effect.
Nonetheless, BNs are not considered as inherently interpretable by the literature, and thus, a series of methods were developed to address this shortcoming.
\cite{timmer2015explaining} note this in the introduction to their paper by stating that \enquote{for non-statistical experts, however, Bayesian networks may be hard to interpret. Especially since the inner workings of Bayesian networks are complicated they may appear as black box models}, \enquote{the interpretation of BNs is a difficult task, especially for domain experts who are not trained in probabilistic reasoning}.

The best overview of the state of explainability in BNs is given by \cite{lacave2002review} in the paper \enquote{A review of explanation methods for Bayesian networks}; here the authors identify various classification criteria for an explanation given by a BN:
\begin{itemize}
  \item \textit{description} vs. \textit{comprehension}: the former consists in displaying the data set or providing further details regarding the output, the latter attempts to guide the user in understanding the model's conclusions.
  \item \textit{micro-level} vs. \textit{macro-level}: detailed description of how a single node is affected vs. showing the main lines of reasoning.
  \item \textit{verbal} vs. \textit{graphical}: \enquote{the most direct and intuitive way of showing the information embodied in a Bayesian network is to display the corresponding graph}.
  When presenting probabilities, \cite{henrion1990qualtitative} strongly suggest that these be \enquote{linguistic probabilities} i.e., for the quantitative probabilities inherent in the model to be converted to a qualitative equivalent.
  This is validated by research showing that linguistic expressions of probability are better understood than the equivalent numerical representation. 
  Some of the many models surveyed in the paper user colours, shading and line thickness to represent the salience of links and nodes.
\end{itemize}
The authors also identify the three components of BNs that need to be explained: the knowledge base, the reasoning process and the evidence propagated.
The first of these \enquote{consists of determining which values of the unobserved variables justify the available evidence} and is, in general, done by finding the solution to the Most Probable Explanation problem (defined in Subsec. \ref{subsec:bnupdating}).
The explanation of the model is considered a static explanation (as opposed to dynamic ones, that will shortly be covered) and is simply the process of verbally or graphically displaying the information already present in the data.
The final element to explain, that is what would most commonly be called an explanation in xAI circles, is the reasoning behind the model's outputs; a system may accomplish this by providing a justification for its outputs, for the results it did not give or via hypothetical reasoning.
The first of these is maybe the most important, because it is paramount for any system, not just a BN, to be able to explain the reasons behind its outputs; returning to the medical setting, it was seen that physicians, in particular, are very reluctant to accept the advice of a machine if they can not understand how it was obtained.
A BN, unlike other ML systems, can also innately exhibit evidence for why it did not provide the output expected by the user and can also reason \textit{counterfactually} i.e. provide alternative outputs.
These last two capabilities are particularly important, from an explainability perspective, in light of \cite{miller2018explanation}'s findings regarding the nature of explanations.

\cite{miller2018explanation}'s paper \enquote{Explanation in Artificial Intelligence: Insights from the Social Sciences} investigates what constitutes an explanation from a psychological perspective; the conclusions are that explanations possess four primary characteristics:
\begin{itemize}
  \item explanations are \textit{contrastive}; that is people do not ask why an event happened by why another event did not happen instead.  
  A Bayesian Network, as noted in the previous paragraph, is capable of modelling counterfactuals which enables them to give contrastive reasons naturally.
  \item explanations are \textit{selected}; people expect an explanation to be selected based on some cognitive bias; they do not expect a complete recount of all causes of an event.
  A BN has the ability to combine an output from its constituent variables in a flexible manner. 
  \item to people, probabilities are not as important and not as well understood as causal relationships.
  BNs, as already mentioned, are closely related to \textit{graphical causal models}, so their explanations have the possibility of being based on causal grounds (\cite{Lipton2016}, \cite{rani2006empirical}).
  \item explanations are \textit{social}; that is they involve an \textit{explainer} and an \textit{explainee}.  
  This is recognised in the most important work on the social aspects of conversation, \cite{Hilton1990}'s \enquote{Conversational processes and causal explanation}, that supports the view that an \textit{explanation} is a \textit{conversation}. 
  A dialogue is an example of a \textit{dynamical explanation}, in the framework set out by \cite{lacave2002review}; these authors also recognise that an explanation \enquote{always means explaining something to somebody} and that thus \enquote{one of the key features of an effective explanation is the ability to address each user's specific needs and expectations, which primarily depends on the knowledge he/she has}.
  	  So \enquote{In the case of a Bayesian network, the explanation generated for a user that is familiar with the concepts of prevalence, prior/posterior odds and likelihood ratios should be very different from the explanation generated for a user who has never heard about them}.
  	  \cite{lacave2002review} again recognise that explainability is a graded notion but go further and define the concept of \textit{fixed user model}, noting that practically all explainable BN systems have made this assumption and ignored the possibility of users having varying knowledge.
  	  Though, some of the systems surveyed in the paper make a step in that direction by incorporating an \textit{importance threshold} mechanism that would let the user only display certain items; this enables these systems to display varying levels of detail without having defined a user model.
\end{itemize}

Bayesian Networks, in the taxonomy set forth by \cite{Doran2018} and discussed in Sec. \ref{sec:explainability}, would most probably fall into the class of \textit{Interpretable systems}, as do many other ML models.
Though, based on the review of the literature, it could be suggested, on quite a strong basis, that Bayesian Networks are better equipped than other Machine Learning models to provide a meaningful explanation to humans.
This could be claimed because it is quite widely believed that our brains and thus our psychology are near-optimal problem-solvers and thus approximate optimal Bayesian solutions.
A standard view in the fields of psychology and neuroscience, as noted by \cite{Bowers2012}, is that our brain processes approximate the \textit{rational player} as presented in the Dutch Book Argument (see Ch.7 of \cite{anand2009handbook}) and are thus Bayesian in nature.
It is of worth to also note that this view has recently been challenged, for example by \cite{Bowers2012}.
Even if our brains were not inherently Bayesian, the characteristics of Bayesian Networks make them more capable than other ML models in being able to generate explanations tailored to our cognitive processes and psychology, as discussed in the above bullet points.

\section{\enquote{Explaining the Most Probable Explanation}} \label{sec:explaining-the-most-probable-explanation}
The paper \enquote{Explaining the Most Probable Explanation} by \citet{Butz2018} places itself in the literature concerned with the explainability of Bayesian networks.
In particular, taking the classification proposed by \citet{lacave2002review} presented in Section \ref{sec:explainability-in-bayesian-networks}, it attempts to define a \textit{linguistic explanation} of the \textit{evidence} and of the \textit{reasoning}.
It differs from the previous attempt to define the explanation of the \textit{evidence} given by \citet{lacave2002review} and in other works, in that the paper is not concerned with finding the most probable assignment of variables that would explain the given evidence but, rather, the inverse problem.
By starting with the evidence and finding a maximally probable configuration, the authors hope \enquote{to look at the complete scenario to get an overview before deciding which variables should be focused on}; i.e., the goal appears to be to give the user an overview of the situation.  

The initial claim of the paper is that BNs are still difficult to interpret for domain experts, even though these models provide a graphical structure to the \textit{knowledge base}.
The examples brought to justify the claim are that edges in the graph do not necessarily represent causal dependencies and that d-separation (Definition \ref{def:d-separation}) may be confusing.
The authors plan to address this claim by constructing a \textit{dialogue} with the user and thus to continue in the long tradition of dialogical approaches to explaining BNs, many of which are presented in \citep{lacave2002review}.

The defining characteristic of their approach is that the domain expert is able to \enquote{argue} with the MPE and investigate alternative explanations.
The complete methodology, executed over three steps, is shown in Figure \ref{fig:butz-methodology}.
The first step is the construction of the \enquote{knowledge base}, which is nothing else than a probability tree representing a \enquote{chain of deduction} constructed following the strongest probabilistic dependencies between variables in the BN.
Such a \textit{knowledge base} is convenient because the document plan for the Natural Language Generation step is directly derived from it.
One issue that is immediately apparent is that this greedy approach does not \enquote{generate the MPE solution} as the authors claim.
This does not discredit the argumentative method as a whole, as \textit{it is not necessary for the user to be arguing the MPE to derive a good explanation}; this ties into one of the main findings in the previous sections that many xAI researchers are only focusing on one half of the explanation.
A good explanation is not given only by its formal properties but, most importantly, by how well it acts as an interface between the real \textit{user} and the model.
This is what \citet{abdul2018trends} mean when they lament that \enquote{despite their mathematical rigour, these works \textit{[referring to the existing explainability methods]} suffer from a lack of usability, practical interpretability and efficacy on real users}.

\begin{figure}[htbp]
\centerline{\includegraphics[width=\textwidth]{literature-review/images/butz-methodology}}
\caption{Overview of methodology followed by \citet{Butz2018}.}
\label{fig:butz-methodology}
\end{figure}

The document plan for the argumentation follows the same chain of strongest dependencies constructed in the \textit{knowledge base} until the expert disagrees; at that point, the user is presented with an alternative \enquote{MPE}.
An example of how the document plan may look after interaction with the user is shown in Figure \ref{fig:butz-tree}.
All the natural language phrasing is generated via boilerplates that take care of realising both the micro-planning phase and the generation of the text.

\begin{figure}[htbp]
\centerline{\includegraphics[width=\textwidth]{literature-review/images/butz-tree}}
\caption{\textit{Document plan} generated from the \textit{probability tree} \citep{Butz2018}.}
\label{fig:butz-tree}
\end{figure}

The authors recognise that such chains of deduction could become long and cognitively overloading in the case of larger BNs, as every variable in the tree is explained by all its ancestors.
A solution they propose is that of \textit{pruning} the probability tree by excluding d-separated nodes and those under a certain threshold of significance.
They also adapt some methods from literature to perform \textit{conflict analysis} i.e., only variables that contribute positively to the explanation are maintained in the document plan.

On the whole, \citet{Butz2018} offer a compelling explanation method for BNs by building on an established tradition of enabling explainability through dialogue.
The work, though, takes some methodological missteps and also continues the \enquote{sin} of not validating its claims on real users, which is one of the primary gaps in the xAI field, as identified in the previous sections.

\section{Summary} \label{sec:literature-review-summary}
The findings outlined in this chapter refer to the concept of explainability of Machine Learning models, to its importance, to ways of evaluating it and to explainability in the specific case of Bayesian Networks.
The concept of explainability is central to the field of Explainable AI (xAI), whose goal is to make Machine Learning systems \enquote{interpretable}.
Explainability can briefly be defined as the property of a system that is able to \enquote{explain or to present in understandable terms to a human} its outputs.
Two main classes of explainable models have been identified by \cite{mittelstadt2019explaining}: ante-hoc or transparent and post-hoc interpretable ones; the formers are inherently inspectable in their inner workings while the latters are made understandable by way of extra techniques.
These two classes have also been refined by \cite{doshi2017towards} into a four-tier taxonomy consisting of: opaque, interpretable, comprehensible and explainable systems.
An opaque system, also know as a \enquote{black-box model}, is one whose inner workings are not inspectable from the outside; an interpretable system corresponds to an ante-hoc interpretable one; a comprehensible one emits extra information together with its output; an explainable system explicitly outputs a human-understandable line of reasoning aimed at clarifying its workings.

Explainability has become a central concept to the field of AI as a whole; as ML models take over more and more functions in our societies, the pressure for them to be able to explain their decisions is increased accordingly.
The General Data Protection Regulation (GDPR) that became effective in 2018 was viewed by many as increasing the societal pressure to make systems explainable; many may have been mistaken as regards the actual rules mandated by the regulation - that aren't really prescribing a broad \enquote{right to an explanation} (\cite{edwards2018enslaving}) - but nonetheless the feeling of urgency is sure to increase the focus of both researchers and laypeople.
In general, explainability is framed as an issue of moral necessity as it easy to find a long series of situations where ML models displayed covert bias or what we would regard as bad moral judgement.

There are all manner of ways to measure explainability and these can be classified into a three-layer taxonomy (\cite{doshi2017towards}) based on the assumption that the best type of evaluation is the one that most involves humans.
The three classes are Functionally-grounded Evaluation, Human-grounded Evaluation and Application-grounded Evaluation, ordered from the one least involving real humans to the one where the presence of the human-in-the-loop is greatest.
As the involvement of humans in evaluating models' explanations increases, so does the cost of such an experiment and its specificity, as the highest evaluation level necessarily entails the collaboration of domain-experts on specific tasks.
A parallel taxonomy identifies: methods to explain black box models, methods to explain black box outcomes, methods to inspect black boxes and methods to design transparent boxes.
An overarching notion that has been stressed is that \textit{explainability is a graded notion} that depends on the knowledge and expertise of the particular user: different users, with different expertises and backgrounds, may rate the quality of a same explanation very differently.

Bayesian Networks (BNs) have enjoyed widespread appeal in mission-critical domains like that of medicine and thus the drive to develop methods to explain their outputs has always been strong.
BNs have three main elements that necessitate an explanation (\cite{lacave2002review}): the knowledge base, the reasoning process and the evidence propagated.
Explaining the first \enquote{consists of determining which values of the unobserved variables justify the available evidence} and is done by solving the Most Probable Explanation (MPE) problem.
For the second, a static explanation of the BN is achieved by displaying it graphically or verbally.
The last element is explained by showing the reasoning that brought the BN to give the outputs it did that can be achieved by providing a justification for its outputs, for the results it did not give or via hypothetical reasoning.
The fact that BNs are able to naturally support counterfactual reasoning, combine single variables into composite outputs and model causality puts them at an advantage compared to other ML systems when generating an effective explanation for a user.
This is because the capabilities of a BN enable it to generate explanations that are uniquely suited to out psychological biases and expectations of what an explanation should entail.

Some of the main gaps that were found during the review of the literature relate to how there is still a great confusion in the field of xAI regarding what an explanation really is and thus what constitutes a good instance of it.
There is also a prevalent methodological confusion, as different authors use terms in incongruous ways, for example sometimes \textit{intepretation} is taken to mean \textit{explanation} while in other cases they refer to different concepts, for example in the taxonomy of interpretable systems proposed by \cite{doshi2017towards}.
This naturally makes it difficult for the field to converge onto methods to evaluate such explanations and this is reflected in the barrage of methods present in the literature, each one focused only on a particular system or instance of model.
This confusion is exacerbated by a seeming lack of interest or awareness of xAI researchers for the sizeable corpus of work in psychology, philosophy, social sciences, neuroscience and human computer interaction that has already investigated the nature of explanations, what desiderata they may possess and which are most effective.
In fact, the great majority of proposed approaches is only focused on proving formal explainability and neglects the human side that is naturally present in any explanation; there has been little work carried out to validate approaches in real settings with real domain experts so many explainability methods are substantiated only at theoretical level.
The underlying issue that I have seen run transversely across the various concepts investigated in this chapter is best summarised by the idea that \enquote{inmates are running the asylum}, meaning that individual researchers are claiming that their models are interpretable referring only to their own personal views and biases and not to established literature and methods.
It would be hard for them to do otherwise, as the field of xAI seems, at present, to be a collection of diverging strands without a comprehensive program able to help it converge onto its stated goal: to make Machine Learning systems understandable by their users and thus increase their social utility and acceptance.

%%%%
%%%% MATHEMATICAL BACKGROUND
%%%%
\chapter{Mathematical Background}\label{chap:mathematical-background}
\section{Introduction}
This chapter will arrive to give a formal definition of Bayesian Networks, a class of Probabilistic Graphical Models that are used to represent systems under conditions of uncertainty.
Once the formalism has been defined, an overview of structure learning algorithms and the notions of Conditional Probability and Maximum a Posteriori Query are given.
To arrive at this, it will first introduce a series of basic concepts from Probability, Information and Graph Theory.
Some of these are not needed for the description of the BN model, but will be useful as a mathematical reference for the work carried out in later chapters of this thesis.

\section{Probability Theory} \label{sec:probability-theory}
We will be dealing with \textit{standard probability} so random variables and probability measures will always be real-valued.
We will also, in the work carried out in this thesis, only be considering the case of random variables that can assume a finite number of possible values/states.
We will refer to these variables as \textit{categorical} to indicate that there is no natural ordering among their states.

\subsection{Random Variables} \label{subsec:random-variables}
\begin{definition}[Event]
	Given $\mathcal{S}$ the space of all possible outcomes of interest, an event $\sigma$ is a subset of $\mathcal{S}$: $\sigma \subseteq \mathcal{S}$.
	
	$\mathcal{F} \subseteq 2^{\mathcal{S}}$ is the set of all events that are under consideration.
	
	Two events $\sigma$ and $\tau$ are called disjoint when $\sigma \cap \tau = \emptyset$.
\end{definition}

\begin{definition}[Random Variable]
	A random variable $X$ is a function $X: \mathcal{S} \rightarrow \mathcal{X} \subseteq \mathbb{R}$ that associates every outcome $s \in \mathcal{S}$ with a value.
\end{definition}

\textit{Random variables} are a way of bringing to the fore the attributes of interest of events while dealing with them in a clean, mathematical way.
The values that a random variable can take are a function of the events in sample space $\mathcal{S}$, with each of these having a value assigned by the random variable function.

\begin{definition}[Probability Measure] \label{def:probability-measure}
	Given a sample space $\mathcal{S}$ and events $\mathcal{F}$, a discrete probability measure $\mathbb{P}$ is a function $\mathbb{P}: \mathcal{F} \rightarrow [0,1]$ that assigns a probability value to every event. 
	In the discrete case all subsets of $\mathcal{S}$ can be treated as events thus $\mathcal{F}$ is the power set of $\mathcal{S}$.
To be a valid probability measure, $\mathbb{P}$ must satisfy:
\begin{itemize}
	\item $\mathbb{P}(\mathcal{S}) = 1$;
	\item If events $\sigma$ and $\tau$ are disjoint then $\mathbb{P}(\sigma \cup \tau)=\mathbb{P}(\sigma)+\mathbb{P}(\tau)$.
\end{itemize}
\end{definition}
Each event $\sigma \in \mathcal{F}$ must have a probability $\mathbb{P}(\sigma) \in [0,1]$ and the sum of all these must equal $1$. 
An event with $\mathbb{P}(\sigma) = 0$ is deemed \textit{impossible} while one with $\mathbb{P}(\sigma) = 1$ is \textit{certain}.

\begin{definition}[Probability Mass Function]\label{def:mass-function}
	A probability mass function of a discrete random variable $X$ is a function $f_X: \mathcal{X} \rightarrow [0,1]$ defined, using a probability measure $\mathbb{P}$, as:
\begin{equation*}
	f_X(x) = \mathbb{P}(\{s \in \mathcal{S} : X(s)=x\}) \,,
\end{equation*}
and thus assigns a probability to every value $x \in \mathcal{X}$ in the domain of $X$.
\end{definition}
The probability mass function returns the probability of a random variable $X$ taking on exactly its value $x$.
This probability is the size of the subset of the event space $\mathcal{S}$ whose events $s$ are mapped to $x$ by the random variable function $X$.

Every random variable has a probability distribution induced by the cardinality of the subsets of its values; in the case of discrete one, such a distribution is \textit{multinomial}.

Often, in the context of random variables the probability distribution $f_X$ is called the \textit{marginal of $X$} and is usually contrasted with the notion of \textit{joint probability distribution}.
\begin{definition}[Joint Probability Mass Function]
	The joint probability mass function of discrete random variables $X$ and $Y$ is a function $f_{XY} : \mathcal{X} \times \mathcal{Y} \rightarrow [0,1]$ defined as:
	\begin{equation*}
		f_{XY}(x,y) = \mathbb{P}( \{s \in \mathcal{S} : X(s)=x \} \cap \{ s \in \mathcal{S} : Y(s)=y\} ) \,,
	\end{equation*}
	and thus assigns a probability to every tuple $(x,y)$ with $x \in \mathcal{X}$, $y \in \mathcal{Y}$.
\end{definition}

In what follows, we will sometimes refer to the marginal probability $f_X$ of $X$ as $\mathbb{P}(X)$, to $f_X(x)$ as $\mathbb{P}(X = x)$, to joint probability $f_{XY}$ of $X$ and $Y$ as $\mathbb{P}(X,Y)$, and to $f_{XY}(x,y)$ as $\mathbb{P}(X=x,Y=y)$ as the only random variables we will be dealing with will be discrete.
Notice that the notation $(E = e)$ is also often overloaded to signify an assignment of values to a set of random variables $E$; in this case what is meant is that every variable in the set $E = {X_1,...,X_k}$ assumes a certain value from its own domain. 
We will denote sets in bold so $\boldsymbol{E} = \boldsymbol{e}$ stands to mean that every variable $E$ in the set of random variables $\boldsymbol{E}$ assumes a value $e$ from its own domain. 
Finally, recall that the set of values that $X$ can take is denoted by the cursive $\mathcal{X}$.

\subsection{Probability interpretations} \label{subsec:probability-interpretations}
There are two main views through which to interpret the probability of an event: the \textit{frequentist} and the \textit{subjectivist/Bayesian} one.

The former views the probability of an event as the expected ratio of times it would occur over a great number of trials.
That is, the probability of an event is seen as the \textit{limiting frequency} of a repeatable event.
So, for example, the probability of observing heads when tossing a coin is said to be $0.5$ because over repeated throws heads was observed half the time.

The other view is the subjectivist or \textit{Bayesian} one (from the 18th century mathematician Thomas Bayes) in which probabilities are instead viewed as the \textit{subjective} degree of belief attributable to the manifestation of an event.
In this interpretation, stating that a coin has probability of heads of $0.5$ simply means that the person making the claim personally believes that the chances of seeing heads or tails are the same.
This is useful in that it enables the characterisation of certain events that haven't yet come about or that are liable to happen only once or a small number of times (that is, they are not repeatable).

Philosophically, Bayesian inference assigns a probability to a hypothesis (a \textit{prior}) while the frequentist method tests a raw hypothesis empirically before assigning it any probability.
As Bayesian inference naturally embraces and deals with uncertainty, it is an enormously useful tool to model and reason about the real, stochastic world we live in.

From the Bayesian point of view, we would consider the probability of a state of a random variable as simply representing the subjective degree of belief we would have over a set of outcomes we believed could manifest themselves.

\subsection{Conditional Probabilities} \label{subsec:conditional-probabilities}
\begin{definition}[Conditional Probability] \label{def:conditional-probability}
	The conditional probability mass function of random variable $Y$ given $X=x$, $x \in \mathcal{X}$, $y \in \mathcal{Y}$ is:
\begin{equation*}
\mathbb{P}(Y=y \mid X=x) = \frac{\mathbb{P}(X=x,Y=y)}{\mathbb{P}(X=x)} \,.
\end{equation*}
To be defined, it must be that $\mathbb{P}(X=x) > 0$.
\end{definition}

Definition \ref{def:conditional-probability} can easily be manipulated in order to obtain another basic result, called the \textit{chain rule of conditional probabilities}:
\begin{equation} \label{eq:chainrule}
	\mathbb{P}(Y=y,X=x) = \mathbb{P}(Y=y \mid X=x) \mathbb{P}(X=x) \,.
\end{equation}
Equation \ref{eq:chainrule} can be generalised to any number of variables:
\begin{align} \label{eq:chainrule-multiple}
\begin{split}
	\mathbb{P}(X_1=x_1, \ldots , X_n=x_n ) = & \mathbb{P}(X_n=x_n \mid X_1=x_1, \ldots, X_{n-1}=x_{n-1}) \times \\
	&  \ldots   \\
	 &\times \mathbb{P}(X_1=x_1 \mid X_2=x_2 ) \; \times \\
	 &\times \mathbb{P}(X_1=x_1) \,.
\end{split}
\end{align}
Intuitively, it means that we can decompose joint probabilities as products of conditional probabilities.  

Another immediate, and crucial, consequence of Definition \ref{def:conditional-probability} is known as \textit{Bayes' Theorem}, which lets us calculate the revised probability of an event given new knowledge regarding another event.
\begin{theorem}[Bayes' Theorem] \label{th:bayes-theorem}
	Given random variables $X$, $Y$ and the events $X=x$, $Y=y$, $\mathbb{P}(Y=y) > 0$, it holds that:
	\begin{equation*}
		\mathbb{P}(X=x \mid Y=y)=\frac{\mathbb{P}(Y=y \mid X=x) \mathbb{P}(X=x)}{\mathbb{P}(Y=y)} \,.
	\end{equation*}
\end{theorem}
Intuitively, this is a process of \textit{belief revision} as the belief in event $X=x$ is revised by the new knowledge that $Y=y$. 

\subsection{Independence} \label{subsec:independence}
\begin{definition}[Random Variables Independence]
	Two random variables $X$ and $Y$ with domains $\mathcal{X}$ and $\mathcal{Y}$ are independent when their joint probability mass $\mathbb{P}(X,Y)$ is equal to the product of their probability densities:
	\begin{equation*}
		\mathbb{P}(X=x,Y=y) =  \mathbb{P}(X=x) \times \mathbb{P}(Y=y) \quad \forall x \in \mathcal{X}, \forall y \in \mathcal{Y} \,.
	\end{equation*}
\end{definition}
In the real world it is hard or even impossible - if we consider Nature being based on chaos theory when viewed at a fine-enough level - to find two such perfectly non-interacting events.
Thus, a more useful concept is that of \textit{conditional independence} where two previously dependent events become independent when conditioned on a third one

\begin{definition}[Random Variables Conditional Independence]
	Two random variables $X$ and $Y$ with domains $\mathcal{X}$ and $\mathcal{Y}$ are conditionally independent on a third random variable $Z$ with domain $\mathcal{Z}$ when their probability densities conditioned on $Z$ are independent.
	That is, when the joint probability mass function conditioned on $Z$ is equal to the product of the conditional probability mass functions:
	\begin{equation*}
		\mathbb{P}(X=x,Y=y \mid Z=z) = \mathbb{P}(X=x \mid Z=z) \times \mathbb{P}(Y=y \mid Z=z) \quad \forall x \in \mathcal{X}, \forall y \in \mathcal{Y}, \forall z \in \mathcal{Z} \,.
	\end{equation*}
\end{definition}
Intuitively, this means that knowing any value of $Z$ makes the probability distributions of $X$ and $Y$ independent.
\section{Information Theory} \label{sec:information-theory}
The birth of the field of \textit{information theory} is usually traced back to the seminal paper \enquote{A Mathematical Theory of Communication} (\cite{Shannon1949}) where Claude Shannon set the mathematical basis for the quantification of the amount of \textit{information} transmissible over a noisy channel. 
In his words \enquote{The fundamental problem of communication is that of reproducing at one point, either exactly or approximately, a message selected at another point.}
The concepts of field are broad enough to have influenced practically every other scientific discipline and deep enough to have enabled the \enquote{digital age}, for example by enabling the creation of ever more complicated coding schemes for the compression, reconstruction and obfuscation of digital data.

\subsection{Entropy} \label{subsec:entropy}
In classical mechanical statistics, entropy can be seen as a measure of the uncertainty, or randomness, of a physical system.  
This concept was reapplied by Shannon to measure the amount of randomness in a random variable.
\begin{definition}
	Given a random variable $X$ with probability distribution $\mathbb{P}(X)$, its entropy $\Eta(X)$ is defined as the expected amount of information content carried by $X$ (\cite{Schneider2005}):
\begin{equation} \label{eq:entropy}
	\Eta(X) = \mathbb{E}(I(X)) = \mathbb{E}(-\log (\mathbb{P}(X)) = -\sum_{i=1}^{n} \mathrm{P}\left(x_{i}\right) \log _{b} \mathrm{P}\left(x_{i}\right)
\end{equation}
\end{definition}
The base $b$ of the logarithm defines the unit of measure.  Shannon used $b=2$ as he was dealing with the transmission of digital, binary-coded data; in this case the unit of measure are $bits$.

The simplest example of how information entropy characterises a random variable $X$, is in imagining $X$ to model a coin and the task being to predict the probability of the outcome of a throw being heads.
If the coin is fair, we will not be any more surprised to see the outcome being heads than tails; the entropy is maximum as there is maximum uncertainty regarding the outcome.
However, if the coin is not fair and tails is more probable the we will be more surprised than not to see the outcome being heads.  
The entropy is sub-maximal because there is less uncertainty regarding the outcome: tails is more probable than heads.
If one of the outcomes is impossible, for example if the coin has two heads, then the entropy of the coin is $0$ as there is no uncertainty regarding the result of a toss.


\subsection{Normalised Entropy} \label{subsec:normalised-entropy}
Plain entropy is not a good choice when trying to characterise random variables with different cardinalities of their sample space.
Let us suppose that the objective is to find the variable with the least \enquote{entropic} distribution and we suppose that their values have all been generated by the same process, say Gaussian.
Simply calculating their entropies and ordering them according to this criterion will bias the selection process towards the variables with smallest cardinality.
This is because we supposed them to be distributed in the same way so there will naturally be less uncertainty when there are fewer possible outcomes.
This can easily be understood by imagining the distributions to all be random uniform.

To obviate to this problem we need to \textit{normalise} the entropy so that different-sized variables can be directly compared to each other.
To achieve this, we can look at a measure of \textit{normalised entropy} or \textit{efficiency}:
\begin{equation} \label{eq:normalisedentropy}
 	\eta(X)=-\sum_{i=1}^{n} \frac{p\left(x_{i}\right) \log _{b}\left(p\left(x_{i}\right)\right)}{\log _{b}(n)}
\end{equation}
From Eq. \ref{eq:normalisedentropy} it can be seen that $\eta(X) \in [0,1]$; it is thus normalised and comparable among distributions.
This ratio expresses the amount of entropy found in the distribution compared to the maximum possible entropy when using $n$ symbols, corresponding to the uniform distribution:
\begin{equation}
\mathrm{H}\left(\underbrace{\frac{1}{n}, \ldots, \frac{1}{n}}_{n}\right) = - \sum_{i=1}^n \frac{1}{n} \log _{b} \left( \frac{1}{n} \right) = -n \cdot \frac{1}{n} \log _{b} \left( \frac{1}{n} \right) = - \log _{b} \left( \frac{1}{n} \right) = \log _{b}(n) 
\end{equation}   
\section{Graph Theory} \label{sec:graph-theory}
Many problems in Machine Learning (ML) do not involve classification or prediction of single data points in isolation, but of sets of entities that may present a more, or less, complex relation with each other. 
Most real-world phenomena fit into the latter framework.
Graphs are one of the most powerful tools for the modelling of this class of problems, as their structure naturally captures the wide variety of relations that may exist between entities.
These range from the atomic structure of a molecule to a social network of friends.  
In both these examples graphs help in reasoning, visualising and making inferences and predictions.

\subsection{Graphs} \label{subsec:graphs}
\begin{definition}[Directed Graph]
	A directed graph is a tuple 
	\begin{equation*}
		\mathcal{G} = (\mathcal{V}, \mathcal{E}) \,,
	\end{equation*}
with $\mathcal{V} = \{ v_1 \ldots v_n \}$ the set of vertices/nodes and $\mathcal{E}\subseteq \mathcal{V} \times \mathcal{V}$ the set of edges.
\end{definition}
We will not be needing the subclass known as \textit{undirected graphs} that are characterised by $\mathcal{E}$ being a \textit{set} of unordered pairs, that is of sets of the form $\{x,y\}$, with $x,y \in \mathcal{V}$.

The class of graphs presently of interest are those where there can be at most a single directed edge between any pair of nodes in $\mathcal{V}$; that is, we are not considering \textit{multigraphs}.
We are also interested in enforcing that there be no \textit{cycles} in the graph so there can be no subset of edges in $\mathcal{E}$ that when followed from vertex $v_i$ eventually ends up in $v_i$ again.
A cycle is a \textit{walk} - a sequence of edges which joins a sequence of vertices - of nodes of the form $v_i, v_j, \cdots, v_i$ i.e., a walk where only the first and last vertex are repeated.
Thus we have also automatically excluded the special case of cycle called \textit{self-loop}: an edge from a node to itself. 
The resulting graph possessing only directed edges and no cycles is commonly called a \textit{directed acyclic graph}, or DAG for short.  
\begin{definition}[Directed Acyclic Graph] \label{def:dag}
	A directed acyclic graph is a graph where every edge is directed and there are no cycles.
\end{definition}

In a DAG we may qualify nodes based on their \enquote{relationship status}:
\begin{description}
	\item[children] The children of node $u$ are all nodes $k$ for which there is a \textit{directed edge} from $u$ to $k$
	\item[parents] The parents of node $u$ are all nodes $k$ for which there is a \textit{directed edge} from $k$ to $u$
	\item[descendants] The descendants of node $u$ are all nodes $k$ for which there is a \textit{directed path} i.e., a walk where all vertices are distinct, from $u$ to $k$
	\item[ancestors] The ancestors of $u$ are all nodes for which there is a directed path from $k$ to $u$
\end{description}

An example of a DAG, containing five nodes, is shown in Figure \ref{fig:bn-example-dag}.

\begin{figure}[htbp]
\centerline{\includegraphics[width=0.5\textwidth]{mathematical-background/images/bn-example-structure}}
\caption{Example DAG representing a subset of the data set used in this thesis}
\label{fig:bn-example-dag}
\end{figure}

Polytrees and trees will also be defined because these are a fundamental concept for the work carried out in this thesis.
\begin{definition}[Tree] \label{def:tree}
	A tree is an undirected graph where there is one and only one walk between every node.	
\end{definition}
\begin{definition}[Polytree] \label{def:polytree}
	A polytree is a DAG whose underlying undirected graph is a tree.
	That is, if the directionality of edges is removed from the DAG, the resulting object is a tree. 
\end{definition}

\subsection{D-separation} \label{subsec:d-separation}
\textit{Dependence-separation} or \textit{d-separation}, as the name entails, is a concept relating to the conditional dependence between variables and was first presented by \citet{Pearl1988}.
We define the notation $u \rightarrow v$ to signify that there is a \textit{trail} between $u$ and $v$ in the graph, with a trail $(u, \ldots ,v)$ being a walk where all edges are distinct.

$u$ and $v$ and a node $z \in Z$ may be arranged in the graph in one of the following four configurations, called \textit{v-structures} in this context:
\begin{itemize}
  \item \textit{chain}: $u \rightarrow z \rightarrow v$
  \item \textit{chain}: $u \leftarrow z \leftarrow v$
  \item \textit{fork}: $u \leftarrow z \rightarrow v$
  \item \textit{collider}: $u \rightarrow z \leftarrow v$
\end{itemize}

These v-structures may be \textit{closed} by the set $Z$, this happens when:
\begin{itemize}
  \item \textit{chain}: $u \rightarrow z \rightarrow v$ and $z \in Z$
  \item \textit{chain}: $u \leftarrow z \leftarrow v$ and $z \in Z$
  \item \textit{fork}: $u \leftarrow z \rightarrow v$ and $z \in Z$
  \item \textit{collider}: $u \rightarrow z \leftarrow v$ and $z \notin Z$ and no descendant $z'$ of $z$ i.e., $z'$ such that $z \rightarrow z'$ exists, is also in the set $Z$
\end{itemize}

We say that $u$ and $v$ are \textit{d-separated} by $Z$ if every v-structure they appear in, is closed by $Z$.
Conversely, if there is at least one \textit{open} v-structure $u$ and $v$ are \textit{d-connected}.

If $u$ and $v$ are d-connected, then knowing something about $u$ also tells us something new about $v$, and viceversa.
An intuition for this can be given by interpreting the paths \textit{causally}.
In the case of a \textit{chain}, $z$ is the cause of $v$ so knowing $z$ tells us everything we need to know about the value of $v$ (or of $u$, if the chain is reversed).
In a \textit{fork}, conditioning on the \textit{common cause} $z$ has the same effect: $z$ is sufficient to know $u$ and $v$.
This is also called the \enquote{Common Cause Principle} \citep{sober1988principle}.

A good intuition for when $u$ and $v$ are \textit{d-separated} was given by \citet{Pearl1988}: imagine that there are two independent causes for a car refusing to start ($z$): having no gas ($u$) and having a dead battery ($v$): $u \rightarrow z \leftarrow  v$.
Only knowing that the battery is charged gives no information about the car having fuel or not.
But if we now know that the battery is charged after observing that the car won't start, we know for sure that it must be out of fuel.
So knowing something about $u$ is informative about $v$, after conditioning on $z$.

\begin{definition}[D-Separation] \label{def:d-separation}
	Given vertices $u$ and $v$ and a set of vertices $Z$, then $u$ and $v$ are d-separated by $Z$ if:
	\begin{itemize}
		\item $Z \neq \emptyset$ and $u$ and $v$ are never part of a collider;
		\item $Z = \emptyset$ and $u$ and $v$ are always part of a collider.
	\end{itemize}
\end{definition}

The independencies between variables are encoded in the structure of the DAG so every probability distribution modelled by a BN that has the same connections between nodes, also necessarily has the same independencies regardless of the values of the variables.

A series of examples using the DAG presented in Figure \ref{fig:bn-example-dag} are shown in Figures \ref{fig:bn-separations-example-1}, \ref{fig:bn-separations-example-2}, \ref{fig:bn-separations-example-3}.
We can see how the network's topology and the nodes chosen to be in the observed set $Z$, define the resulting separations.
In all cases $u=  \text{eta arrotondata} $ and $Y=V \mysetminus u \mysetminus Z$; we are asking for the set of all nodes in the DAG that are d-separated from $u$, given evidence $Z$.
This can easily be answered by enumerating all the v-structures in the network and applying Definition \ref{def:d-separation}.
In the case shown in Figure \ref{fig:bn-separations-example-1} we see that the node \textbf{eta arrotondata} is separated from nodes \textbf{recettori estrogeni}, \textbf{differenziazione} and \textbf{pN} given the observed evidence \textbf{mut17q21}.
The reason for this is because \textbf{$\text{eta arrotondata} \leftarrow \text{mut17q21} \rightarrow \text{recettori estrogeni}$} is a \textit{fork} and thus the flow of information from the rest of the network is blocked.
The way in which changing the conditioning set $Z$ also changes the independencies, can clearly be seen by comparing Figures \ref{fig:bn-separations-example-2} and \ref{fig:bn-separations-example-3}.

\begin{figure}[htbp]
\centerline{\includegraphics[width=0.5\textwidth]{mathematical-background/images/bn-example-separations-1}}
\caption{D-Separations in a subset of the provided data set (see Section \ref{sec:data-set})}
\label{fig:bn-separations-example-1}
\end{figure}

\begin{figure}[htbp]
\centerline{\includegraphics[width=0.5\textwidth]{mathematical-background/images/bn-example-separations-2}}
\caption{D-Separations in a subset of the provided data set (see Section \ref{sec:data-set})}
\label{fig:bn-separations-example-2}
\end{figure}

\begin{figure}[htbp]
\centerline{\includegraphics[width=0.5\textwidth]{mathematical-background/images/bn-example-separations-3}}
\caption{D-Separations in a subset of the provided data set (see Section \ref{sec:data-set})}
\label{fig:bn-separations-example-3}
\end{figure}

\section{Bayesian Networks} \label{sec:bayesiannetworks}
\subsection{Bayesian Networks Definition}
Given a variable $X$ and a set of variables $\boldsymbol{Y} = \{Y_1, ..., Y_n\}$, a \textit{conditional probability table} (CPT) is a table whose columns are in one-to-one correspondence with one of all the possible combinations of values of the variables in $\boldsymbol{Y}$. 
Each column is a probability mass function over $X$, say $P(X \mid Y_1=y_1, \ldots ,Y_n=y_n)$, conditional on the tuple $(Y_1=y_1, \ldots ,Y_n=y_n)$.
An example CPT can be seen in Table \ref{tab:b-cpd}.

\begin{definition}[Bayesian Network] \label{def:bayesian-network}
	A Bayesian Network (BN) is a probabilistic graphical model represented by a DAG $\mathcal{G}$ whose vertices are in one-to-one correspondence with a set of random variables $\mathcal{U}$ and the edges model the dependencies among these.
	The probability distribution of each variable $U \in \mathcal{U}$ is given by a CPT whose entries depend only on its parents in the graph structure.
	
	The so-called Markov Condition states that every variable $U$ is independent of all nodes in the network, except its descendent $Desc(U)$, given its parents $Pa(U)$:
	\begin{equation*}
		\forall U \in \mathcal{U}:  ( U \perp \neg Desc(U) \mid Pa(U)) \,.
	\end{equation*}
\end{definition}
The Markov Condition (named after the 19th century mathematician Andrej Markov) defines a d-separation on the DAG where each node/variable $U$ is d-separated from all others given the conditioning set $Pa(U) = \boldsymbol{Z}$.

A BN model is basically a way of representing an explicit joint distribution of random variables $\mathbb{P}(U_1=u_1, \ldots ,U_n=u_n) $ in a compact way.
The compactness is achieved by leveraging the Markov Condition, that is the independencies that exist among the random variables, and the Chain Rule (see Equation \ref{eq:chainrule}) to rewrite the joint as:
\begin{equation}
	\mathbb{P}(U_1=u_1, \ldots ,U_n=u_n) = \prod_i \mathbb{P}(U_i=u_i \mid Pa(U_i))
\end{equation} 
A BN gives the flexibility to drop the many weak dependencies that are bound to exist between variables thus leading to an even simpler model.
A full probability table for a joint distribution of random variables obscures the independencies and requires an exponential number of entries for the representation.
A Bayesian Network on the other hand can represent the same distribution using only a linear number of parameters.
The way that Bayesian Networks can be used to reduce the storage requirements for uncertain information is by taking advantage of the conditional independencies embedded in the underlying distribution being modelled.
The power of BNs comes from the additional information encoded in their structure and this was first explicitly described in its entirety by \citet{Pearl1988}, who defined the concept of dependence separation (see Subsection \ref{subsec:d-separation}) and applied it to Bayesian Networks.
A classic example of BN has been shown in Figure \ref{fig:asia-bn}.

One nice characteristic of BNs is that they very naturally model the type of mixed causal and stochastic processes that we find in all of Nature.
Imagine we want to represent the process modelled by joint distribution $\mathbb{P}(U,V)$; using the chain rule for conditional probabilities (Equation \ref{eq:chainrule}) we can write this as $\mathbb{P}(V \mid U) \mathbb{P}(U)$.
A BN modelling this process would be composed of two nodes $U$ and $V$ with an edge from the former to the latter $U \rightarrow V$, $U$ is called the \enquote{parent} of $V$.  Each of these two nodes would have its own probability table, with $\mathbb{P}(U)$ representing the \textit{prior} distribution over $V$ and $\mathbb{P}(V \mid U)$ the \textit{conditional probability distribution} of $V$ given $U$.

We can now see why these types of models are named \textit{Bayesian} Networks: the inference process is based in a given prior distribution/belief and evolves through a parent $\rightarrow$ child relationship to constantly yield an updated \textit{posterior} belief.
The BN DAG encodes a generative sampling where each variable's value is determined stochastically by Nature, based on the value of its parents.
This process is also highly compatible with our view of causality and this is one of the reason that makes BNs highly interpretable.
The prior $\mathbb{P}(A)$ can be seen as the result of some stochastic process caused by a series of latent (unmodelled) variables while the posterior $\mathbb{P}(B \mid A)$ is stochastically, causally determined by $A$. 
As mentioned in the previous paragraphs, there are probably no truly ``prior'' distributions in the Universe, at the modelling scale we are usually interested in.
Only on arriving on the quantum particle level may we find ``pure'' stochastic, uncaused processes due to quantum collapse.

A good example of how BNs are well compatible with our notion of causality may be to imagine $A$ as the random variable modelling the predisposition to having a certain disease and $B$ the one indicating actually developing the symptoms for it.
\textit{First}, genetic and epigenetic factors such as the environment stochastically contributed to having the predisposition and \textit{then} the development of the symptoms was stochastically determined by the degree of predisposition.
Adding an extra time dimension certainly helps in dealing with this class of models.

If the example show in Figure \ref{fig:bn-example-dag} is taken as the underlying graph structure of a Bayesian Network, each node now represents a random variable with an associated \textit{Conditional Probability Table} (CPT), that defines its probability distribution based on that of its parents.
The distributions for \enquote{eta arrotondata} and \enquote{mut17q21} in the Bayesian Network in question are shown in Table \ref{tab:mut-cpd} and \ref{tab:eta-cpd}.
\enquote{Mut17q21} is a root node i.e., has no parents, in the DAG so its probability distribution is unconditional or \textit{marginal}.
\enquote{Eta arrotondata}, on the other hand, is a child of \enquote{mut17q21} so the probability of its values is conditional on that of its parent and is represented by a CPT.
For example, \enquote{eta arrotondata} takes on value \enquote{<40} $44\%$ of the time when \enquote{mut17q21} has value \enquote{mut}, but only $4\%$ of the time when \enquote{mut17q21} has value \enquote{unknown}.

\begin{table*}[htbp]
\centering
\caption{mut17q21 mass function}
\begin{tabularx}{0.5\textwidth}{ccX}
\toprule
 \multirow{2}{*}{\textbf{mut17q21}} & mut & 0.01  \\
 & unknown & 0.99 \\
\bottomrule
\end{tabularx}
\label{tab:mut-cpd}
\end{table*}

\begin{table*}[htbp]
\centering
\caption{eta arrotondata CPT}
\begin{tabularx}{0.5\textwidth}{ccXX}
\toprule
      & &  \multicolumn{2}{c}{\textbf{mut17q21}} \\
\cmidrule(lr){3-4}
 & & mut & unknown    \\ 
 \multirow{3}{*}{\textbf{eta arr.}}  & <40 & 0.42 & 0.04  \\
 & 40-50 & 0.42 & 0.17    \\
 & >50 & 0.15 & 0.78 \\
\bottomrule
\end{tabularx}
\label{tab:eta-cpd}
\end{table*}

Probabilistic graphical models such as Bayesian networks - as just defined - are often used to express expert knowledge about a particular domain and perform reasoning on that problem. 
Alternatively, the specification of the network can be automatically achieved from a sufficient amount of data about the variables under consideration for a particular reasoning task. 
In this thesis we focus on the case of Bayesian network learned from data, but the methods presented in Chapter \ref{chap:methodology} would apply also to a user-designed network, as would be the case in an Expert System. 
As the Bayesian Network formalism consists of both a qualitative element (the directed graph) and a quantitative one (the conditional probability tables), in the following sections we will detail how these two components can be obtained automatically from data.

\subsection{Learning Bayesian Networks Structure from Data} \label{subsec:learning-bn-structure} 
Learning a BN DAG from data is typically addressed as an optimisation task and is known as the \textit{Bayesian Network Structure Learning problem}. 
In many probabilistic models initialisation is fast but then fitting the data is slow (for example in \textit{k-means}).
For Bayesian Networks the converse is true: fitting is fast as only sums of the counts in the data are needed, but learning the correct graph structure can take super-exponential time - more precisely, the space of Bayesian Networks that have $|V|$ variables has size $2^{O(|V|^2)}$ \citep{berzan2012exploration} - in the number of variables and thus easily becomes an intractable problem.

Let us consider the specification of a BN over the variables $\boldsymbol{X}=(X_1,\ldots,X_n)$ and denote as $\boldsymbol{D}$ a data set of joint and complete observations of $\boldsymbol{X}$. 
A \textit{score function} is a map $f$ giving a score to any possible DAG $\mathcal{G}$ whose nodes are in correspondence with $\boldsymbol{X}$ as a function of the dataset $\boldsymbol{D}$. 
The resulting score $f(\mathcal{G},\boldsymbol{D})$ is a measure of how well a BN with graph $\mathcal{G}$ fits the dataset $\boldsymbol{D}$.
The simplest approach consists in using the likelihood (i.e., the probability assigned by the BN to the data) as a score. 
Yet, to prevent over-fitting additional terms penalised complex models are added.
Given the score, the problem is basically a search over the set $\Gamma$ of all the possible DAGs with $|V|$ nodes i.e., $\mathcal{G}^* = \underset{\mathcal{G} \in \Gamma}{\arg\max} f(\mathcal{G},\boldsymbol{D})$. 
Such a task is NP-hard but approximate search procedures to solve it efficiently can be defined.

The methods to solve this problem can be roughly categorised into three categories:
\begin{description}
	\item[Search and Score] This is the most na{\"i}ve method as it does a brute force search over all the possibile graph structure space - i.e., all DAGs with the same number of variables as the input data - and scores all these depending on some cost function.
	
	There are many cost functions that have been proposed over the years; for example, a Bayesian cost function represents the probability of the DAG $G$ given the data $\boldsymbol{D}$: $\mathbb{P}(G \mid \boldsymbol{D})$, an Information Theory one scores the fitness of a DAG by its ability to balance graph description length and data description length given the graph. 
	
		This process is super-exponential in the number of variables but through the use of dynamic programming and heuristic search algorithms it can become sub-exponential.
		Nonetheless, solving the exact problem is only feasible up to $\approx 30$ variables.
	\item[Constraint Learning] Methods of this type calculate some measure of correlation to identify the presence and direction of edges between nodes and are much less used than the other ones presented.
		A typical test is to iterate over all triplets while testing for conditional independencies.
		Thanks to the d-separation properties outlined in Subsection \ref{sec:bayesiannetworks}, this test is able to identify the correct edges.
		The algorithm is quadratic in time and in the number of vertices.
	\item[Approximations] Several heuristical approaches have been developed to be able to find good network structures in an efficient manner.
		Examples of these are:
		\begin{itemize}
		  \item Chow-Liu, that builds a tree approximation of the probability distribution.
		  \item Greedy Hill-Climbing, that adds/removes/flips an edge at a time.
		  \item Optimal Reinsertion, that iteratively calculates the optimal \textit{Markov blanket} (the subset of all nodes that are sufficient to determine the value of another subset) of an ever-smaller subset of nodes.
		\end{itemize}
\end{description}

\subsection{Learning Bayesian Networks Parameters from Data} \label{subsec:learning-bn-parameters}
Once the DAG structure is given, learning the CPTs from the data $\boldsymbol{D}$ can be accomplished by one of two approaches:
\begin{description}
	\item[Frequentist] The frequentist approach treats the parameters $\theta$ to be learned as unknown but fixed and attempts to find a $\theta^*$ that maximises the likelihood function $\mathbb{P}(\boldsymbol{D} \mid \theta)$.
		Given symbol $j$ in the CPT of variable $i$ conditioned on the parents having value $k$ and $N_{i j k}$ the count of times that this combination of symbols appears in the data $\boldsymbol{D}$ then the Maximum Likelihood Estimator $\hat{\theta}_{i j k}^{MLE}$ for the entry $j,k$ in the CPT of $i$ is given by:
	\begin{equation}
		\hat{\theta}_{i j k}^{MLE}=\frac{N_{i j k}}{\sum\limits_{j^{\prime}} N_{i j^{\prime} k}}
	\end{equation}
	\item[Bayesian] The Bayesian method instead treats $\theta$ as a random variable with a prior probability $\mathbb{P}(\theta \mid \alpha)$, with $\alpha$ virtual pseudo-count, and uses Bayes' Rule (see Theorem \ref{th:bayes-theorem}) and a likelihood $\mathbb{P}(\boldsymbol{D} \mid \theta)$ to calculate the posterior probability $\mathbb{P}( \theta \mid \boldsymbol{D}, \alpha)$.
		Given symbol $j$ in the CPT of variable $i$ conditioned on the parents having value $k$ and $N_{i j k}$ the count of times that this combination of symbols appears in the data $\boldsymbol{D}$ then the Maximum a Posteriori Estimate $\hat{\theta}_{i j k}^{MAP}$ for the entry $j,k$ in the CPT of $i$ is given by:
		\begin{equation}
			\hat{\theta}_{i j k}^{MAP}=\frac{N_{i j k}+\alpha_{i j k}}{\sum\limits_{j^{\prime}}\left(N_{i j^{\prime} k}+\alpha_{i j^{\prime} k}\right)}
		\end{equation}	
\end{description}

\subsection{Bayesian Networks Updating} \label{subsec:bnupdating}
All the types of inference presented are instances of \textit{diagnostic reasoning}, also known as \textit{abductive reasoning}. 
Abductive reasoning is a form of inference that starts from observed evidences and looks for the best/most simple \textit{explanation} for them.
Unlike \textit{deductive reasoning}, abductive reasoning is not based on logical necessity, but on inferences based on observations; thus it can not verify the conclusion with absolute certainty but only yield, at best, a highly probable explanation.
The direction of inference is reversed in abduction - this is why it is sometimes called \textit{retroduction} - as compared to deduction.
It would be a logical fallacy know as \enquote{affirming the consequent} to state that any explanation were certain, because there may be multiple possible explanations for the same observation.

The following examples may help to clarify the difference between the two inference processes:
\begin{description}
	\item[deductive reasoning] given \enquote{Every man is mortal}$(a_1)$ and \enquote{Diogenes is a man}$(a_2)$ it necessarily follows that \enquote{Diogenes is mortal}$(b)$: $(a_1) \wedge (a_2) \implies (b) $.
	\item[abductive reasoning] given that \enquote{Diogenes is mortal}$(b)$ it is very probable that \enquote{Diogenes is a man}$(a_2)$; this is not certain as it may also be that \enquote{Diogenes is a dog}$(a_3)$: $(b) \centernot\implies (a_2) \wedge (b) \centernot\implies (a_3) $.
\end{description}

Abductive reasoning can either be modelled as a conditional probability or a MAP query and is of fundamental importance in many important problems of machine learning including medical diagnosis, that is of particular interest in this thesis.
Let us define three sets of variables of interest in the BN: $\boldsymbol{U_q}$ the \textit{query variables}, $\boldsymbol{U_e}$ the \textit{evidence variables} and $\boldsymbol{U_m}$ all other variables.

We can thus define the conditional probability query as the updated probability of $\boldsymbol{U_q}$ based on the observation of the values of $\boldsymbol{U_e}$.

\begin{definition}[Conditional Probability Query]
	The conditional probability query $\mathbb{P}(\boldsymbol{U_q} \mid \boldsymbol{U_e}=\boldsymbol{e})$ for variables $\boldsymbol{U_q}$ given evidence $\boldsymbol{U_e}=\boldsymbol{e}$ is:
\begin{equation*} 
	\mathbb{P}(\boldsymbol{U_q} \mid \boldsymbol{U_e}=\boldsymbol{e}) = \frac{\mathbb{P}(\boldsymbol{U_q},\boldsymbol{U_e}=\boldsymbol{e})}{\mathbb{P}(\boldsymbol{U_e}=\boldsymbol{e})} = \frac{ \sum\limits_{\boldsymbol{U_m}} \mathbb{P}(\boldsymbol{U_q},\boldsymbol{U_e}=\boldsymbol{e},\boldsymbol{U_m}) }{ \sum\limits_{\boldsymbol{U_m, U_q}} \mathbb{P}(\boldsymbol{U_q},\boldsymbol{U_e}=\boldsymbol{e},\boldsymbol{U_m}) } = 
	\frac{ \sum\limits_{\boldsymbol{U_m}} \prod_i \mathbb{P}(U_i \mid Pa(U_i)) }{ \sum\limits_{\boldsymbol{U_m, U_q}} \prod_i \mathbb{P}(U_i \mid Pa(U_i)) } \,.
\end{equation*}
\end{definition}

Another common type of question we might ask a BN is the following: \enquote{given evidence $\boldsymbol{U_e}$ which is the most likely assignment of a subset of variables $\boldsymbol{U_q}$?}.
This is know as \textit{Maximum a posteriori (MAP)} inference and is a much harder problem that a conditional probability query.
The solution is given by solving an optimisation problem.
\begin{definition}[Maximum a Posteriori Query]  \label{def:map}
Given sets $\boldsymbol{U_q} \subseteq \mathcal{U}$ and $\boldsymbol{U_z} = \mathcal{U} \mysetminus \boldsymbol{U_e} \mysetminus \boldsymbol{U_q}$, the MAP query for $\boldsymbol{U_q}$, $\text{MAP }( \boldsymbol{U_q} \mid \boldsymbol{U_e}=\boldsymbol{e} )$, is:
	\begin{equation*}
		\text{MAP }( \boldsymbol{U_q} \mid \boldsymbol{U_e}=\boldsymbol{e} ) = \underset{\boldsymbol{q}}{\arg\max} \sum\limits_{\boldsymbol{z}} \mathbb{P}(\boldsymbol{U_q}=\boldsymbol{q}, \boldsymbol{U_z}=\boldsymbol{z} \mid \boldsymbol{U_e}=\boldsymbol{e}) \,.
	\end{equation*}
\end{definition}
The solution will be the assignment of values $\boldsymbol{z}$ to all variables in the set $\boldsymbol{U_z}$ that maximises their joint probability.

An important thing to note is that the greedy assignment where each variable picks its most likely value can be very different from the most likely joint assignment of all variables.
A simple example showing this is given by \citet[pag. 26]{koller2007}.
Suppose a Bayesian Network is composed of two nodes $U$ and $V$ with the former the parent of the latter: $U \rightarrow V$.
Assume their CPDs are represented by the CPTs shown in Tables \ref{tab:a-cpd} and \ref{tab:b-cpd}.
The most probable value for $U$ is $U=u_1$ and this constrains $V$ to choose equivalently between $V=v_0$ or $V=v_1$.
The probability of the assignment $(U=u_1,V=v_1)$ given by $( \underset{u}{\arg\max} \mathbb{P}(U=u), \underset{v}{\arg\max} \mathbb{P}(V=v)) )$ is $0.6 \times 0.5 = 0.3$.
On the other hand, the most likely joint assignment given by $\underset{u,v}{\arg\max} \mathbb{P}(U=u)\mathbb{P}(V=v)$ is $(U=u_0,V=v_1)$ and has probability $0.4 \times 0.9 = 0.36$.

\begin{table*}[htbp]
\centering
\caption{U CPT}
\begin{tabularx}{\textwidth/4}{ccX}
\toprule
 \multirow{2}{*}{\textbf{A}} & $u_0$ & 0.4  \\
 & $u_1$ & 0.6 \\
\bottomrule
\end{tabularx}
\label{tab:a-cpd}
\end{table*}

\begin{table*}[htbp]
\centering
\caption{V CPT}
\begin{tabularx}{\textwidth/3}{ccXX}
\toprule
       &  \multicolumn{3}{c}{\textbf{A}} \\
\cmidrule(lr){3-4}
 & & $u_0$ & $u_1$   \\ 
 \multirow{2}{*}{\textbf{B}}  & $v_0$ & 0.1 & 0.5  \\
 & $v_1$ & 0.9 & 0.5    \\
\bottomrule
\end{tabularx}
\label{tab:b-cpd}
\end{table*}

The MAP problem is hard to solve efficiently because it is part of the \textit{NP-hard} complexity class, as proved by \citet{Shimony1994}.
Calculating it in a brute-force way would mean elencating all the possible variable-value tuples and computing their joint probabilities; as these are exponential in the number of variables, the problem is evidently untractable.

This is true even in a Bayesian Network.  
Such a model may possess a linear number of parameters but the underlying distribution is still implicitly exponential in size.
Explicitly calculating the MAP defeats the very purpose of the BN, that is computational efficiency.
For this reason, there exist a host of approaches to optimising MAP: elimination algorithms, gradient methods, simulated annealing and other stochastic local searches, belief propagation and integer linear programming.

A special case of MAP is the \textit{Most probable explanation (MPE)} and is an easier problem to solve.
\begin{definition}[Most Probable Explanation Query]  \label{def:mpe}
 Given set $ \boldsymbol{U_q}= \mathcal{U} \mysetminus \boldsymbol{U_e}$, the MPE query for $\boldsymbol{U_q}$, $\text{MPE }( \boldsymbol{U_q}\mid \boldsymbol{U_e}=\boldsymbol{e} )$, is:
	\begin{equation*} 
		\text{MPE }( \boldsymbol{U_q} \mid \boldsymbol{U_e}=\boldsymbol{e} ) = \underset{\boldsymbol{q}}{\arg\max}\, \mathbb{P}(\boldsymbol{U_q}=\boldsymbol{q} , \boldsymbol{U_e}=\boldsymbol{e}) \,.
	\end{equation*}
\end{definition}
The solution will be the assignment of values $\boldsymbol{u_q}$ to all variables in the set $\boldsymbol{U_q}$ that maximises their joint probability.

An intuition for why the MPE is easier to solve can be given by comparing Definition \ref{def:map} with Definition \ref{def:mpe}; unlike MPE, MAP presents both a summation and a maximisation and as such is part conditional probability query, part MPE query.
All algorithms for the computation of MAP obviously apply to MPE too, but there exist efficient approximate algorithms for MPE that do not generalise to MAP such as Loopy Belief Propagation \citep{Pearl1988} and Stochastic Local Search \citep{Kask1999}.

\section{Summary}
The chapter has introduced a number of concepts from Probability, Information and Graph Theory to be used as groundwork for the formal description of Bayesian Networks that are a widely used class of probabilistic graphical models.

The section dealing with Probability Theory opened by describing the fundamental concept of \textit{probability distribution} that is a function from a set of events of interest to the real numbers.
The probability of an event may be interpreted through the \textit{frequentist} or the \textit{Bayesian} lens; the former sees probability as simply the limit of the ratio between the number of times the event of interest occurred and the total number of trials; the latter views probabilities as the subjective degree of belief of the manifestation of the event.
A \textit{random variable} is a mathematical construct that associates a value to every outcome in the set of possible events and is used to bring to the fore the attributes of interest while dealing with them in a clean way.
The main results presented are \textit{Bayes' Theorem}, that states how to update a prior belief in light of new knowledge, and the concept of \textit{independence} between events, that will be central when introducing \textit{d-separation}.
The final concept presented in the Probability Theory section is that of Correlation, a universally accepted measure for the interrelatedness between random variables.

The second section introduces a few key concepts from Information Theory relating to entropy and distance measures.
The \textit{Entropy} is a measure for the expected amount of information carried by a random variable, first introduced by Claude Shannon under inspiration by mechanical statistics. 
A derived quantity is \textit{normalised entropy}, also known as \textit{efficiency}, that is convenient because it varies between $0$ and $1$ and thus enables random variables and probability distributions of different sizes to be compared.
A second method, closely related to entropy, of measuring the interrelatedness of two random variables is that of \textit{mutual information}; this measure quantifies the amount of information of one variable already contained in the other.
Two popular distance measures were introduced: \textit{Hamming and Jaccard Distance}; the former measures the similarity of strings, based on the number of substitutions needed to transform one into the other, and the latter quantifies the similarity of sets, given the size of their intersection over union.

The third section relates to Graph Theory and starts by defining the basic notion of \textit{graph}, a set of vertices and edges connecting them, and of \textit{Directed Acyclic Graph}, a special case of graph whose edges are directed and that contains no cycles between vertices.
\textit{Trees} and \textit{polytrees} are briefly introduced and characterised as a particular case of DAG.
Finally, \textit{d-separation}, first introduced by Judea Pearl, is discussed.
D-separation is a concept relating to the conditional dependence between variables; sets of variables may become independent i.e., not influence each other, based on conditioning on a third set of evidence variables.
The independence properties depend on the topology of the graph, specifically in how the variables of interest are connected to each other; they may be organised into \textit{chains}, \textit{forks} or \textit{colliders}.

The final section of the chapter deals with introducing \textit{Bayesian Networks}, using many of the concepts laid out in the previous sections.
A Bayesian Network is a probabilistic graphical model represented by a DAG where each vertex corresponds to a random variable and the edges model the dependencies among these.
The basic idea is to factorise a complete joint distribution of the constituent variables into a series of Conditional Probability Tables, one for each variable, that are assigned to the nodes in the DAG.
The defining characteristic is that each variable's node values depend only on those of its parents.
Such a representation efficiently represents a joint distribution and very naturally models the type of mixed causal and stochastic processes found in Nature.
The DAG of a BN can either be given or learned directly from data; in the latter case, that is a super-exponential problem, there are three main classes of algorithms that may be applied: Search and Score, Constraint Learning and Approximations.
Once a DAG has been learned, the problem moves to querying (updating) the BN; the main classes of queries are Conditional Probability and Maximum a Posteriori Queries.
The first class asks for the value of a set of variables given the observation of the values of others in the network.
The second class, known as MAP queries, ask the question of finding the most probable assignment of values to a subset of variables, given the observation of the values of another subset.
This is, in general, an NP-hard problem but efficient solutions exist to a special case known as the Most Probable Explanation, where the set of query variables in the complementary subset to the evidence one.

%%%%
%%%% METHODOLOGY
%%%%
\chapter{Methodology} \label{chap:methodology}
\section{Introduction} \label{sec:methodology-introduction}
 The inspiration for the work carried out in this thesis was the paper \enquote{Explaining the Most Probable Explanation} by \cite{Butz2018}, that has been reviewed in detail in Sec. \ref{sec:explaining-the-most-probable-explanation}.
 The paper proposed a system that, starting from a Bayesian Network modelling a medical data set, would learn a \enquote{knowledge base} tree representing the chain of most probable deductions, starting from a set of initial evidence.
 This tree, deemed to represent the solution to the MPE query, could then be used to generate a dialogue in natural language with the medical expert, that the authors claim could lead to the extraction of extra knowledge from the original data set.  
 The driving hypothesis of the paper was that Bayesian Networks and the solution to the MPE problem would be a powerful tool in helping medical experts gain insights into data.
 
The paper did not provide any indication that a such a system had ever been built and any validation of the method was left by the authors for future work.
 This lack of real-world validation has been seen, as discussed in Chap. \ref{chap:literature-review}, to, unfortunately, be the norm in most papers published under the Explainable AI moniker.
 Many works are content to only give a \textit{Functionally-grounded Evaluation} (in the taxonomy of \cite{doshi2017towards}) for the methods they propose.
\cite{Butz2018}'s does not even give such an evaluation of the methods it proposes.
 As of the finalisation of this thesis (\today), there has been no work carried out in substantiating \cite{Butz2018}'s conclusions.
 As introduced in Chap. \ref{chap:introduction} and discussed in detail in Sec. \ref{sec:importance-of-explainability}, there is an ever greater need for Machine Learning models and systems to be explainable, especially in mission-critical domains as healthcare.
 
 For these reason the belief is that building a proof-of-concept system, whose logic is inspired by the method presented in the aforementioned paper, and validating it with real medical experts, will prove to be an important step forwards in the direction of providing the work with the highest level of corroboration, an \textit{Application-grounded Evaluation}.
 The objective of this thesis is not only to provide an assessment of the paper, but also to set a methodological precedent for the evaluation of a Machine Learning system with real domain experts on real tasks.
 This, as has been discussed in Chap. \ref{chap:literature-review}, is one of the main gaps existing today in the field of xAI; thus, carrying out such an evaluation presents a substantial element of novelty.
 As anticipated in Chap. \ref{chap:introduction}, the work carried out in this thesis had a certain degree of collaboration with a third party, the \textit{Istituto Cantonale di Patologia}, based in Locarno in the Swiss canton of Ticino \cite{istitutocantonalepresentazione}.
It is also hoped that the proof-of-concept system may be of real use to the medical experts who were provided with it, in performing their work.
 
 The chapter opens with an introduction of the partner involved in the evaluation of the methods: \textit{Istituto Cantonale di Patologia} (ICP), Locarno, that specialises in the histological analysis of tissue samples.
 The data set, that was provided as base for the methods of this thesis, is comprised of the clinical profiles of 3218 breast cancer patients in the Swiss canton of Ticino.
 
 The chapter continues with a presentation of the technological tools used to build the proof-of-concept system that was given to the ICP.
 The main ones of interest are \textit{Pomegranate}, an open-source probabilistic models package for Python, \textit{Pgmpy}, another graphical model package for Python and \textit{DAOOPT}, a specialised solver for the MPE problem that takes input in the \texttt{.uai} format, also described in the section.
 Two graphical model packages were used because there does not yet exist a Python library that conveniently implements all the functionality needed to work with Bayesian Networks.
 
 Next, the algorithms that are part of the methods of this thesis but that making use of standard techniques are presented.
 These include how the data set is imported, preprocessed and how the actual BN is learned using Pomegranate's structure learning functions.
 A classic algorithm for \textit{d-separation} by \cite{koller2007dseparation} is presented together with how it has been adapted to work together with the other methods introduced.
 After this, a method for calculating the MPE by using the external solver DAOOPT is proposed.
 Finally, a brief survey of classic Machine Learning methods for classification is laid out, with the objective of making the reader familiar with them before describing how they are used as a benchmark for the BN's classification performance.
 
 The main section of the chapter is the one presenting the novel methods applied in this thesis.
 The first topic addressed is a justification for using normalised entropy as the selection criterion to build the chain of deduction of \enquote{knowledge base}, as it is termed by \cite{Butz2018} in their paper (see Sec. \ref{sec:explaining-the-most-probable-explanation}).
 Then the novel algorithms that are at the core of this thesis are introduced: the three variants of the \textit{dialogue} that are adaptations of the method presented by \cite{Butz2018}, an algorithm to generate alternative explanation branches when the domain expert disagrees with the system during a dialogue, a greedy algorithm to construct a \enquote{pseudo-MPE} branch from random evidence that is then compared to the true MPE solution in a specific procedure.
Then, the rationale and algorithm for the calculation of the \textit{mutual information} (see Subsec. \ref{subsec:mutualinformation}) between pairs of variables in the BN.
 The last algorithm introduced is the one generating the natural language that is used in all others to interface with the user.
 
 The last section of the chapter presents the methodology used to validate the proof-of-concept tool, that implements all of the methods presented in the previous sections, from the point of view of its clinical relevance and its capacity to surface explainable outputs to the user.
 
% !TEX root = thesis-thomas-tiotto.tex

\section{Data set} \label{sec:data-set}
As anticipated in Chap. \ref{chap:introduction} the work carried out in this thesis had a certain degree of collaboration with a third party, \cite{istitutocantonalepresentazione}.

\subsection{Istituto Cantonale di Patologia}
Istituto Cantonale di Patologia (ICP) is an institute based in Locarno that is specialised in the histological analysis of tissue samples received from private patients, clinics and hospitals, mainly in support of cancer diagnosis.
Its Laboratory of Medical Diagnostics supports pathologists in the diagnosis of neoplastic diseases through the application of cytogenetic techniques; that is, the focus is on understanding how chromosomes relate to cell behaviour, particularly during mitosis and meiosis.
One of the techniques used is Fluorescence in Situ Hybridization (FISH) that is able to localise the presence or absence of specific DNA sequences in chromosomes.
These tests are aimed at identifying the precise profile of the cancer cells and thus inform the clinician on the best treatment for the specific patient.

In addition to its clinical support activities, the Istituto also carries out scientific research aimed at better understanding certain types of cancers at a basic level.
In the last ten years, the ICP has published more than 200 peer-reviewed papers and more than 100 works in non-peer reviewed journals and is active at a national and international level.

\subsection{Motivation}
My first contact with the ICP was during a meeting with Dr. Vittoria Martin (\cite{martin2012}), molecular citogenetist, in date 28/01/2019.
The institute had expressed interest in bringing machine learning into their workflow in order to both augment their profiling capabilities for patients and to be able to extract new knowledge from their existing data.
This knowledge-extraction may lead towards the confirmation of current scientific theories or may be the first step towards the formulation of novel ones.

\todo[inline]{chiedere a Vittoria cosa si aspettavano/aspettano}

My interest in collaborating with the Istituto stemmed from the desire to apply the methods described in Sec. \ref{sec:novel-contributions} to a real-world case.
Being the theoretical work being carried out in this thesis an expert-driven MPE approximation, collaboration with the institute has also provided the opportunity to implement a proof of concept using real histological data.
The doctors and researchers of the Istituto have been able to validate, from an Explainable AI and clinical relevance points of view, the model software that I have developed.
That is to say, they have validated the capacity of the developed software both in its capacity to support clinical decision making and to surface clarifying explanations of the data set and in its adherence to established medical literature.

\todo[inline]{da confermare con Vittoria piu avanti}

\subsection{Provided Data Set}
The data set I was provided with was created by \textit{Registro Tumori Ticino} (Locarno, Ticino) in order to highlight possible new relations between clinical, histopathological and molecular features, as well as to potentially discover novel biomarkers for the progression of the disease 
It consists of the histological records over 38 recorded variables of 3218 breast cancer patients who have been diagnosed between the years 2005 and 2014 within the Ticino canton of Switzerland.
The data set had been pre-processed by collaborators of IDSIA in agreement with the ICT with 13 of the variables being dropped, because not considered relevant.
In particular, all variables relating to the patient post-treatment were discarded as well as those recording the diagnosis date.
In Tab. \ref{tab:datasetvariables} is a description of the measured variables, together with their clinical meaning.
The value distribution of the data set is shown in Tab. \ref{tab:datasetdistribution}.

The indications from Dr. Martin on how to further preprocess the data are shown in Tab. \ref{tab:datasetpreprocess}.
Note that some variable names were simplified.

\begin{table*}[htbp]
\caption{Data set variables}
\begin{tabularx}{\textwidth}{@{} l Y @{}}
\toprule 
\textbf{Variable} & Clinical meaning \\
\midrule 
\textbf{Codice globale} & Unique patient identifier \\
\textbf{mut17q21} & Mutation of chromosome 17 \\
\textbf{loss 17} & Loss of chromosome 17 \\
\textbf{et\`a arrotondata} & The age of the patient at diagnosis \\
\textbf{Lateralit\`a} & The affected breast \\
\textbf{Situ SUBGROUP MZ} & The primary site code of the tumour \\
\textbf{Morfologia SUBGROUP MZ} & The morphology classification of the tumour \\
\textbf{pT SUBGROUP MZ} & Primary tumour in the TNM classification for breast cancer \\
\textbf{pN SUBGROUP MZ} & Pathologic in the TNM classification for breast cancer \\
\textbf{M 8.2.96} & Distant metastasis in the TNM classification for breast cancer \\
\textbf{Differenziazione} & Tumour grade \\
\textbf{Recettori estrogeni percento 1.1.2003} & Expression of estrogen receptors \\
\textbf{Recettori progestinici percento 1.1.2003} & Expression of progestin receptors \\
\textbf{c erbB 2  cod percento 1.1.2003} & ErbB2 marker expression \\
\textbf{Ki67 cod percento} & Tumoural proliferation index \\
\textbf{FISHRatio} & FISH analysis result \\
\bottomrule
\end{tabularx}
\label{tab:datasetvariables}
\end{table*}

\begin{table*}[htbp]
\caption{Data set distribution before pre-processing}
\begin{tabularx}{\textwidth}{@{} l X c @{}}
\toprule 
\textbf{Variable} & Unique values & Distribution \\
\midrule 
\textbf{mut17q21} & 2 & \includegraphics[width=0.2\textwidth, height=10mm]{methodology/images/mut17q21}  \\
\textbf{loss 17} & 3 & \includegraphics[width=0.2\textwidth, height=10mm]{methodology/images/loss_17}\\
\textbf{eta arrotondata} & 74 & \includegraphics[width=0.2\textwidth, height=10mm]{methodology/images/eta_arrotondata}\\
\textbf{ateralit\`a} & 3 & \includegraphics[width=0.2\textwidth, height=10mm]{methodology/images/lateralita} \\
\textbf{situ} & 5 & \includegraphics[width=0.2\textwidth, height=10mm]{methodology/images/situ} \\
\textbf{morfologia} & 5 & \includegraphics[width=0.2\textwidth, height=10mm]{methodology/images/morfologia} \\
\textbf{pT} & 23 & \includegraphics[width=0.2\textwidth, height=10mm]{methodology/images/pt} \\
\textbf{pN} & 6 & \includegraphics[width=0.2\textwidth, height=10mm]{methodology/images/pn} \\
\textbf{M} & 3 & \includegraphics[width=0.2\textwidth, height=10mm]{methodology/images/m} \\
\textbf{differenziazione} & 5 & \includegraphics[width=0.2\textwidth, height=10mm]{methodology/images/differenziazione}  \\
\textbf{recettori estrogeni} & 40 & \includegraphics[width=0.2\textwidth, height=10mm]{methodology/images/recettori_estrogeni} \\
\textbf{recettori progestinici} & 40 & \includegraphics[width=0.2\textwidth, height=10mm]{methodology/images/recettori_progestinici}\\
\textbf{c erbB 2} & 4 & \includegraphics[width=0.2\textwidth, height=10mm]{methodology/images/c_erb_2}\\
\textbf{Ki67} & 52 & \includegraphics[width=0.2\textwidth, height=10mm]{methodology/images/ki67}\\
\textbf{FISH} & 5 & \includegraphics[width=0.2\textwidth, height=10mm]{methodology/images/fish}\\
\bottomrule
\end{tabularx}
\label{tab:datasetdistribution}
\end{table*}

\begin{table*}[htbp]
\caption{Data set preprocessing steps}
\begin{tabularx}{\textwidth}{@{} l Y @{}}
\toprule 
\textbf{Variable} & Action \\
\midrule 
\textbf{Codice globale} & Remove variable \\
\textbf{mut17q21} & Remove blanks \\
\textbf{loss 17} & Remove blanks \\
\textbf{eta arrotondata} & Bin into \enquote{$< 40$}, \enquote{$40-50$}, \enquote{$\geq 50$} \\
\textbf{lateralita} & Remove blanks and \enquote{sconosciuta} \\
\textbf{situ} & Remove blanks \\ \addlinespace
\textbf{morfologia} & Remove blanks and \enquote{unuseful} if performance on classification is subpar \\ \addlinespace
\textbf{pT} & Remove blanks and \enquote{unuseful}  \\
\textbf{pN} & Remove blanks and bin into \enquote{0} and \enquote{$\neq0$}\\
\textbf{M} & Remove blanks \\ 
\textbf{differenziazione} & Remove blanks and \enquote{Sconosciuto o non applicabile} \\ \addlinespace
\textbf{recettori estrogeni} & Remove blanks and bin into \enquote{negativo} if $\leq 10$,
		\enquote{debolmente positivo} if $\leq 50$, 
		\enquote{fortemente positivo} if $> 50$ \\ \addlinespace
\textbf{recettori progestinici} & Remove blanks and bin into \enquote{negativo} if $\leq 10$, 
		\enquote{debolmente positivo} if $\leq 50$, 
		\enquote{fortemente positivo} if $> 50$ \\ \addlinespace
\textbf{c erbB 2} & Remove blanks \\ 
\textbf{ki67} & Remove blanks and bin into \enquote{<14}, 
		\enquote{14-20}, \enquote{20-30}, \enquote{>30} \\ 
\textbf{FISH} & Remove blanks \\
\bottomrule
\end{tabularx}
\label{tab:datasetpreprocess}
\end{table*}


% !TEX root = thesis-thomas-tiotto.tex

\section{Methods} \label{sec:methods}

\subsection{Libraries}\label{subsec:libraries}
The system developed in this thesis was coded in Python and as such made use of an array of standard and less-know packages.
The most significant for the development of the system are Pomegranate and Pgmpy, that are both packages implementing probabilistic graphical models.

\subsubsection{Probabilistic Graphical Models Packages}
\textit{Pomegranate}\footnote{\url{https://Pomegranate.readthedocs.io/en/latest/}} is an open-source probabilistic models package for Python.
Its core philosophy is that every probabilistic model, from Hidden Markov to Bayesian Network, can be seen as a probability distribution and, as such, can be flexibly composed into hierarchical mixture models \citep{Schreiber2017}.
The package implements Bayesian Network as well as many other probabilistic models but currently only supports Discrete Bayesian Networks, so the random variable of each node must have a categorical distribution.
This is not an issue as the provided data set (see Section \ref{sec:data-set}) was already composed of only categorical variables.
Also, working with discrete entities should make explainability easier as the number of possible variable values at hand can be reduced at will; this should in turn reduce the cognitive load requested from the user.

Pomegranate was chosen among others for its good implementation of Bayesian Networks and its performance.
The package is written in Cython and natively supports multi-core parallelism and out-of-core learning.
Network \textit{structure learning from data} is claimed to be particularly efficient, thanks to a novel c\textit{onstraint learning} (see Subsection \ref{subsec:learning-bn-structure}) method that implements prior knowledge into the graph selection process \citep{schreiber_noble_2017}.
The claim made by the paper is that this innovative graph selection process should possess the speed of a heuristic approach, while still yielding a far better quality estimate of the correct graph structure.

\textit{Structure learning} from data is achieved using the \texttt{from\_samples} method of the \\ \texttt{BayesianNetwork} class, with the default algorithm being the novel one described by \citet{schreiber_noble_2017}.
The \textit{probability} of a sample is calculated using the \texttt{probability} function; the \texttt{predict\_proba} function is used to return the probability of each variable in the model given some evidence.
\textit{Predictions} (described in detail in Subsection \ref{subsec:bnupdating}) are run by passing an object a matrix with \texttt{None} as placeholders for missing values to the \texttt{predict} function.
\textit{Fitting} is done thought the \texttt{fit} function that uses MLE estimates to update each node's distribution in the model based on the input data.

A \texttt{BayesianNetwork} object can also be displayed graphically by calling its \texttt{plot} function.
The output is a \texttt{DOT} file that is generated using the PyGraphviz package\footnote{\url{https://pygraphviz.github.io}}, a Python interface to the famous Graphviz\footnote{\url{https://www.graphviz.org}} graph visualisation software.
An example of such an output is shown in Figure \ref{fig:Pomegranate_graph_example}.

\begin{figure}[htbp]
\centerline{\includegraphics[width=0.8\textwidth]{methodology/images/Pomegranate_example}}
\caption{Example output of \texttt{plot} \citep{Pomegranatetutorial}}
\label{fig:Pomegranate_graph_example}
\end{figure}

\textit{Pgmpy}\footnote{\url{http://pgmpy.org}} is, like Pomegranate, another recent probabilistic graphical model package for Python.
Unlike Pomegranate, it natively implements various exact and approximate inference algorithms (see Subsection \ref{subsec:bnupdating}), like variable elimination, belief propagation and max-product linear programming.

The reason that two different probabilistic graphical model libraries were used, is because there is currently no Python package that offers all the needed functionality.
Pomegranate implements a novel structure learning algorithm but is severely lacking in functionality in many other areas.
Pgmpy, on the other hand, has a very good API as regards inference.

\subsubsection{External Solvers}
\textit{DAOOPT}\footnote{\url{https://github.com/lotten/daoopt}} is an open-source implementation of the sequential AND/OR Branch-and-Bound algorithm proposed by \citet{Marinescu2006}.
Search-based algorithms traverse the model's space and are much more efficient in their use of memory, compared to inference-based algorithms such as variable elimination.

DAOOPT builds an AND/OR search space to generate an AND/OR graph that takes advantage of information encoded in the graphical model, namely its independencies.
The DAOOPT implementation found at \citet{daoopt}, is an exact solver for finding an MPE solution in Bayesian Networks.
The software is written in C++ and accessible through a command-line interface; the only required parameter is a \texttt{.uai} file representing a Markov Random Field or a Bayesian Network but in most cases an optional \texttt{.uai.evid} file will also be given, containing the observed evidences.

The \texttt{.uai} file format is a simple text file used to represent problem instances.
Such a file is composed of:
\begin{itemize}
  \item \textit{Preamble}: containing the type of the network (MARKOV or BAYES), the number and cardinality of variables and the cliques, that in the BAYES case are simply the variables appearing in each Conditional Probability Table (CPT).
  \item \textit{Function tables}: containing the actual definition of the CPTs i.e. the values of each node give its parents or, in the case of root nodes, the marginal probabilities.
\end{itemize}

The \texttt{.uai.evid} is a very simple file containing the number of variables in the evidence set followed by the index of each variable and its observed value.
In both formats the variables and their values are represented only by a numerical index, starting from $0$, with the ordering being defined in the preamble of the \texttt{.uai} and maintained consistent throughout the \texttt{.uai} and \texttt{.uai.evid}.

Following, is the \texttt{.uai} representing the network shown in Figure \ref{fig:bn-example-dag}, that has been the running example throughout the last chapters.
Lines starting with \texttt{c} are interpreted as comments; these are misinterpreted by DAOOPT and are thus removed when running it, but are here shown for clarity and because they are part of the official \texttt{.uai} format.
The file starts by stating that the model is a Bayesian Network composed of 5 random variables; these will then be referenced by an ordinal index starting at 0.
The first variable (index 0) is of cardinality 3, the second (index 1) is of cardinality 2 and so on.
We can then see the definition of the cliques or more precisely, as the model is BAYES  and not MARKOV, of the CPTs; there are 5 of these, each one associated to one of the five variables just stated.
The first CPT involves 2 random variables: the first (0) and the second (1); the second CPT involves only one variable (1) and this tells us that variable 1 is a root node in the BN's DAG.
The ordering is such that the child node is the last in the definition of each CPT's nodes so, for example, in the first CPT we find that variable 1 is the child of variable 0.
Finally, we have a complete definition of the function tables/CPTs.
The tables are printed so that each row corresponds to the conditional probability value of the child node and increasing rows correspond to increasing enumeration of the parents' states, in the order given when defining the variables involved in the CPTs.
The first table corresponds to \textbf{eta arrotondata}'s CPT shown in Table \ref{tab:eta-cpd}.
It contains 6 elements as it involves variables 0 (\textbf{mut17q21}) and 1 (\textbf{eta arrotondata)} that are of cardinality 2 and 3, respectively.
So, each row corresponds to the probability distribution of the three states of variable 1, given each of the two states of variable 0.

\begin{minipage}{\linewidth}
\begin{framed}
\begin{verbatim}
c
c Bayesian Network exported from Pomegranate - Thomas Tiotto (2019)
c

BAYES
5
3 2 3 3 2 

c
c Cliques
c

5
2 0 1 
1 1 
2 2 1 
3 3 2 4 
1 4 

c
c CPTs
c

6
 0.42105263157894735 0.42105263157894735 0.15789473684210523 
 0.043798177995795384 0.17063770147161877 0.7855641205325858 

2
 0.006613296206056387 0.9933867037939436 

6
 0.6842105263157895 0.0 0.3157894736842105 
 0.1373510861948143 0.021723896285914507 0.8409250175192712 

18
 0.004385964912280701 0.2412280701754386 0.7543859649122807 
 0.022598870056497175 0.11864406779661016 0.8587570621468926 
 0.10344827586206899 0.41379310344827586 0.4827586206896552 
 0.2121212121212121 0.45454545454545453 0.3333333333333333 
 0.14094488188976378 0.6362204724409449 0.22283464566929131 
 0.289612676056338 0.5677816901408451 0.1426056338028169 

2
 0.5315001740341107 0.4684998259658893 
\end{verbatim}
\end{framed}
\end{minipage}

The following is an example of a randomly generated \texttt{.uai.evid} evidence file that simply states that the evidence set has cardinality 2 and contains variable 4 (in the ordering given in the \texttt{.uai}) in its state 1 and variable 3 in state 2.

\begin{framed}
\begin{verbatim}
 2
  4 1
  3 2
\end{verbatim}
\end{framed}

Both the \texttt{.uai} and the \texttt{.uai.evid} were generated by the custom functions presented in Subsection \ref{subsec:algorithms} under the \textbf{MPE} header.
These are able to export a \texttt{Pomegranate} model and randomly generated evidence to the correct input format for DAOOPT.

\subsubsection{Standard Packages}
\textit{pandas}\footnote{\url{https://pandas.pydata.org/about.html}} is an extremely widely-used open-source Python library that provides data structures and methods to support data analysis.
The package excels in the manipulation of tabular data in the form of \texttt{DataFrame}, that is the analogous of R's \texttt{data.frame}.
A \texttt{DataFrame} can be seen as a \enquote{general 2D, size-mutable structure with potentially heterogeneously-typed columns}.
The syntax for slicing is very close to R's, as are many other functionalities; this is because one of Pandas' explicit goals was to offer all of CRAN's functionalities and to be easily approachable by anyone already knowing the other language.

Pandas was the default choice for this thesis' implementation because it is the \textit{de facto} standard in data analysis applications when using Python.
Its flexibility in reading Excel spreadsheets (the format of the provided data set, see Section \ref{sec:data-set}) and in then manipulating the data confirmed that this was a good choice.
Note that the additional \texttt{xlrd} package is needed to read files in the Excel format.

\textit{Scikit-learn}\footnote{\url{http://scikit-learn.github.io/stable}} aims at providing a unified API for basic machine learning; it does not include advanced paradigms such as Reinforcement Learning or graphical models for structured learning.
The latter omission was the reason that lead to select Pomegranate as the basis for the implementation of the system.

What is included are a stack supervised and unsupervised ML tools to prepare data sets, define machine learning models ranging from spectral analysis-based to ensemble methods to clustering and multiple evaluation and model selection utilities.

\textit{NumPy}\footnote{\url{http://numpy.org}} is another \textit{de facto} standard package when doing scientific computing with Python.
Most scientific packages (including Pandas, Scikit-learn and TensorFlow) depend on NumPy for low-level operations; this is because NumPy provides a fast implementation of n-dimensional array objects together with powerful manipulation functions.
In addition to this, NumPy implements linear algebra operations, Fourier Transform and random number generation.

The closest parallel to NumPy - as R was for Pandas, is MATLAB.

\textit{NetworkX}\footnote{\url{https://networkx.github.io}} is another widely-used package; it is specialised in the creation and manipulation of graph-structured data.
The main use for this package was in building the \enquote{knowledge base} structure that  the dialogue with the expert is based on.

\subsection{Algorithms} \label{subsec:algorithms}
This subsection is concerned with presenting algorithms and methods that were adapted and used for this thesis, but that were not part of the original work.

\subsubsection{Model Construction}
The data was given in \texttt{.xlsx} format and was imported using Panda's \texttt{read\_excel} function that returned a \texttt{DataFrame} object.
The imported data was then preprocessed by dropping unwanted records and binning the remaining ones following the instructions outlined in Table \ref{tab:datasetpreprocess}.
The actual BN representation is learned at runtime by calling the \texttt{from\_samples} method of Pomegranate's \texttt{BayesianNetwork} to solve the structure learning problem (defined in Subsection \ref{subsec:learning-bn-structure}).

The binned data was codified into integer representations before being passed to Pomegranate's structure learning algorithm.
Thus the network's state names are in natural language but the internal representation of the values of each random variable is an integer number.
A dictionary object is used to translate one representation into the other when interacting with the user.

\subsubsection{D-separation}
A na{\"i}ve implementation to check for d-separation between nodes $X$ and $Y$, according to Definition \ref{def:d-separation}, would have a complexity in the order of the number of trails between $X$ and $Y$; this would lead to an exponential, in the size of the graph's vertices set, running time.
Luckily, \citet{koller2007} present a linear time algorithm to solve the problem, whose pseudocode is shown in Algorithm \ref{alg:koller-d-separation}.

\begin{algorithm}[htp!]
	\caption{reachable procedure by \citet{koller2007}}
	\label{alg:koller-d-separation}
	\begin{algorithmic}[1]
		\State $\mathcal{G}$ BN graph
		\State $X$ source variable
		\State $\boldsymbol{Z}$ observations
		\State
		\State $\boldsymbol{L}= \boldsymbol{Z}$ \Comment{Phase 1}
		\State $\boldsymbol{A} = \emptyset$
		\While{$L \neq \emptyset$}
			\State Select some $Y$ from $\boldsymbol{L}$
			\State $\boldsymbol{L}=\boldsymbol{L} \mysetminus \{Y\}$
			\If{$Y \notin \boldsymbol{A}$}
				\State $\boldsymbol{L} = \boldsymbol{L} \cup Pa(Y)$
			\EndIf
			\State $\boldsymbol{A}=\boldsymbol{A} \cup \{Y\}$
		\EndWhile
		\State
		\State $\boldsymbol{A} = \{(X, \uparrow)\} $ \Comment{Phase 2}
		\State $\boldsymbol{V} = \emptyset$
		\State $R = \emptyset$
		\While{$\boldsymbol{L} \neq \emptyset$}
			\State Select some $(Y,d)$ from $\boldsymbol{L}$
			\State $\boldsymbol{L} = \boldsymbol{L} \mysetminus \{(Y,d)\}$
			\If{$(Y,d) \notin \boldsymbol{V}$}
				\If{$Y \notin \boldsymbol{Z}$}
					\State $R = R \cup \{Y\}$
				\EndIf
				\State $\boldsymbol{V} = \boldsymbol{V} \cup \{(Y,d)\}$
				\If{$d=\uparrow$ and $y \notin \boldsymbol{Z}$}
					\For{each $Z \in Pa(Y)$}
						\State $\boldsymbol{L} = \boldsymbol{L} \cup \{(Z,\uparrow)\}$
					\EndFor
					\For{each $Z \in Ch(Y)$}
						\State $\boldsymbol{L} = \boldsymbol{L} \cup \{(Z,\downarrow)\}$
					\EndFor
				\ElsIf{$d=\downarrow$}
					\If{$Y \notin \boldsymbol{Z}$}
						\For{each $Z \in Ch(Y)$}
							\State $\boldsymbol{L} = \boldsymbol{L} \cup \{(Z,\downarrow)\}$
						\EndFor
					\EndIf
					\If{$Y \in \boldsymbol{A}$}
						\For{each $Z \in Pa(Y)$}
							\State $\boldsymbol{L} = \boldsymbol{L} \cup \{(Z,\uparrow)\}$
						\EndFor
					\EndIf
				\EndIf
			\EndIf
		\EndWhile
		\State \textbf{return} R
	\end{algorithmic}
\end{algorithm}

The \texttt{reachable} procedure, as defined in the book, takes as input the DAG representing the Bayesian Network $\mathcal{G}$, a source variable $X$ and a set of observed variables $\boldsymbol{Z}$; on exit it returns the set of variables $R$ that are reachable from $X$.
The procedure runs in two phases, traversing the graph twice: first bottom-up from leaves to roots, then vice-versa.
During the first stage, the algorithm finds all nodes $\boldsymbol{A}$ that are ancestors of the evidence set $\boldsymbol{Z}$.
During the second phase, the procedure distinguishes the direction it visits each node in order to determine if it is traversable or not.
Any node $Y$ that is not in the evidence set is marked as reachable; if it is being visited in direction \enquote{up} ($(Y,\uparrow)$) it can be traversed as the v-structure (see Subsection \ref{subsec:d-separation} for a primer) is a \textit{chain}.
All the parents of $Y$ are marked to be visited in the \enquote{up} direction (i.e. from below) and the converse is done for $Y$'s children.
If $Y$ is being visited in the \enquote{down} ($(Y,\downarrow)$) direction its children are again added to be visited in the \enquote{down} direction, because $Y$ is traversable.
Additionally, if $Y$ happened to be in the set $\boldsymbol{A}$, found in the first step, then $Y$'s parents are marked to be visited in the \enquote{up} direction because the \textit{collider} is active and $Y$ can be traversed (a collider is open if and only if the central node or any of its descendants are observed).

The implementation in this thesis follows the pseudocode of the book very closely but the procedure \texttt{d-separated}, instead of finding all nodes $R$ that are d-connected to the input $X$, only tests if a given target $Y$ is d-separated from $X$ or not, as is shown in Algorithm \ref{alg:d-separation}.
This gives some extra flexibility in how the function can be used.
To find the set $S$ of all nodes d-separated from $X$, the \texttt{d-separated} is iterated to test over all nodes $V$ in the BN.

\begin{algorithm}[htp!]
	\caption{d-separation algorithm}
	\label{alg:d-separation}
	\begin{algorithmic}[1]
		\State $separated\_list = \emptyset$
		\For{target $Y \in V$} 
			\State append $d-separated(X, Y, E)$ to $separated\_list$ \Comment{will return true or false}
		\EndFor
	\end{algorithmic}
\end{algorithm}

\subsubsection{MPE}
The solution to the Most Probable Explanation problem (MPE) (Definition \ref{def:mpe}) is planned to be found by using DAOOPT (described in Subsection \ref{subsec:libraries} under the \textbf{DAOOPT} header) as an external solver.
The latest version of DAOOPT was downloaded from the official repository \citep{daoopt} and compiled into an executable.
DAOOPT only offers a command line interface so some extra work is needed in order to integrate it with the Python-based application under development.
The connection is provided by first writing to stable storage a \texttt{Pomegranate.uai} containing the model definition and a \texttt{Pomegranate.uai.uai.evid} with the chosen evidence.
These files are then fed to DAOOPT by using Python's \texttt{subprocess} module, by running the following command in a background shell:
\begin{verbatim}
	./daoopt -f Pomegranate.uai -e Pomegranate.uai.evid
\end{verbatim}
The shell output is captured and also written to stable storage, in order for the solution to be parsed from it.

To exemplify the process, we return to the example used while presenting DAOOPT in its Subsection in \ref{subsec:libraries}.
Given the \texttt{.uai} representing the BN and the \texttt{.uai.evid} random evidence:
\begin{framed}
\begin{verbatim}
 2
  4 1
  3 2
\end{verbatim}
\end{framed}

DAOOPT would give the following output:

\begin{minipage}{\textwidth}
	\begin{framed}
\begin{verbatim}
	--- Starting search ---
[0] u 3 4 -1.3581 5 2 1 2 2 1
[0] Cache statistics: . . . .

--------- Search done ---------
Problem name:  Pomegranate
OR nodes:      3
AND nodes:     4
OR processed:  3
AND processed: 8
Leaf nodes:    2
Pruned nodes:  4
Deadend nodes: 1
Time elapsed:  0 seconds
Preprocessing: 0 seconds
-------------------------------
-1.3581 (0.0438433)

p 2 1 2
l 2 1 6
s -1.3581 5 2 1 2 2 1
\end{verbatim}	
\end{framed}
\end{minipage}
The end of the final line is the one of interest, as it is the assignment of values to the variables that solves the MPE problem.
The \texttt{5 2 1 2 2 1} string is to be interpreted as meaning:
\begin{itemize}
  \item there are 5 variables in the solution
  \item the variable indexed by 0 (in the ordering given in the preable of the \texttt{.uai}) is assigned its second value (the ordering is inferred by the CPTs defined in the \texttt{.uai}) in the MPE solution
  \item variable 1 is assigned its second value
  \item variable 2 is assigned its third value
  \item variable 3 is assigned its third value
  \item variable 4 is assigned its second value
\end{itemize}
Variables 3 and 4 are constrained to assume the value specified in the input \texttt{.uai.evid}; in this case 1 and 2, respectively.

All the functionality relating to solving the MPE with DAOOPT is encapsulated in the \texttt{daoopt\_solver} function that given the input \texttt{.uai} files, returns the MPE solution.

%\subsubsection{Other Machine Learning Methods}
%In order to have a benchmark for the classification performance of the Bayesian Network, a series of tests were implemented with the aim of finding the best performing algorithm on the data set.
%Given that the performance of Machine Learning algorithms is heavily dependent on the input classes, a process of \textit{exhaustive variable elimination} was used, in order to identify the most relevant features for the predictions.
%Each input subset was scored using the following ML algorithms, in order to find the best performing one:
%\begin{itemize}
%  \item \textit{linear regression}: this method assumes that the relationship between the dependent variable $y$ and the regressors $x$ is linear i.e., that $y$ can be written as a linear combination of $x$'s components: $y = \beta_0 + \beta_1 x_1 + \ldots \beta_n x_n$.
%  \item \textit{logistic regression}: is used in lieu of Linear Regression when the values of the variables are categorical; it assumes that the relationship between the regressors $x$ and the log-odds of $y$ are linear i.e. $\log _{b} \frac{p}{1-p}=\beta_{0}+\beta_{1} x_{1}+ \ldots + \beta_{n} x_{n}$ with $p$ the probability of the event of interest.
%  \item \textit{linear discriminant analysis}: LDA is related to Principal Component Analysis (PCA) in that it attempts to find a linear equation modelling the data but LDA explicitly tries to express the difference between the data classes.
%  \item \textit{decision tree}: a decision tree is built using a recursive, greedy algorithm that continually splits the dataset into two.  The variable along which to bisect is the one that yields the lowest accuracy loss in the resulting split.
%  \item \textit{na{\"i}ve Bayes}: a na{\"i}ve bayes classifier is a conditional probability model that given features $x_1 \ldots x_n$, attempts to assign a probability to each of the possible outcomes $O_{k}$ of interest by using Bayes' Theorem (Definition \ref{th:bayes-theorem}): $\mathbb{P}\left(O_{k} | x \right)=\frac{\mathbb{P}\left(O_{k}\right) \mathbb{P}\left(x| O_{k}\right)}{\mathbb{P}(x)}$.  The method is called \enquote{na{\"i}ve} because of the strong (ofter unrealistic) assumption that all the features $x_1 \ldots x_n$ are independent.
%  \item \textit{k-nearest neighbours}: the algorithm is non-parametric with the output class depending on the predominant class among the $k$ nearest neighbours (according to some distance metric) of the input vector $x$.
%  \item \textit{support vector}: a support vector machine (SVM) is an algorithm that attempts to find the set of \textit{best-separating hyperplanes} between classes of objects, seen as points in a high-dimensional space.  Such a hyperplane is the one that has maximum distance from the closest representatives of each class.
%  \item \textit{random forest}: this is an ensemble method that aims to correct decision trees' tendency to overfit the data.  A multitude of decision trees is constructed and the final classification output is the class that appears most often in the intermediate step.
%  \item \textit{AdaBoost}: short for \textit{adaptive boosting}; this meta-learning, ensemble algorithm combines a series of \textit{weak classifiers}, that may only be slight better than a random guess, through a weighted sum into a \textit{strong classifier}.  It is a meta-learning algorithm because the weak classifiers are revised over a series of iterations in order to improve their performance on previously misclassified instances.
%  \end{itemize}




\section{Novel contributions}\label{sec:novel-contributions}
 \todo{tengo qui o sposto nel cap 4?}
\subsection{Theory}

\subsection{Algorithms}
\subsubsection{\enquote{pseudo-MPE}}
\subsubsection{alternative explanation branches}
\subsubsection{\enquote{pseudo-MPE} from random evidence}
\section{Validation Methodology} \label{sec:validation}
Having direct access to expert pathologists has not only helped in guiding research into the theoretical explainability properties of the system but also enabled their \textit{application-grounded evaluation} (see Section \ref{sec:evaluation-of-explainability}).
There are two main validation points of view to be addressed: the clinical (Subsection \ref{subsec:clinical-validation-methodology}) and the explainability (Subsection \ref{subsec:explainability-validation}), with the results of the latter depending on part on those of the former.
 
\subsection{Clinical Validation} \label{subsec:clinical-validation-methodology}
A validation of the methods carried out in this thesis in their adherence to established clinical literature is of paramount importance.
A failure on the Bayesian network's part in capturing the true relationships between the variables would hamper it in being able to give any meaningful representation of them.
For the experts to even start to trust the system or to be able to make sense of its outputs, it is vital that there be as little cognitive dissonance between their basic beliefs and expectations and those that they see represented in the system.

For this reason, the initial validation phase with the ICP concentrated on the clinical aspect.
The methodology chosen to clinically validate the system was for the ICP to formulate a series of natural language queries; each one of these questions was annotated with the queried variable and its value, together with the values of any evidence variables.
The experts included the expected reply to the queries together with its likelihood, based on the latest medical literature and their personal, knowledge-based expertise.
These questions can be abstracted as:
\begin{center}
\enquote{Given that the value of $var_1$ is $a_1$ and $\ldots$ and the value of $var_n$ is $a_n$, what is the probability that $var_{n+1}$ takes value $a_{n+1}$?}.	
\end{center}

The natural language questions formulated by the ICP can be classified along two axes:
\begin{itemize}
  \item based on their intended purpose: \textit{validation} vs. \textit{research}.
  The former questions' replies are known from established clinical literature and are the queries that will actually be used to validate the system from a clinical point of view.
  The latter are queries that don't have a definite clinical answer but that are nonetheless extremely interesting in helping to understand the types of questions a domain expert may want to ask the system.
  \item based on the way they may be answered: by a \textit{conditional probability query} (Definition \ref{def:conditional-probability}), a \textit{d-separation query} (Definition \ref{def:d-separation}) or an \textit{MPE query} (Definition \ref{def:mpe}).
\end{itemize}
The complete series of thirty questions has been organised according to the second criterion.
Appendixes \ref{app:conditionalprobability1} and \ref{app:conditionalprobability2} present fourteen questions that can be answered by conditional probability queries.
Appendix \ref{app:dseparation} shows a series of eight natural language questions that can be answered by running a d-separation query.
Appendix \ref{app:conditionalanddseparation} presents five questions that could be answered by a conditional probability query but also, at a higher level, by a d-separation query.
This is because what is being asked, is basically wether changing the value of the evidence variable has an influence on that of the target variable.
This could be answered by running multiple conditional probability queries and comparing the resulting target variable values or, more simply, by checking if the target and evidence variables are d-connected or not.
The first method would give a finer grained answer as it would also \textit{quantify} the magnitude of the effect of one variable on the other; checking for d-separation would only give a \textit{qualitative} answer, which may nonetheless be sufficient. 
Finally, Appendix \ref{app:mpe} shows three questions that are naturally mapped onto a query of the MPE type.

Most importantly at this stage, all questions can be implemented on the proof of concept system and consequently this shows a good coverage on the tool's part of the use cases that can be imagined by a domain expert.
If the system can, in principle, answer every question imagined by the expert then this is an indication that it conforms to her \textit{worldview} and thus could be well positioned to interact fruitfully with her.

The questions marked as \textit{validation} will be posed to the system, in autonomy, by the ICP's representatives, who will then compare the outputs with the result they would have expected, based on established medical literature and their expertise.
The columns containing the experts' expected results and their comments have been omitted from the natural language questions shown in Appendix \ref{app:natural-language-questions} and included directly in the discussion of the results in Subsection \ref{subsec:clinical-validation-results}, alongside the system's outputs.
If the system's outputs conform to the experts' preconceived ideas in a high number of cases (as confirmed by the experts themselves) then the system can be said to have been \textit{clinically validated}.
This is important because the enabling condition for the user to trust the predictions made by the software is that these shouldn't be in strong discordance with her existing beliefs.
Not having a strong \textit{cognitive dissonance} is a \textit{necessary} - but not sufficient - condition to enable trust and therefore explainability.

\subsection{Explainability Validation} \label{subsec:explainability-validation}
In general, there is strong resistance to novelty in the field of medicine, both for ethical reasons and because of the need for clinicians to be conservative in attending to established best practices in the field.
Any tool that is too onerous in terms of time and cognitive load is liable to remain underutilised.
In this field, \textit{a tool must therefore only be the means by which a question is answered}, not itself become a question; the methods developed in this thesis aim to conform to this objective, barring the experimental nature of the software and the consequent lack of refinement of its interface.
The need for a comprehensible and efficient tool is especially present because the goal of a pathologist is to arrive at a diagnosis, containing the elements useful to define prognosis and therapeutical approach, in the briefest time possible.
The main reasons are ethical, since for a patient waiting for a report is extenuating, and clinical, because a timely diagnosis is the first factor at the base of life expectancy.
Obviously, the highest possible accuracy is always strived for.
The clinical field and that of biomedicine are forced to embrace uncertainty, as this is an integral part of their practice.
Consequently, any tool able to support in comprehension and decision-making is automatically useful, once it has been clinically validated; in other words, even though a specific system may not be decisive or applicable to all reviewed cases, it will nonetheless be taken into account.

Thus, a system validated in terms of its adherence to clinical literature could then also meaningfully be validated from an explainability point of view.
The main question to be addressed is its capacity to relate to the expert user.
Is the system able to engender the user's trust?
In doing so, is she able to extract more knowledge from existing data when using the system than not?
Especially in cases where there may be a dearth of data, can the expert maximise the benefit from the available information?
Does the user subjectively feel that the system may positively impact her work?
These are all hard questions to answer, as there is a very high degree of subjectivity involved.
Thus to attempt to answer them, the chosen methods were borrowed from the social sciences.

In an earlier stage, the experts were introduced to the system in prototype form and instructed on the use cases it offered.
This process would enable the collection of feedback on the functionalities of the system and help in shaping its subsequent design.

The finalised system was, in a later phase (early August 2019), provided to the experts at the ICP for use in their daily work.
To quantify the performance of the system, as perceived by its users in a real setting over an extended period of time, a follow-up was done after three weeks by way of an \enquote{explainability evaluation questionnaire}, designed to test the gaps identified in Chapter \ref{chap:literature-review}.
The full questionnaire can be found in Appendix \ref{app:questionnaire}.

The \enquote{explainability evaluation questionnaire} presents five sections:
\begin{itemize}
  \item \textit{confidence}: aimed at assessing wether the use of the system incremented the confidence the clinician felt in making her decisions;
  \item \textit{features}: to understand in more detail which interaction modes were perceived as most useful and the subjective reasons for this.
  Of particular interest is the understanding of the perceived quality of the dialogical interaction modes and of the \enquote{pseudo-MPE} query;
  \item \textit{time}: questions focusing on the the temporal element, mainly the time needed to understand various explanations offered by the system.
  This element is often overlooked in the relevant xAI literature (see Section \ref{sec:explainability-in-bayesian-networks});
  \item \textit{tool}: general questions regarding the use of tool and if any important use-case was felt to be missing;
  \item \textit{clinical}: investigating if the tool was clinically relevant in day-to-day work.
  Unlike the clinical validation presented in Subsection \ref{subsec:clinical-validation-methodology}, these questions investigate \textit{a posteriori} the use of the tool and as such should provide a broader evaluation of its clinical relevance;
  \item \textit{satisfaction}: simple question asking to rate the general satisfaction with the proof of concept system.
\end{itemize}

As discussed throughout Chapter \ref{chap:literature-review} and summarised in Section \ref{sec:literature-review-summary}, one of the main gaps in the field of explainable AI is the absence of real-world validation of the - supposedly - explainable models.
The objective of the questionnaire is to act as an \textit{application-grounded evaluation}, in the taxonomy proposed by \citet{doshi2017towards} and presented in Section \ref{sec:evaluation-of-explainability}, and thus provide what is considered the gold standard for the evaluation of a machine learning system.
Also included, since it is almost always neglected in literature, is a focus on the \textit{temporal element} of the explanations that was noted as important by \citet{gilpin2018explaining}.
Of particular interest is evaluating the Bayesian network - underlying the tool's capabilities - in its capacity to surface cogent explanations for the target user; the questionnaire inflects the questions in order to identify which particular characteristics of the system and BN were perceived by the user as the most useful in order to gain an understanding of the underlying data set.
As noted in Section \ref{sec:explainability-in-bayesian-networks}, by acknowledging the psychological characteristics of an explanation identified by \citet{miller2018explanation}, explanations have various essential characteristics that seem to also be inherent in BNs; the questionnaire thus seeks to understand if these are actually present and perceived as useful, in the sense of enabling explainability, by the domain experts.

The questionnaire is not the only source of the results relating to the \textit{application-grounded evaluation} of the developed system; similarly to \citep{stumpf2009interacting} in their \enquote{think-aloud experiment}, many results and details throughout Chapter \ref{chap:results} will be the outcome of observing and listening to the expert users while they were engaging with the system.
We refer to these as \enquote{informal explainability evaluation results} contrasting them with the \enquote{formal explainability evaluation results} that will be the outcomes of the questionnaire.

\section{Summary}
This chapter has presented a series of methods whose aim is to enable the creation of a proof-of-concept system inspired by the paper \enquote{Explaining the Most Probable Explanation} by \cite{Butz2018} and to validate this system from the point of view of its explainability.
This evaluation with expert pathologists at the Istituto Cantonale di Patologia (ICP), whose results will be presented in Chap. \ref{chap:results}, aims at validating the methods proposed by \cite{Butz2018} and to act as a methodological framework for future work.
This is important because, as discussed in Chap. \ref{chap:literature-review}, the lack of evaluation of explainability is one of the main gaps in the xAI literature.

The chapter opened by presenting the data set that was supplied by the ICP and how this was integrated into the system and used to learn the structure and Conditional Probability Tables of the Bayesian Network, based on the Pomegranate Python package.
A series of standard-based algorithms have been presented to learn the BN from the data, to calculate the \textit{d-separation} between sets of nodes in the BN and to use the external solver DAOOPT to calculate the solutions to the MPE problem.
After these, a set of novel algorithms that constitute the core of this thesis have been presented: three variants of the \textit{dialogue} that is the realisation of the interaction method proposed by \cite{Butz2018} together with a procedure to generate alternative explanation branches to the \enquote{knowledge base} when and if the expert user dissents with the system on a proposal.
The remaining two algorithms are connected to evaluating the \enquote{pseudo-MPE} as compared to the true MPE solution.
Then, the rationale relating to the user interface has been presented together with the boilerplates in Extended Backus-Naur form used to generate the natural language outputs of the system.
Connected to interfacing with the user, the method for calculating and displaying pairwise \textit{mutual information} between the variables in the BN is also introduced.
The final topic presented is the evaluation methodology that will be used to assess the proof-of-concept system from the point of view of both its adherence to clinical literature i.e., its capacity to give outputs coherent with the experts' beliefs, and its explainability: its capability to explain its outputs to the users of the ICP and to support them in their daily work.

%%%%
%%%% RESULTS
%%%%
\chapter{Results}\label{chap:results}
\section{Introduction}

\section{Implemented Tool} \label{sec:implemented-tool}
\subsection{Overview}
The system referenced in Section \ref{sec:novel-contributions} and especially in Subsection \ref{subsec:interfacing-user}, is a proof of concept terminal-based software tool that was developed in order to test the hypotheses referenced in the \nameref{sec:introduction-results} to this chapter and laid out in Sections \ref{sec:response} and \ref{sec:methodology-introduction}.
The whole software tool has been made freely available for research purposes\footnote{\url{https://github.com/Tioz90/Bayesian-Networks-Explainability-Tool}}.
The major goal in the creation of such a tool is to have a working software that could be given to a number of clinicians at the ICP (see Subsection \ref{subsec:istituto-cantonale}) in order to carry out the research program defined in Section \ref{sec:response}, which is a response to the gaps identified in Chapter \ref{chap:literature-review}.
This being only a prototype, the implementation was carried out using Python, as this was the language that enabled the best focus on rapid development, due to its familiarity and to its vast array of available libraries.

Despite never having been intended to be production software, particular care was taken in the design of the interfacing methods, as described in Subsection \ref{subsec:interfacing-user}, in line with the spirit of this work that is to study human-machine interaction.

In the following section, the various interaction modes that were developed are presented using screenshots.
The basic methods underlying the software tool have already been discussed at length in Section \ref{sec:novel-contributions} so the current examination will focus on the user interface and how these methods have been incorporated into the system.
Where relevant, the information and descriptions given in Section \ref{sec:novel-contributions} will be integrated.

Figure \ref{fig:sw_0} shows the initial screen presented during use.
The user can input the path to the data set to use or can accept the hardcoded one, which in this case is the one described in Section \ref{sec:data-set}.
Next, the number of entries before and after preprocessing are shown; the data set in question sees its number of valid records go from 3217 to 2873, after the rules summarised in Table \ref{tab:datasetpreprocess} have been applied.
The \enquote{Inspect data set} and \enquote{ML} options are only for testing purposes; the former surfaces a pair of options to visualise the distribution of the data set's variables' values and their normalised entropies, the latter runs a series of tests that will not be discussed.
%the latter runs the machine learning tests that will be discussed in Section \ref{sec:bn-prediction-evaluation}.

The user-oriented section of the software is the one accessed by selecting \enquote{Build Bayesian Network}; this is where all the methods discussed in Chapter \ref{chap:methodology} are to be found and will be the object of the present evaluation.
Selecting this option automatically uses the Pomegranate package to construct a Bayesian network model using the previously selected data set.
The user is then shown the main menu of the application, as can be seen in Figure \ref{fig:sw_1}.

\begin{figure}[htbp]
\centerline{\includegraphics[width=0.7\textwidth]{results/images/sw_0}}
\caption{Initial screen in the developed tool.}
\label{fig:sw_0}
\end{figure}

\begin{figure}[htbp]
\centerline{\includegraphics[width=0.7\textwidth]{results/images/sw_1}}
\caption{Main interaction menu.}
\label{fig:sw_1}
\end{figure}

\subsection{Plot Model}
The \enquote{Plot Model} interaction mode would be an example of a \textit{static}, \textit{graphical} explanation in the framework defined by \citet{lacave2002review}, aimed at \textit{explaining the model}.
Compared to the characteristics of an explanation identified by \citet{miller2018explanation}, this might be erroneously regarded as a \textit{causal} explanation. 
Yet, it is important to remember that the directed graphs underlying a Bayesian network are not necessarily describing causal relations, but only probabilistic dependencies.

The \enquote{Plot model} interaction mode gives the expert an overview of the variables present in the system and their relationships by displaying the underlying BN's DAG, with the directionality of edges removed for the reasons explained in \ref{subsec:results-independencies-dialogue}.
Apart from the DAG, mutual information (Definition \ref{def:mutual-information}) between every pair of connected variables is shown on the edges in order to help the expert gauge the strength of the connection.

The users at the ICP considered this interaction modality a good solution to immediately visualise all the features of the data set at a high level together with their relationships; i.e., it gave the user a sense of the \textit{context} of the data set at hand. 
The scaling of the thickness of an edge in a manner proportional to the mutual information of the variables it connects was also considered useful in helping to appreciate the varying strength of the correlations between clinical variables.

The output for the data set presented in Section \ref{sec:data-set} in shown in Figure \ref{fig:sw_plot_result}.

\begin{figure}[htbp]
\centerline{\includegraphics[width=\textwidth]{results/images/plot_result}}
\caption{Plot model output.}
\label{fig:sw_plot_result}
\end{figure}

\subsection{Independencies} \label{subsec:results-independencies-query}
The \enquote{Independencies} interaction mode would be an example of a \textit{static}, \textit{linguistic} and \textit{graphical} explanation in the framework defined by \citet{lacave2002review} aimed at \textit{explaining the model}.
Compared to the characteristics of an explanation identified by \citet{miller2018explanation}, it could be seen as possessing the \textit{selected} and \textit{causal} elements.

The \enquote{Independencies} interaction mode gives the expert the possibility of verifying which d-separations (Definition \ref{def:d-separation}) exist in the constructed Bayesian network's DAG (Definition \ref{def:dag}) .
The concept of d-separation is here reworded into a higher-level notion of \enquote{choosing a source variable and a set of evidence to see which other variables have influence on the source, given the evidence}.
This recasting was deemed necessary because the clinicians at the ICP initially had difficulty in conceptualising at the level of graph theory, probably due to the the misinterpretation of the directionality of the edges in the graphs of a Bayesian network.
Graphs and trees are quite widely-used in clinical practice; however, the presence of an edge is commonly interpreted not as a correlation, but often as an indication of causality. 
Thus, the concept of d-separation could be misinterpreted because of this consolidated viewpoint.
For this same reason, after having chosen first the source variable and then the observed set of evidence variables, the user is presented with an output both in graph (Figure \ref{fig:independencies_output}) and in natural language (Figure \ref{fig:sw_2_independencies}) form.
Having both output modalities present was seen to reduce the confusion that users trained in the medical sciences felt for such an unfamiliar concept.

The new visualisation of d-separation (see Figure \ref{fig:independencies_dialogue_output}), introduced on the basis the the ICP's suggestions, was confirmed by the clinicians to be very intuitive, especially when compared to the initial design (see Figure \ref{fig:independencies_dialogue_output_old}).

\begin{figure}[htbp]
\centerline{\includegraphics[width=0.7\textwidth]{results/images/sw_2_independencies}}
\caption{Independencies query natural language output.}
\label{fig:sw_2_independencies}
\end{figure}

\begin{figure}[htbp]
\centerline{\includegraphics[width=\textwidth]{results/images/independencies_output}}
\caption{Independencies query graph output.}
\label{fig:independencies_output}
\end{figure}

\subsection{Conditional Probability Query} \label{subsec:results-conditional-probability-query}
The \enquote{Conditional Probability Query} would be an example of a \textit{static}, \textit{linguistic} explanation in the framework defined by \citet{lacave2002review} mainly aimed at \textit{explaining the evidence}.
Compared to the characteristics of an explanation identified by \citet{miller2018explanation}, it could be seen as possessing the \textit{selected} element.

Conditional probability queries (Definition \ref{def:conditional-probability}) were seen to be instinctively understood by the clinicians at the ICP.
Indeed, many of the natural language questions that they defined to clinically validate the system (see Subection \ref{subsec:clinical-validation-methodology}) could be framed as and answered by instances of this type of query.

As can be seen in Figure \ref{fig:sw_3_query}, the user is asked for a target variable (in magenta) of which to observe the conditioned values and for a set of variables, together with their observed values (in green).
The output, in natural language, includes all elements of the query together with the colour-coding described in Subsection \ref{subsec:interfacing-user}.
The answer to the question (in cyan), shows the probability of each of the states of the target variable quantified in natural language i.e., as linguistic probabilities, using the coding defined in Table \ref{tab:naturallanguageprobabilities}, and as raw probabilities, shown as percentages.

In addition, the use of colours was appreciated by the users at the ICP because they felt that it helped them to orient themselves among the different elements of the query and also to remember how they had posed it.  

\begin{figure}[htbp]
\centerline{\includegraphics[width=0.7\textwidth]{results/images/sw_3_query}}
\caption{Conditional probability query output.}
\label{fig:sw_3_query}
\end{figure}

\subsection{MPE Query}
The \enquote{MPE Query} would also be an example of a \textit{static}, \textit{linguistic} explanation in the framework defined by \citet{lacave2002review} mainly aimed at \textit{explaining the evidence}.
Compared to the characteristics of an explanation identified by \citet{miller2018explanation}, it could be seen as possessing the \textit{selected} element.

Queries of the MPE type (Definition \ref{def:mpe}) were not initially understood until a bridge to concepts familiar to clinical practitioners had been established.
When presented at an abstract, mathematical level, the experts of the ICP were not sure of the utility of such a query class.
With some work, it was understood that an MPE query could be linked to a concept familiar to any clinician: that of \enquote{a maximally likely patient profile}.
That is, given a set of known parameters it is of interest for the clinician to find which is the most likely assignment to the others.
As each record in the data set represents a patient's clinical profile, this is equivalent to finding the most probable patient given a set of know values.

Another way than an MPE query makes clinical sense, is in the crucial task of predicting missing values for a patient.
This is not an unlikely case, as discussed in Subsection \ref{subsec:motivation}, because there is more than one reason that patients may be missing one or more entries in their clinical profiles.
Executing an MPE query with the known patient's values will yield the most probable assignments to the missing ones and is thus equivalent to a prediction task.
The clinical significance of such an interaction mode can also be inferred from the fact that a number of the natural language questions, that were spontaneously defined to validate the system (see Section \ref{sec:validation}), were seen to map onto instances of this type of query.

At a technical level, the MPE calculation is executed using Pgmpy's \texttt{map\_query} function. 

The output, shown in Figure \ref{fig:sw_4_mpe}, presents, in colour-coded natural language, the input evidence (in green) and most probable assignments to the remaining variables (in cyan).

\begin{figure}[htbp]
\centerline{\includegraphics[width=0.7\textwidth]{results/images/sw_4_mpe}}
\caption{MPE query output.}
\label{fig:sw_4_mpe}
\end{figure}

\subsection{Pseudo-MPE Query} \label{subsec:results-pseudo-mpe-query}
The \enquote{Pseudo-MPE Query} would be an example of a \textit{static}, \textit{linguistic} and \textit{graphical} explanation in the framework defined by \citet{lacave2002review} mainly aimed at \textit{explaining the evidence} but also the \textit{reasoning}.
Compared to the characteristics of an explanation identified by \citet{miller2018explanation}, it could be seen as possessing the \textit{selected} and \textit{causal} elements, while remembering that the implications resulting from a BN are not necessarily causal.

The \enquote{pseudo-MPE query} interaction mode is aimed at generating a \enquote{maximally probable} assignment using the methods described in Subsection \ref{subsec:algorithms-novel} under the \enquote{pseudo-MPE from Initial Evidence} header.
The hypothesis is that this should be a valid explainability tool, as it is not only a \textit{linguistic} but also a \textit{graphical} explanation, with the latter element being identified by \citet{lacave2002review} as one of the most effective ways of giving a satisfactory explanation in a BN.\footnote{If the cardinality of the set of variables to explain is one, i.e., $|E| = |V|-1$, with $E$ the evidence set and $V$ the set of variables in the BN, then the \enquote{pseudo-MPE} and true MPE assignments will be identical.}

The user is first asked for the probability threshold under which to discard the \textit{(state,value)} pairs whose probability is deemed too low.
Then, after being asked for the initial observed evidence, the expert is presented with the constructed polytree (Definition \ref{def:polytree}); an output example can be seen in Figure \ref{fig:pseudo_mpe_output}.
This polytree will have the initial evidence, that the expert specified, as roots and a single chain of \textit{(state,value)} pairs, each one quantified with its probability (in natural language) given all of its ancestors.

A doubt, that presented itself quite early during the ICP's evaluation, concerned the quantification of probabilities in the chain.
The pathologists were unsure of why \textit{(state,value)} pairs appeared before others that had been reported as more probable.
For example, in Figure \ref{fig:pseudo_mpe_output}, \textit{(\enquote{morfologia},\enquote{Infiltrating duct carcinoma})} that is considered \textit{likely} appears before \textit{(\enquote{recettori estrogeni},\enquote{fortemente positivo})} which is considered \enquote{very likely}.
The pathologist's intuition brought her to expect this deduction chain to be monotonically decreasing in probability from the initial evidence (that, as such, is certain).

What turned out to be the point of confusion, was that it was unclear that the probability of every node added to the chain depends on all its ancestors.
In the specific example, \textit{(\enquote{recettori estrogeni},\enquote{fortemente positivo})}'s evidence set also contains \textit{(\enquote{morfologia},\enquote{Infiltrating duct carcinoma})}.
There is thus, mathematically, no reason for the chain to be monotonically decreasing in probability because adding new evidence is liable to boost the likeliness of some unobserved variables.
Returning to the example, the marginal probability $\mathbb{P}((\text{\enquote{recettori estrogeni},\enquote{fortemente positivo}}))$ may very well have been less probable than \enquote{very likely}, maybe it was only \enquote{likely} or even \enquote{unlikely}, but this says nothing about the posterior probability $\mathbb{P}((\text{\enquote{recettori estrogeni},\enquote{fortemente positivo}}) \mid (\text{\enquote{morfologia},\enquote{Infiltrating duct carcinoma}}) )$, which is what the polytree displays.

The unclearness of the chain of inferences is certainly not a point to underestimate, as the hope was for the \enquote{pseudo-MPE} output to be able to clarify the underlying reasoning process of the models and thus help in guiding the expert's through process.
If this reasoning process itself were unclear, this could hardly lead to a good explanation; thus an effort will have to be made to explain the underlying assumptions better, while also being mindful when evaluating if this output mode genuinely presents the characteristics of a good explanation.

\begin{figure}[htbp]
\centerline{\includegraphics[width=0.7\textwidth]{results/images/pseudo_mpe_output}}
\caption{Pseudo-MPE query output with threshold $0.5$.}
\label{fig:pseudo_mpe_output}
\end{figure}
%
%\subsection{Belief revision}
%\todo[inline]{da lasciare?  da espandere?  da spostare?}
%Philosophically, what is being done during the dialogues is a process of \textit{hard belief revision}.
%This is because the expert's choice is taken by the system as absolute truth and added to the previous evidence set.
%
%The process of \textit{belief revision} is distinct from that of \textit{belief updating} as in the latter previous beliefs are updated to conform to new information received.
%Belief revision on the other hand does not modify previous evidence, as this is simply supposed to be less reliable than the newer one.
%This last setting is the one that happens during the dialogues, because previous \textit{(state,value)} tuples are not modified in their probabilities given the new evidence supplied when the expert accepts a proposal.
%We talk about \textit{hard} belief revision because the new evidence generated by the user is treated as if it were certain - i.e., with probability 1 - and added to the evidence set without creating inconsistencies.
%\todo[inline]{come si potrebbe giustificare?  ha senso tenere il paragrafo?}

\subsection{Exhaustive Dialogue} \label{subsec:dialogue-results}
The three dialogue variants would be examples of \textit{dynamic}, \textit{contrastive}, \textit{linguistic} and \textit{graphical} explanations in the framework defined by \citet{lacave2002review} aimed at \textit{explaining the evidence} but also, and most importantly, to \textit{explaining the reasoning}.
Compared to the characteristics of an explanation identified by \citet{miller2018explanation}, these could be seen as possessing all the necessary elements: \textit{contrastive}, \textit{selected}, \textit{causal} and \textit{social}.

The three \enquote{dialogues} are the most experimental interaction modes and thus also the most alien to a user.
None of the natural language questions defined by the ICP in the form described in Subsection \ref{subsec:clinical-validation-methodology} could be directly mapped onto such a dialogical process.
On the other hand, the dialogue aims to build an \textit{expert-driven MPE approximation} and could thus be regarded as essentially answering the same question as the \enquote{pseudo-MPE} and \enquote{MPE} queries (Subsection \ref{subsec:results-pseudo-mpe-query}).
The research hypothesis is wether this could be a better explainability tool, as it is not only a \textit{linguistic} and \textit{graphical} explanation but also a \textit{dynamic dialogue} that \citet{Hilton1990} and \citet{lacave2002review} identify as a key ingredient in having an effective explanation.
Another important fact is that the dialogues offer a \textit{counterfactual} branch when the expert dissents with the model; \citet{miller2018explanation} singles out being \textit{contrastive} as one of the defining characteristics of an effective explanation, as this feature closely aligns with our expectations of what an explanation should entail.

Because of the novel nature of such a knowledge-extraction process, three different versions were implemented with each one adding a different set of behaviours to the \enquote{exhaustive} version described in this subsection.
This helped in exploring the space of possibilities and aided in understanding which features were preferred by the clinicians of the ICP, both as a means for knowledge-extraction from the data set and from a comprehensibility point of view.
It should here be noted that comprehensibility of the outputs is a \textit{necessary} but \textit{not sufficient condition} to be able to gain knowledge from data.
Both variants to the basic dialogue - the independencies-aware and the thresholded one - aim to prune the space of variables proposed to the user in order to reduce her cognitive load.
This is in keeping with the insight by \citet{miller2018explanation} that explanations are \textit{selected}, meaning that we humans expect that the explaining factors be picked based on some criterion.

The \enquote{exhaustive dialogue}, as described in much more detail in Subsection \ref{subsec:algorithms-novel} under the \enquote{Dialogues} header, is so named because it ends only when the expert user has reviewed all the variables present in the data set.
It starts by asking the clinician for a set of initial evidence and from thereon after iteratively proposes the \textit{(state,value)} pair with the least entropic \textit{state}, based on the accumulated evidence (the rationale behind this is explained in Subsection \ref{subsec:entropy-based-selection}).
An example of such an ongoing interaction is shown in Figure \ref{fig:sw_5_exhaustive_dialogue}.

An issue that was highlighted early on was that the experts had great trouble in building the \textit{knowledge base} from a single evidence; this was the driving motive that pushed the representation of the \enquote{pseudo-MPE} branch beyond a simple \textit{tree} (Definition \ref{def:tree}) - as in \citep{Butz2018} - but towards a \textit{polytree} (Definition \ref{def:polytree}).
This way the expert is able to inject the query with as much domain knowledge as she feels comfortable with.
It was an unrealistic assumption to expect a domain expert to bear the cognitive load of selecting a \textit{single best initial evidence}; this would be a hard task to do in its own right but it is made even more difficult by the fact that the subsequent dialogue \textit{depends} on the initial evidence.
To effectively select one best evidence, the expert should also have been able to \textit{predict} how the dialogue would have evolved from that initial point onwards.
The dialogue is an \textit{exploratory tool} that the user utilises with the objective of extracting knowledge from the data set; expecting the user to already know the outcome of her choices would mean that she already had the domain knowledge necessary to predict the consequences of those same choices; a clear instance of \textit{circular reasoning}.
This was confirmed by the ICP: having multiple initial evidence helped the users because it reduced the number of tuples proposed by the system and therefore the quantity of choices the users were tasked to deal with.

The general feeling being echoed by the users at the ICP was that the dialogue was the hardest interaction mode to understand and to utilise.
The way they used the dialogues was by nearly always replying \enquote{yes} to its proposals, because they were mostly interested in seeing what the machine would propose.
Nonetheless, the users understood the high potential of this method especially when applied with the objective of conducting research, but they reported they would need to \enquote{trust} the interaction mode before feeling comfortable with using it in such a manner.
They felt that probably having more time to experiment with this kind of interaction might improve their confidence felt in using it, that was lower than that perceived for the other interaction methods.

\begin{figure}[htbp]
\centerline{\includegraphics[width=0.7\textwidth]{results/images/sw_5_exhaustive_dialogue}}
\caption{Ongoing Exhaustive Dialogue.}
\label{fig:sw_5_exhaustive_dialogue}
\end{figure}

\subsection{Independencies Dialogue} \label{subsec:results-independencies-dialogue}
The first variant to the \enquote{exhaustive dialogue} takes the approach of excluding variables based on their d-separation properties (Definition \ref{def:d-separation}) in the underlying DAG (Definition \ref{def:dag}).
Thus the cardinality of the set of variables proposed to the user varies in a non-linear way, depending on the topology of the graph and the order of insertions into the evidence set.
In the \enquote{exhaustive dialogue}, presented under the previous header, the relationship between the set of variables still to explain at step $t$, $W_t = V \setminus E_t$, with $V$ all the variables and $E_t$ those already added to evidence, obeys the recurrence relation:
\begin{align}
\begin{split}
		W_0 := V, \\
	E_0 := \emptyset, \\
	|W_{t+1}| = |W_t| - 1, \\
	|E_{t+1}| = |E_t| + 1.
\end{split}
\end{align}
That is, at each step $t$ of the \enquote{exhaustive dialogue}, one variable moves from the set still to explain $W$ to the explained one $E$ i.e., after any iteration step, the number of instantiated variables increases by one unit, while the number of variables to explain also decreases by one.
In the \enquote{independencies dialogue}, this relationship depends on the set of variables $Z$ that are d-separated from those already in $E$.
The relationship between the cardinalities is modelled by an operator $\zeta$ that is unique to the DAG of the BN (or to any \textit{i-equivalent}\footnote{I-equivalence identifies classes of graphs that present the same d-separation properties.} one):
\begin{align}
\begin{split}
	W_0 := V, \\
	E_0 := \emptyset, \\
	|W_{t+1}| = \zeta(|E_t|), \\
	|E_{t+1}| = |E_t| + 1.
\end{split}
\end{align}
As d-separation is not monotonic (adding a variable to $E$ may open new paths and d-connect new variables), the cardinality of the set $W$ may vary, from the point of view of the user, in an unpredictable manner.
To attempt to offset this effect, during the dialogue the user is supported by an updated view of the independencies in the graph (an example during the dialogue is shown in Figure \ref{fig:independencies_dialogue_output}).

Before receiving feedback from the ICP, the visualisation of the independencies was the one shown in Figure \ref{fig:independencies_dialogue_output_old}.
The most striking difference was the use of colour-coding to identify the role and the separation of variables with pink identifying the query variables, blue the evidence, red the separated variables and green the connected ones.
As already noted in Subsection \ref{subsec:results-independencies-query}, the concept of d-separation turned out to be quite unfamiliar to the clinicians of the ICP so the first priority was to represent the concept visually in the clearest way possible.
This was achieved, and confirmed in its efficacy by the pathologists, by fading the separated variables and marking those in evidence in bold, as can be seen in Figure \ref{fig:independencies_dialogue_output}.
The fading of separated variables was felt to successfully reinforce the concept of these not influencing the remaining ones and its directness was especially appreciated.

The use of directed arcs to represent the DAG raised another critical issue that hadn't been foreseen.
In the visual representation of a Bayesian network, an arc between two variables represents a correlation between their values while the direction identifies the \textit{parent} and the \textit{child} in the relationship; for example the graphical representation $X \rightarrow Y$ means that $X$ is the parent of $Y$.
This is a defining characteristic of such a model, because the fundamental idea of a BN is to factorise the joint distribution such that each variable's values depend only on that of its parents; the concept of conditional probability table is explained in Section \ref{sec:bayesiannetworks} and some more examples can be seen in Subsection \ref{subsec:algorithms} under the \enquote{MPE} header.
Nonetheless, the pathologists explained that the DAG representing a BN is very similar to diagrams used during clinical research, with the crucial difference that in those a directed arrow represents \textit{causation} and not \textit{correlation}.
In these diagrams a correlative relationship would have usually been represented by an undirected edge.
For this reason the DAG representation of the BN was \textit{disoriented} in all visualisations.
The ICP confirmed that this new formulation was more closely aligned with the intuition that could be expected by a clinician.

The third element of difference, is the addition of the mutual information coefficient (Definition \ref{def:mutual-information}) on the arcs connecting each couple of variables; the coefficient also scales the width of its associated edge, giving further visual feedback to the user.
This functionality was a direct request from the ICP's representatives since after inspecting the initial DAG visualisation they felt the need for a feature that would increase their understanding of the relationships between the variables.
D-separation is binary while mutual information can give the practitioner a much wider (theoretically infinite) range of information.
For example looking at Figure \ref{fig:independencies_dialogue_output} it is quite easy to see that, while \enquote{morfologia} and \enquote{recettori estrogeni} are d-connected, the amount to which they influence each other's values is small compared to other connected variables.
Some arcs are missing the mutual information coefficient because one of the two variables is in the evidence set, e.g., those belonging to \enquote{recettori estrogeni}.

\begin{figure}[htbp]
\centerline{\includegraphics[width=\textwidth]{results/images/independencies_dialogue_output}}
\caption{Ongoing Independencies Dialogue.}
\label{fig:independencies_dialogue_output}
\end{figure}

\begin{figure}[htbp]
\centerline{\includegraphics[width=\textwidth]{results/images/independencies_dialogue_output_old}}
\caption{Previous visualisation during Independencies Dialogue.}
\label{fig:independencies_dialogue_output_old}
\end{figure}

\subsection{Thresholded Dialogue}
The final \enquote{dialogue} variant adopts a different strategy for pruning; namely one based on the probability of the proposed tuples and on the number of times they have been refused by the expert.
This implements a suggestion found in \citet{lacave2002review} that explanations should be graded on the user and not on a \textit{fixed user model}; one of the ways this has been addressed in literature is by the introduction of thresholds to filter unwanted information.

The cardinality of the set $W$ of states to explain decreases linearly, similarly to the \enquote{exhaustive dialogue}, but potentially with a slope coefficient $\alpha \leq -1$, as many states may be infra-threshold i.e., too improbable to be considered.
Unlike the independencies dialogue, the cardinality of $W$ cannot increase:
\begin{align}
\begin{split}
	W_0 := V, \\
	E_0 := \emptyset, \\
	|W_{t+1}| = \alpha |W_t|, \\
	|E_{t+1}| = |E_t| + 1.
\end{split}
\end{align}
The default values for the threshold and the maximum number of times a \textit{(state, value)} tuple could be proposed were decided together with the ICP and set to:
\begin{itemize}
  \item \textit{threshold}: 0.4, a \textit{(state, value)} tuple is ignored if the probability of \textit{value} is less that 0.4;
  \item \textit{refusal limit}: 2, a \textit{(state, value)} is ignored if it has already been refused twice.
\end{itemize}
\section{Bayesian Networks Predictions Strength} \label{sec:prediction-evaluation}
\section{Validation results}
\subsection{Clinical Validation}
It was, for example, observed that the approach under investigation has particular relevance in contextualising the variables \enquote{mut17q21} and \enquote{loss17}.
These variables represent two different types of mutation of chromosome 17 that could determine peculiar phenotypes depending on both their intrinsic characteristic and the results of the combination with other variables. 
Even though they can be relevant in profiling patients’ cancer, their diagnostic role it is still not clear. 
Moreover, the clarification of their clinical role could also aid in the elucidation of the pathological mechanism of action, thus allowing the rational design of the intervention and therapy. 
Hence, extracting information in terms of the relationship between variables could help not only in understanding the function of these variables in patients profiling but also in helping the potential definition of novel, optimised guidelines concerning the best practice in the presence of evidence. 
Indeed, the possibility to define the \enquote{informational flux} of the explanation depending on the evidence could allow a more robust profiling, especially in the presence of partial knowledge or reduced resources (budget and time).
For example, Fig. \ref{fig:independencies_output} shows the graph for the query on \enquote{mut17q21}.
 
\todo{completare}
It is interesting to note how ….
Moreover, updating the graph on the evidence is possible.….

In current practice, it is quite common to have a single, standardised, certified procedure for the management of the diagnosis and treatment, independently of the evidence that is already present (or missing). 
Typically, it is a simple decision tree with a \enquote{yes/no} progression that is independent on the pieces of evidence that are already known. 
Information about d-separation could help to introduce a novel concept of data managing \enquote{tuned} on the specific case.
It is worth highlighting that the available molecular biomarkers undergo a process of continuous updating thanks to ever more accurate and accessible high throughput screening. 
Thus, new variables can be integrated, for specific patients, in the presence or not of evidence, and this can progressively aid the onset of personalised medicine. 
At the same time, features that are already well established in the clinical practice could find new interpretation, allowing the creation of innovative hypothesis and thus of a clinical evolution. 
For example, despite it being conceivable that the independency of the variable \enquote{lateralita} (generally annotated for all the patients for a long time) from the other features have biological meaning, there is not, to date, a well establish validation of this hypothesis. 
In clinical practice, this modality could help in defining alternatives able to optimise time, cost or diagnosis. 
For example, it is quite common for a sudden cyto-histo-molecular analysis to have to be performed before surgery, in order for the doctors to decide on the best way to proceed.
 In this case, the priority of ICP is on the sample of the patient that is under surgery. 
 It is possible that the biological sample could be degraded or not in sufficient quantity to be able to complete all the necessary assays. 
 In this context, the importance of being able to obtain the information of interest in a quick but accurate way is self-evident.
The type of analysis that is to be carried out should be prioritised in order to be able to formulate a diagnosis.
At the same time, in case of degraded material the possibility to infer the missing value, in agreement with the opinion of the expert pathologist, could offer further basis and support for the clinician.

\subsection{Explainability Validation}



\section{Pseudo-MPE Evaluation}
This is not a user-facing feature \textit{per se} but a way of running a test to compare the outputs of the Pseudo-MPE algorithm with the exact solution (this is described in detail in Subsec. \ref{subsec:algorithms-novel} under the \textbf{MPE Algorithms Comparison} header.
I also implemented the possibility of scoring the MPE calculated with pgmpy's \texttt{map\_query} and with DAOOPT.
The Hamming and Jaccard distances between these should have been zero, as they both use exact methods to solve the MPE problem.
It was seen that this was not the case and this lead to the discovery of the problems described in Sec. \ref{sec:issues}.
As is explained in Sec. \ref{sec:issues}, the benchmark against which to compare the Pseudo-MPE was taken to be pgmpy's \texttt{map\_query} function.

The tests were run ...


\section{Pseudo-MPE Complexity}
\section{Issues} \label{sec:issues}

\subsection{Zero Probabilities in Learned CPTs}
It was noticed that some of the conditional probability tables that Pomegranate learned from the data set (i.e., by solving the structure learning problem defined in Subsection \ref{subsec:learning-bn-structure}) presented entries with $0$ value.

The assignment of the extreme probabilities $1$ and $0$, while perfectly coherent in the frequentist approach, is not in line with the Bayesian one.
This is because of the different conception of probabilities between the two approaches, as discussed in Subsection \ref{subsec:probability-interpretations}.
A frequentist practitioner would happily assign probability zero to an event not present in the data set while a Bayesian would refrain in doing so, as having a zero or one prior belief makes every posterior, calculated using Bayes' Rule (Definition \ref{th:bayes-theorem}), also zero or one.
The necessity of avoiding the assignment of prior probability beliefs equal to $0$ or $1$, has been named \enquote{Cromwell's rule} by \citet{Jackman2009}.

A useful distinction to make that could help in deciding when to accept extreme probabilities or not, is between \textit{a priori} and \textit{a posteriori} propositions; the former are those whose truth value is not empirical but can, in general, be deduced based on logical necessity alone; the latter are those whose truth value is based on experience, for example any statement regarding the physical world.
An example of \textit{a priori} proposition could be a tautology, such as \enquote{every wife is married}; an example of \textit{a posteriori} proposition could be an assertion regarding the state of the world, for example \enquote{yesterday it rained}.
It would be an epistemological error to believe in the absolute truth or falsity of any proposition that is not necessary and thus one should refrain from assigning absolute belief or disbelief to their truth value.
This is because, apart from issues regarding the \textit{ontological determinism of Reality} and of the \textit{contingency of experience}, it is also against the intended use of statistics; statistical methods should only be applied in cases of uncertainty and not in those where a \textit{deterministic mechanism} is implied.
If a proposition's truth value can be known without resorting to the senses, then its value does not depend on the state of things in the sensible world.
Thus, we can assign absolute confidence in the truth value of a priori statements.

To solve the issue in the context of this work, a simple post-learning correction was applied; specifically a small positive constant was added to every zero-valued entry in the CPTs in Pomegranate's model.
The methodologically correct approach would have been to apply \textit{Laplace smoothing}, a standard technique used to \textit{smooth} categorical data.
The method would have entailed adding a \textit{pseudocount} $\alpha$ to every empirical probability estimated from the data set.
Given $x_i$ the count of occurrences of event $i$ in a set of $N$ events, the un-smoothed empirical probability is:
\begin{equation}
	p_i = \frac{x_{i}}{N} \,,
\end{equation}
while the smoothed one would be given by:
\begin{equation}
	p_i^*=\frac{x_{i}+\alpha}{N+\alpha d} \,,
\end{equation}
with $d$ the number of possible categories.

The value of $\alpha$ should be chosen to reflect any prior knowledge regarding the events; in the case there were none, a non-informative prior should be chosen, as stated by the \textit{principle of indifference} (in absence of any evidence one should distribute his beliefs uniformly).
The simplest possible non-informed approach is to increment every event's count in the data set by one, including the ones not appearing.
Thus the relative frequency between events will be maintained but there will be no event $i : x_i=0=p_i$.

Unfortunately implementing this approach, while recognisably the most methodologically sound way of proceeding, would have been very time-consuming and outside the main focus of this thesis.
The chosen strategy of adding a small positive constant $\epsilon$ to each empirical probability $p_i$, while not strictly correct, is extremely unlikely to change the learned CPTs.
This cannot be ruled out \textit{a priori} and would require \textit{sensitivity analysis} to be decided, but we are working with the prior belief that this is very unlikely to happen.

The developed algorithm, termed \enquote{epsilon smoothing}, is based on adding and subtracting \enquote{probability atoms} $\epsilon$ with the objective of removing zeros in the CPTs and maintaining the normalisation, so that the result will still be a valid probability distribution (based on Definition \ref{def:probability-measure}).
We work with probability atoms $\epsilon$ so as not to incur in numerical imprecisions in the implementation, as only additions and subtractions need to be used.
Every zero-valued element in a CPT column has a quantity $\epsilon \times \#!0$ added to it, with $\#!0$ the number of elements in the distribution that are not zero, and every non-zero entry has $\#0$, the number of zero elements, atoms subtracted from it.
The end result is obviously still a correctly normalised probability distribution because given a probability distribution $P$, with $n$ elements $p_i$ of which $\#0$ are zero, and the  distribution $P^*$ resulting from to the described procedure:
\begin{align}
	&\sum\limits_{i=0}^{n} p_i = 1 \\
	\wedge  \quad &1 - \{ \#0 \times [(n - \#0) \times \epsilon] \} + \{ (n - \#0) \times [\#0 \times \epsilon] \}  = 1 \\
	\implies \quad  &\sum\limits_{i=0}^{n} p^*_i = 1
\end{align}
The pseudocode is shown in Algorithm \ref{alg:epsilon-smoothing}.

\begin{algorithm}[htp!]
	\caption{Epsilon Smoothing algorithm pseudocode}
	\label{alg:epsilon-smoothing}
	\begin{algorithmic}[1]
		\State $\epsilon=$ smallest positive constant
		\For{$s$ in model CPTs}
			\For{$c$ in $s$'s columns} \Comment{distributions of values are organised column-wise}
				\State $num\_zeros$ = number of zero-valued entries in $c$
				\State $num\_non\_zeros$ = $|c|$ - $num\_zeros$
				\For{$v$ in $c$}
					\If{$v$ equal to $0$}
						\State $v += \epsilon \times num\_non\_zeros$
					\Else
						\State $v -= \epsilon \times num\_zeros$
					\EndIf
				\EndFor
			\EndFor
		\EndFor
	\end{algorithmic}
\end{algorithm}

If the procedure were applied to the CPT in Table \ref{tab:rec-cpd-issues}, it would yield the one shown in Table \ref{tab:rec-cpd-epsilon}.

\begin{table*}[htbp]
\centering
\caption{\enquote{recettori estrogeni} CPT}
\begin{tabularx}{0.5\textwidth}{ccXX}
\toprule
      & &  \multicolumn{2}{c}{\textbf{mut17q21}} \\
\cmidrule(lr){3-4}
 & & 0 & 1    \\ 
 \multirow{3}{*}{\textbf{rec. estr.}}  & 0 & $0.68 - \epsilon$ & 0.13  \\
 & 1 & $0.0 + 2\epsilon$ & 0.02    \\
 & 2 & $0.31 - \epsilon$ & 0.84 \\
\bottomrule
\end{tabularx}
\label{tab:rec-cpd-epsilon}
\end{table*}


\subsection{MPE Calculation} \label{subsec:results-mpe-calculation-issues}
The main issue encountered during the implementation of the system described in Section \ref{sec:implemented-tool}, was that of correctly calculating the MPE.
Initially, an attempt was made to write a custom function \texttt{export\_model\_to\_uai} to generate a UAI file (described in \ref{subsec:libraries}) directly from the Pomegranate Bayesian network model.
This UAI file was fed to DAOOPT to generate the MPE solution as recounted in Subsection \ref{subsec:algorithms} under the \enquote{MPE Algorithms Comparison} header.
When compared with the MPE solution generated directly using Pgmpy's \texttt{map\_query} function, it was seen that these disagreed in almost all cases.
This shouldn't have been the case as both were generated using exact methods: \textit{variable elimination} in Pgmpy's case and \textit{AND/OR branch-and-bound} for DAOOPT.

While investigating the cause for this divergence, an undocumented feature of Pgmpy was discovered: a \texttt{UAIWriter} class that should have converted the Pgmpy-based model (which was converted in turn from the Pomegranate-based model, as outlined in Subsection \ref{subsec:algorithms} at the \enquote{Pairwise Correlations} header) to the correct UAI file representing it.
When this alternative UAI file was used as input for DAOOPT, the resulting MPE not only diverged from that calculated based on \texttt{export\_model\_to\_uai}, as was to be expected, but also from that calculated directly with Pgmpy using \texttt{map\_query}, which was surprising.

The data flow for the MPE calculation is shown in Figure \ref{fig:mpe_conversion_process}; initially a Pomegranate-based BN is learned from the data set by using the built-in \texttt{from\_samples} method, then a custom \texttt{convert\_to\_pgmpy} function converts the Pomegranate-based model to a Pgmpy-based one.
The custom function \texttt{export\_to\_uai} and Pgmpy's built-in \texttt{UAIWriter} class are used to generate the \texttt{.uai} and \texttt{.uai.evid} files that are the input for DAOOPT to generate the MPE solutions.
The \texttt{map\_query} method that is part of Pgmpy's API is also used to generate an MPE assignment.

\begin{figure}[htbp]
\centerline{\includegraphics[width=\textwidth]{results/images/mpe_conversion_process}}
\caption{MPE calculation flow.}
\label{fig:mpe_conversion_process}
\end{figure}

The conversion from the Pomegranate-based to the Pgmpy-based model was thoroughly tested using conditional probability and independencies queries, so the issue is most likely to be found elsewhere.
As the DAOOPT MPE solution generated starting from the UAI differs from the one calculated directly with Pgmpy, there must be a bug either in Pgmpy's UAI exporter or in its inference method.
It is unclear if Pgmpy is still presenting issues in its inference methods\footnote{\url{https://github.com/pgmpy/pgmpy/issues/856}} but a series of tests on simple networks, where the MPE calculations were carried out manually, seemed to confirm that \texttt{map\_query} was returning the correct MPE solution.

For example, in the small BN whose structure is shown in Figure \ref{fig:issues-bn} and the CPDs of the nodes in Tables \ref{tab:mut-cpd-issues}, \ref{tab:eta-cpd-issues}, \ref{tab:rec-cpd-issues} and \ref{tab:diff-cpd-issues}, Pgmpy's \texttt{map\_query} and DAOOPT returned different solutions to the following MPE query:
\begin{align}
\begin{split}
	\text{MPE}( \text{\enquote{differenziazione}}=x, \text{\enquote{mut17q21}}=y, \text{\enquote{recettori estrogeni}}=z \\
	\mid \text{\enquote{eta arrotondata}}=0 )
\end{split} 
\end{align}
\texttt{map\_query} returned the assignment: 
\begin{align} \label{eq:pgmpy-assignment}
  (\text{\enquote{differenziazione}}=1, 
  \text{\enquote{mut17q21}}=1, 
  \text{\enquote{recettori estrogeni}}=2)
\end{align}
while DAOOPT on the UAI exported with Pgmpy's returned:
\begin{align} \label{eq:daoopt-assignment}
  (\text{\enquote{differenziazione}}=0, 
  \text{\enquote{mut17q21}}=1, 
  \text{\enquote{recettori estrogeni}}=2)
\end{align}

In such a small network it is easy to verify that the probability of Equation \ref{eq:pgmpy-assignment} is: $0.99 \times 0.84 \times 0.60 = 0.50$ and that it is the MPE solution.
The fact that the probability of Equation \ref{eq:daoopt-assignment} is $0.99 \times 0.84 \times 0.21 = 0.17$, makes it obviously incorrect.

\begin{figure}[htbp]
\centerline{\includegraphics[width=0.5\textwidth]{results/images/issues-bn}}
\caption{Independencies query natural language output.}
\label{fig:issues-bn}
\end{figure}

\begin{table*}[htbp]
\centering
\caption{\enquote{mut17q21} distribution}
\begin{tabularx}{\textwidth/3}{ccX}
\toprule
 \multirow{2}{*}{\textbf{mut17q21}} & 0 & 0.01  \\
 & 1 & 0.99 \\
\bottomrule
\end{tabularx}
\label{tab:mut-cpd-issues}
\end{table*}

\begin{table*}[htbp]
\centering
\caption{\enquote{eta arrotondata} CPT}
\begin{tabularx}{0.5\textwidth}{ccXX}
\toprule
      & &  \multicolumn{2}{c}{\textbf{mut17q21}} \\
\cmidrule(lr){3-4}
 & & 0 & 1    \\ 
 \multirow{3}{*}{\textbf{eta arr.}}  & 0 & 0.42 & 0.04  \\
 & 1 & 0.42 & 0.17    \\
 & 2 & 0.15 & 0.78 \\
\bottomrule
\end{tabularx}
\label{tab:eta-cpd-issues}
\end{table*}

\begin{table*}[htbp]
\centering
\caption{\enquote{recettori estrogeni} CPT}
\begin{tabularx}{0.5\textwidth}{ccXX}
\toprule
      & &  \multicolumn{2}{c}{\textbf{mut17q21}} \\
\cmidrule(lr){3-4}
 & & 0 & 1    \\ 
 \multirow{3}{*}{\textbf{rec. estr.}}  & 0 & 0.68 & 0.13  \\
 & 1 & 0.0 & 0.02    \\
 & 2 & 0.31 & 0.84 \\
\bottomrule
\end{tabularx}
\label{tab:rec-cpd-issues}
\end{table*}

\begin{table*}[htbp]
\centering
\caption{\enquote{differenziazione} CPT}
\begin{tabularx}{0.5\textwidth}{ccXXX}
\toprule
      & &  \multicolumn{3}{c}{\textbf{recettori estr.}} \\
\cmidrule(lr){3-5}
 & & 0 & 1 & 2   \\ 
 \multirow{3}{*}{\textbf{diff.}}  & 0 & 0.012 & 0.16 & 0.21  \\
 & 1 & 0.18 & 0.43 & 0.60    \\
 & 2 & 0.80 & 0.40 & 0.18 \\
\bottomrule
\end{tabularx}
\label{tab:diff-cpd-issues}
\end{table*}

\subsection{Late Removal of Clinical Variables} \label{subsec:removal-clinical-variables}
As discussed in Section \ref{sec:data-set}, it was decided to drop certain variables from the initial data set.
Among these, \enquote{mut17q21}, \enquote{loss 17} and \enquote{FISHRatio} were initially included in the post-processed data set and became part of the Bayesian network.
Thus, in the initial phases of development and validation the network topology was the one shown in Figure \ref{fig:old-bn-plot}, which can be compared with the current one shown in Figure \ref{fig:sw_plot_result}.

\begin{figure}[htbp]
\centerline{\includegraphics[width=\textwidth]{results/images/old-bn-plot}}
\caption{Bayesian network topology before the removal of \enquote{mut17q21}, \enquote{loss 17} and \enquote{FISH}.}
\label{fig:old-bn-plot}
\end{figure}

At a later phase a decision was made, in agreement with the ICP, to remove these three variables from the data set during the preprocessing phase; this was due to their being extremely skewed in their values, as can be seen by inspecting the \enquote{Distribution} column in Table \ref{tab:datasetdistribution}.

The reason for their skewness was briefly mentioned in Subsection \ref{subsec:motivation} and can be traced to the fact that these variables are all connected to the technique of \textit{fluorescence in situ hybridisation}.
This can be understood by analysing the \enquote{Clinical meaning} column in Table \ref{tab:datasetvariables} and knowing that FISH enables the analysis of specific DNA sequences on chromosomes and in particular of chromosome 17, because of its significance in breast cancer \citep{zhang2011important}.
FISH was a technique that was not available prior to 2010 and thus nearly 70\% of the patients in the data set had a value of \enquote{NCO} for \enquote{FISHRatio}, meaning this test had not been carried out on them.
Even worse, \enquote{mut17q21} presented more than 99\% \enquote{unknown} values and \enquote{loss 17} had 78\% of \enquote{FISH non fatta/FISH non valutabile}, meaning that the great majority of cases presented values with no real clinical meaning.

It can be seen in Figure \ref{fig:old-bn-plot} how strong the association between \enquote{loss 17} and \enquote{FISH} was - \textit{a posteriori}, a clear case of spurious correlation - and how feebly \enquote{mut17q21} was connected to the rest of the network.
Having these variables was introducing a very large amount of bias that confounded the resulting model.
An example of this effect was experienced in the early stages of validation; a previous series of validation questions had been prepared by the ICP and included queries in a form similar to:
\begin{quotation}
	In the general population, if \textit{[...]} is \textit{[...]}, then is it more/less probable that \textit{[...]} is \textit{[...]}?
\end{quotation}
and
\begin{quotation}
	In young patients, if \textit{[...]} is \textit{[...]}, then is it more/less probable that \textit{[...]} is \textit{[...]}?
\end{quotation}
Queries of these forms, when compared against each other, reliably returned identical answers, indicating that the age of the patient (\enquote{eta arrotondata} in the benchmark data set) had little or no influence of the values of other variables.
Inspection of Figure \ref{fig:sw_plot_result} will show how, after the removal of \enquote{mut17q21}, \enquote{loss 17} and \enquote{FISHRatio}, \enquote{eta arrotondata} actually becomes disconnected from the rest of the network.
The removal of the spurious \enquote{bridge} created by \enquote{mut17q21} now explicitly shows and confirms that patient age can have no influence on other variables' values, as had already been noticed before removal. 

The ICP confirmed that removing these three variables helped its users in better understanding the differences between a correlation in the variables representing the data set and a correlation in the actual clinical meaning behind the data.
That is, they felt that it reduced the \textit{confounding factors} thus allowing a better appreciation of the explainability methods used.
\section{Summary}


%%%%
%%%% CONCLUSIONS
%%%%
\chapter{Conclusions}\label{chap:conclusions}

\section{Reply to Introduction}
\section{Future developments}\label{sec:future-developments}
\todo{incorporate user feedback into the data set as in Interacting meaningfully with machine learning systems: Three experiments}


\appendix %optional, use only if you have an appendix


\backmatter

\chapter{Glossary} %optional

%\bibliographystyle{alpha}
%\bibliographystyle{dcu}
\bibliographystyle{plainnat}
\bibliography{biblio}

%\cleardoublepage
%\theindex %optional, use only if you have an index, must use
	  %\makeindex in the preamble


\end{document}
