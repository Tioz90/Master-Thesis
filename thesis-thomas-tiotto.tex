\documentclass[mscthesis]{usiinfthesis}
\usepackage{lipsum}
\input{macros}

\usepackage{tikz}
\usetikzlibrary{graphs}

\usepackage{natbib}

\usepackage{listings}
\newcommand{\Eta}{H}

\usepackage{tabularx}
\usepackage{ragged2e}
\newcolumntype{Y}{>{\RaggedRight\arraybackslash}X} 
\usepackage{booktabs}
\renewcommand\tabularxcolumn[1]{m{#1}}

\usepackage{algorithm, algorithmicx, algpseudocode}

\newtheorem{theorem}{Theorem}[chapter]
\newtheorem{definition}[theorem]{Definition} 
\newtheorem{corollary}[theorem]{Corollary} 

\lstdefinelanguage{algebra}
{morekeywords={import,sort,constructors,observers,transformers,axioms,if,
else,end},
sensitive=false,
morecomment=[l]{//s},
}



\title{Expert-driven approximation of MPE} %compulsory
\specialization{Artificial Intelligence}%optional
\subtitle{Subtitle: Reinventing the World} %optional 
\author{Thomas Francesco Tiotto} %compulsory
\begin{committee}
\advisor{Prof.}{Alessandro Facchini}{} %compulsory
\coadvisor{Prof.}{Alessandro Antonucci}{}{} %optional
\end{committee}
\Day{Yesterday} %compulsory
\Month{September} %compulsory
\Year{2019} %compulsory, put only the year
\place{Lugano} %compulsory

\dedication{To my beloved} %optional
\openepigraph{Someone said \dots}{Someone} %optional

%\makeindex %optional, also comment out \theindex at the end

\begin{document}

\maketitle %generates the titlepage, this is FIXED

\frontmatter %generates the frontmatter, this is FIXED

\begin{abstract}
This is a very abstract abstract. 

\lipsum
\end{abstract}

\begin{acknowledgements}
\lipsum 
\end{acknowledgements}

\tableofcontents 
\listoffigures %optional
\listoftables %optional

\mainmatter

%%%%
%%%% INTRODUCTION
%%%%

\chapter{Introduction}\label{chap:introduction}

% !TEX root = thesis-thomas-tiotto.tex

\section{Context}
\textbf{state the general topic and give background of what your reader needs to know to understand the problem outline the current situation evaluate the current situation (advantages/ disadvantages) and identify the gap}

\todo[inline]{In genere mancano referenze}

While neural networks and Artificial Intelligence (AI) - as a field - have existed for nearly seventy years, the concept of artificial intelligence dates back to at least Ancient Greece.  In ancient times, artificial intelligence embodied in mechanical men was part of the domain of myth; in the twentieth century, of that of science.  During this last decade, Artificial Intelligence can't anymore \todo{What do you mean? Why ``reductive''?} be described by a limited set of terms, as it has materialised out of Man's imagination, broken out of laboratories and has been given lease to act in the world at large.


No sector of our economy has been left untouched by the recent and rapid rise of machine learning that has been enabled by the rediscovery of deep neural networks, the availability of Big Data and cheap parallel computing power.  Fields as diverse and as critical as are government, healthcare, finance and bioinformatics have been revolutionised and the possibility has been set for new ones - such as self-driving vehicles - to be born.  
The ever increasing reliance of our society on ever more complex machine learning-driven algorithms can only make us worry ever more about the ethical dilemmas posed by such a situation.  
Our society has only very recently been confronted with the dilemma of assigning blame when a driverless car causes the death of a person but this moral problem is only the tip of the iceberg, even when focusing only on the automotive industry.  For example, how should a self-driving car behave when confronted with a real-world analogous of the classic Trolley Problem - a situation where each course of action is liable to cause harm?  
On what basis should a person be denied a mortgage, access to university or a job interview?  How can we be sure that there is no bias in the system?  How do we even define if the system is behaving morally?  Would it currently be feasible for a person that feels they have been harmed by such a decision to appeal it, as prescribed by the recent EU General Data Protection Regulation (GDPR)? 
As more and more decisions are made in an automated way, with many of them significantly impacting both individuals and society at large, it comes natural to stop and wonder what are the characteristics we would want the systems making these decisions to have.   

\textit{Explainable AI }(xAI) is the sub-field of AI that rests at the intersection between Computer Science, Social Sciences and Philosophy and whose aim is to define our desiderata of artificially intelligent systems and machine learning algorithms from the point of view of their explainability.  The basic idea is that the prerequisite for the evaluation of the ethical and moral implications of a machine's decision is for the system to be \enquote{interpretable} or \enquote{explainable}.  
Within the xAI community, \todo{references to articles ?} there is currently no unanimously agreed upon definition of which these desiderata should be or of the best way to implement them in real systems.
There is also no common, agreed-upon, definition of what is meant by the phrase \enquote{understanding a system}: some authors equate \todo{explain better what follows here, since it is important} it to having a \todo{what does functional mean?} \textit{functional understanding}, void of the \todo{what are low level details} low-level details, while others decline it into the  concepts of \textit{interpretation} and \textit{explanation}, the former indicating \todo{??? explain better. gives examples, help the reader} the output of a format that a human user can comprehend and the latter a set of features that have contributed to generating the system's decision. 

The difficulties start even in trying to define what interpretability really is.  \todo{does not sound well expressed here. also what has trust to do with interpretability? explain, give intuitions, motivations} Does it mean to gain the trust the system's user?  Of type of user in particular?  Does trust stem from some property of the decisions the system makes or from some other inherent characteristic of the machine?
A common approach to solving the difficulty in defining interpretability is to try and define it post-hoc by categorising systems into \todo{why are you speaking about ontologies here (and not simply of categories etc?} ontologies, based on their perceived interpretablility; unfortunately this seems like a circular way of approaching the problem: the classification of system models is being done utilising the same criterion that is trying to be uncovered by doing so.
For reference, a commonly used classification is the following: 
\begin{itemize}
	\item \textbf{Opaque systems}: these are systems that offer no insight into the mapping between \todo{in general. explain what a system, input and output are} inputs and outputs; all closed-source algorithms fall under this definition;
	\item \textbf{Interpretable systems}: this is the vastest category, as the characteristic of these systems is \textit{transparency} i.e. their inner workings are accessible but the onus of comprehensibility falls completely onto the user.  The classical example is that of neural networks where the mapping from inputs to outputs (the \textit{weights}) is inspectable by the user who can, theoretically and depending on her skill, interpret them;
	\item \textbf{Comprehensible systems}: systems falling into this category emit additional symbols together with their outputs with the explicit intent of giving the user the means to interpret and understand the automated decisions; the additional symbols may be visualisations, natural-language text or any other means of demystifying the output.  These extra symbols would need to be graded based on the user's expertise, as comprehension is a property that involves both man and machine but materialises on the human side.
\end{itemize}
\todo{References!!!} Some authors propose to classify systems as \textit{non-interpretable}, \textit{ante-hoc interpretable/transparent} and \textit{post-hoc interpretable}; this roughly corresponds to the ontology presented above.

What I hope can be gleamed from this brief introduction to the field of Explainable Artificial Intelligence, is that many of the problems it aims to tackle are hard \textit{per-se} and may not have a unique optimal solution.  This is because these issues are not only engineering problems, but exist at the intersection between man and machine and as such can't be tackled using only the methods of Computer Science.  There is no way to satisfactorily investigate the human element of the situation without resorting to the \todo{for instance? which methods? example?} well-established methods of the Social Sciences.  There is little hope to know in which direction to procede without the guiding force that can only come from philosophy, because of its millennia-long tradition in thinking about ethical and high-level issues.  
It should be clear that when the human - and particularly the ethical - domain are part of the equation, it is impossible \textit{by definition} to find an optimal and unique solution.

\todo[inline]{you have  to link what you just said with what you are gonna do later. for instance, illustrate what you are saying with examples. use one close to what you are going to do. also, you speak about NN but never about graphical methods, such as BN. you have to speak about them here too. }

% !TEX root = thesis-thomas-tiotto.tex

\section{Problem and Significance}
\textbf{identify the importance of the proposed research - how does it address the gap? state the research problem/ questions state the research aims and/or research objectives state the hypotheses
}

\todo[inline]{again, references!}

AI has a trust problem.  The bigger problem with AI is not anymore its utility, as that has mostly been solved by \todo{ok but come on, there is not only NN in this world} deep neural networks, but its capacity to elicit the trust of the users.
To be truly useful, an automated system should be able to make itself be trusted in a manner proportional to the criticality of its application.  Unfortunately, the explainability and, by extension, the \enquote{trustability} of machine learning models are inversely proportional.  There are many examples of modern methods - such as boosted trees, random forests, bagged trees, kernelized-SVMs - that show this tendency, but it is best exemplified by\textit{ deep neural networks} (DNN). Deep Neural Networks are machine learning models constructed by stacking many layers of artificial neurons, these systems are currently state of the art on a variety of tasks but are among the least easily interpretable systems due to the fact that they represent information in an implicit and distributed manner among their network weights.  Some older methods, like decision trees or rule-based methods, are inherently more interpretable due to their simplicity and the fact that they can explicit state their reasoning steps, but are less accurate and flexible than more modern techniques. 

The runaway success obtained by modern Machine Learning in a variety of domains, on a spectrum that goes from engineering to social work, has created the desire to also start applying these methods to mission-critical and traditionally more entrenched fields.  A perfect example of a field exhibiting both these characteristics is that of medicine.  The first successful artificially intelligent systems date back to the 1970s and '80s and were based on \textit{symbolic methods} integrated with \textit{knowledge-bases}.  These systems were by design capable of providing an explanation for their reasoning and were thus accepted by the medical community in an implementation known as \textit{expert systems} that aimed to perform functions similar to those of a human expert.  The deficiency of modern AI methods in being able to provide causal links for their reasoning process has held back their acceptance in the field of medicine, regardless of their superior performance and accuracy.

In a high-stakes domain such as the medical one, it would be unthinkable for a doctor to trust the predictions of an AI system a priori; any decision with profound moral implications - such as prescribing or interrupting the treatment of a patient - would have to first be validated by a human.  The possibility of carrying out this validation and its quality are dependent on the degree of interpretability of the model that made the decision.  Unfortunately, as has been repeated many times, the best performing models are often also the most opaque to inspection.

Explainability is not a necessary condition only for the verification of the system which, \todo{not clear. you were speaking about "validation" before. is this taken as a synonymous of "verification"? if yes, why two words for the same concept? if not, then explain better} as we have just discussed, is a presupposition for it to be applied in mission-critical domains, but also \todo{why?} for the extraction of knowledge from data.  The amount of information that a machine learning model can process is many orders of magnitude greater than that inspectable by any human; this may let a computer spot new patterns in the data that aren't immediately apparent or are latent given only a moderate amount of samples.  Being able to turn this information into \todo{so, what is knowledge? why explainability is necessary to create knowledge? you do not explain this here, imho} new knowledge implies the system having the ability to output human-interpretable symbols that are capable of communicating it in a comprehensible and effective way.

There has recently been much research carried out on trying to explain and extract knowledge from deep neural networks together with attempts to marry the connectionist and symbolic approaches to artificial intelligence - a subfield known as \textit{neuro-symbolic computation} while also reconsidering mixed approaches such as \textit{Bayesian Networks}.
A Bayesian Network is a graphical and computationally efficient way of representing dependencies between random variables.  The graphical component is immediate as in the model each random variable is represented by a node of a Directed Acyclic Graph (DAG), with the edges connecting them standing for their dependencies.  The efficiency stems from the fact that the graph structure imposes a factorisation of the joint probability space and thus lets each variable be calculated using only the values of its parents.

\todo[inline]{again, a lot of nice talk about NN, but then little about BN and thus what you are gonna do}
% !TEX root = thesis-thomas-tiotto.tex

\section{Response}
\textbf{outline the methodology used - 
outline the order of information in the thesis - a roadmap - Maximum 2500 words.}

The work carried out in this thesis concentrates on explainability in the medical domain and presents both a practical part, with the implementation of a Bayesian network-based system focused on \textit{knowledge-extraction}, as defined at the end of the previous section, and a theoretical one, regarding the definition and validation of desiderata for an artificially intelligent system using the aforementioned system.

The implemented system aims at supporting medical decision making through the instauration of a dialogue with the user/domain expert.  To this end, the information implicit in the data is used as basis for a constructive dialogue with the user; this starts with the expert informing the system of which knowledge is certain i.e. a variable's value that has been observed in a specific patient, and continues via a process where the next most probable \textit{(variable, state)} pair is proposed, with the expert having the choice of accepting it or refusing it, if she believes that the variable under examination doesn't adequately explain the accumulated evidence.  Each accepted variable is added to the evidence set, as the system gives priority to the domain expert's judgement.
The result of the dialogue is an \textit{explanation tree} whose nodes represent \textit{(variable, state)} pairs and are organised into branches, depending on the flow of the dialogue; more specifically, there will always be a \textit{main branch} corresponding to the choices of the user and none or more \textit{alternative branches} whose role is to inform the expert of the possible alternative outcomes to his decisions.

This software system was developed and tested tested in collaboration with \textit{Istituto Cantonale di Patologia}, a medical institute in Locarno, Ticino, Switzerland that specialises in the analysis of tissue samples received from hospitals, clinics and private doctors.  Its main activity is to characterise the samples by using \textit{histo-cytopathologic techniques}, with particular focus on the diagnosis of cancer and tumoural diseases in general.

The theoretical part of this thesis aims to understand how an  
 

%%%%
%%%% LITERATURE REVIEW
%%%%  


\chapter{Literature review}\label{chap:literaturereview}
What is explainability?
How is it defined?  By whom?  When?
Why is it important?
Notable works in the field

Suggestions  when  collecting  references:

Copy  everything  in  one  document  (  with  references  !),  but  do  not  use it  directly  when  writing  the  text 
 
Copy it in  the  document  (  with  references  !)  and  color it  ,  but  do  not  use it  directly  when  writing  the  text   

What  is  the    main  point  of  the  sentence  /  paragraph  /  article  ?  

What    do I  want  to  convey  ?  

Read a  few  references  /  paragraphs  and  only  then  write  it  down

\section{Explainability}

\section{Importance of Explainability}

\section{Explainability in Bayesian Networks}

\section{Notable Works}

\section{\enquote{Explaining the Most Probable Explanation}} \label{sec:explaining-the-most-probable-explanation}
The paper \enquote{Explaining the Most Probable Explanation} by \citet{Butz2018} places itself in the literature concerned with the explainability of Bayesian networks.
In particular, taking the classification proposed by \citet{lacave2002review} presented in Section \ref{sec:explainability-in-bayesian-networks}, it attempts to define a \textit{linguistic explanation} of the \textit{evidence} and of the \textit{reasoning}.
It differs from the previous attempt to define the explanation of the \textit{evidence} given by \citet{lacave2002review} and in other works, in that the paper is not concerned with finding the most probable assignment of variables that would explain the given evidence but, rather, the inverse problem.
By starting with the evidence and finding a maximally probable configuration, the authors hope \enquote{to look at the complete scenario to get an overview before deciding which variables should be focused on}; i.e., the goal appears to be to give the user an overview of the situation.  

The initial claim of the paper is that BNs are still difficult to interpret for domain experts, even though these models provide a graphical structure to the \textit{knowledge base}.
The examples brought to justify the claim are that edges in the graph do not necessarily represent causal dependencies and that d-separation (Definition \ref{def:d-separation}) may be confusing.
The authors plan to address this claim by constructing a \textit{dialogue} with the user and thus to continue in the long tradition of dialogical approaches to explaining BNs, many of which are presented in \citep{lacave2002review}.

The defining characteristic of their approach is that the domain expert is able to \enquote{argue} with the MPE and investigate alternative explanations.
The complete methodology, executed over three steps, is shown in Figure \ref{fig:butz-methodology}.
The first step is the construction of the \enquote{knowledge base}, which is nothing else than a probability tree representing a \enquote{chain of deduction} constructed following the strongest probabilistic dependencies between variables in the BN.
Such a \textit{knowledge base} is convenient because the document plan for the Natural Language Generation step is directly derived from it.
One issue that is immediately apparent is that this greedy approach does not \enquote{generate the MPE solution} as the authors claim.
This does not discredit the argumentative method as a whole, as \textit{it is not necessary for the user to be arguing the MPE to derive a good explanation}; this ties into one of the main findings in the previous sections that many xAI researchers are only focusing on one half of the explanation.
A good explanation is not given only by its formal properties but, most importantly, by how well it acts as an interface between the real \textit{user} and the model.
This is what \citet{abdul2018trends} mean when they lament that \enquote{despite their mathematical rigour, these works \textit{[referring to the existing explainability methods]} suffer from a lack of usability, practical interpretability and efficacy on real users}.

\begin{figure}[htbp]
\centerline{\includegraphics[width=\textwidth]{literature-review/images/butz-methodology}}
\caption{Overview of methodology followed by \citet{Butz2018}.}
\label{fig:butz-methodology}
\end{figure}

The document plan for the argumentation follows the same chain of strongest dependencies constructed in the \textit{knowledge base} until the expert disagrees; at that point, the user is presented with an alternative \enquote{MPE}.
An example of how the document plan may look after interaction with the user is shown in Figure \ref{fig:butz-tree}.
All the natural language phrasing is generated via boilerplates that take care of realising both the micro-planning phase and the generation of the text.

\begin{figure}[htbp]
\centerline{\includegraphics[width=\textwidth]{literature-review/images/butz-tree}}
\caption{\textit{Document plan} generated from the \textit{probability tree} \citep{Butz2018}.}
\label{fig:butz-tree}
\end{figure}

The authors recognise that such chains of deduction could become long and cognitively overloading in the case of larger BNs, as every variable in the tree is explained by all its ancestors.
A solution they propose is that of \textit{pruning} the probability tree by excluding d-separated nodes and those under a certain threshold of significance.
They also adapt some methods from literature to perform \textit{conflict analysis} i.e., only variables that contribute positively to the explanation are maintained in the document plan.

On the whole, \citet{Butz2018} offer a compelling explanation method for BNs by building on an established tradition of enabling explainability through dialogue.
The work, though, takes some methodological missteps and also continues the \enquote{sin} of not validating its claims on real users, which is one of the primary gaps in the xAI field, as identified in the previous sections.
\todo{controllare se qualcuno ha lavorato nell'estendere i metodi del paper}

\section{A Progressive Explanation of Inference in \enquote{Hybrid} Bayesian Networks for Supporting Clinical Decision Making}

%%%%
%%%% MATHEMATICAL BACKGROUND
%%%%
\chapter{Mathematical Background}\label{chap:mathematical-background}
Bayesian Networks (BN) are a class of Probabilistic Graphical Models that are used to represent systems under conditions of uncertainty.
To give a formal definition we will first need a few basic concepts from probability and graph theory.
\section{Probability Theory} \label{sec:probability-theory}
We will be dealing with \textit{standard probability} so random variables and probability measures will always be real-valued.
We will also, in the work carried out in this thesis, only be considering the case of random variables that can assume a finite number of possible values/states.
We will refer to these variables as \textit{categorical} to indicate that there is no natural ordering among their states.

\subsection{Random Variables} \label{subsec:random-variables}
\begin{definition}[Event]
	Given $\mathcal{S}$ the space of all possible outcomes of interest, an event $\sigma$ is a subset of $\mathcal{S}$: $\sigma \subseteq \mathcal{S}$.
	
	$\mathcal{F} \subseteq 2^{\mathcal{S}}$ is the set of all events that are under consideration.
	
	Two events $\sigma$ and $\tau$ are called disjoint when $\sigma \cap \tau = \emptyset$.
\end{definition}

\begin{definition}[Random Variable]
	A random variable $X$ is a function $X: \mathcal{S} \rightarrow \mathcal{X} \subseteq \mathbb{R}$ that associates every outcome $s \in \mathcal{S}$ with a value.
\end{definition}

\textit{Random variables} are a way of bringing to the fore the attributes of interest of events while dealing with them in a clean, mathematical way.
The values that a random variable can take are a function of the events in sample space $\mathcal{S}$, with each of these having a value assigned by the random variable function.

\begin{definition}[Probability Measure] \label{def:probability-measure}
	Given a sample space $\mathcal{S}$ and events $\mathcal{F}$, a discrete probability measure $\mathbb{P}$ is a function $\mathbb{P}: \mathcal{F} \rightarrow [0,1]$ that assigns a probability value to every event. 
	In the discrete case all subsets of $\mathcal{S}$ can be treated as events thus $\mathcal{F}$ is the power set of $\mathcal{S}$.
To be a valid probability measure, $\mathbb{P}$ must satisfy:
\begin{itemize}
	\item $\mathbb{P}(\mathcal{S}) = 1$;
	\item If events $\sigma$ and $\tau$ are disjoint then $\mathbb{P}(\sigma \cup \tau)=\mathbb{P}(\sigma)+\mathbb{P}(\tau)$.
\end{itemize}
\end{definition}
Each event $\sigma \in \mathcal{F}$ must have a probability $\mathbb{P}(\sigma) \in [0,1]$ and the sum of all these must equal $1$. 
An event with $\mathbb{P}(\sigma) = 0$ is deemed \textit{impossible} while one with $\mathbb{P}(\sigma) = 1$ is \textit{certain}.

\begin{definition}[Probability Mass Function]\label{def:mass-function}
	A probability mass function of a discrete random variable $X$ is a function $f_X: \mathcal{X} \rightarrow [0,1]$ defined, using a probability measure $\mathbb{P}$, as:
\begin{equation*}
	f_X(x) = \mathbb{P}(\{s \in \mathcal{S} : X(s)=x\}) \,,
\end{equation*}
and thus assigns a probability to every value $x \in \mathcal{X}$ in the domain of $X$.
\end{definition}
The probability mass function returns the probability of a random variable $X$ taking on exactly its value $x$.
This probability is the size of the subset of the event space $\mathcal{S}$ whose events $s$ are mapped to $x$ by the random variable function $X$.

Every random variable has a probability distribution induced by the cardinality of the subsets of its values; in the case of discrete one, such a distribution is \textit{multinomial}.

Often, in the context of random variables the probability distribution $f_X$ is called the \textit{marginal of $X$} and is usually contrasted with the notion of \textit{joint probability distribution}.
\begin{definition}[Joint Probability Mass Function]
	The joint probability mass function of discrete random variables $X$ and $Y$ is a function $f_{XY} : \mathcal{X} \times \mathcal{Y} \rightarrow [0,1]$ defined as:
	\begin{equation*}
		f_{XY}(x,y) = \mathbb{P}( \{s \in \mathcal{S} : X(s)=x \} \cap \{ s \in \mathcal{S} : Y(s)=y\} ) \,,
	\end{equation*}
	and thus assigns a probability to every tuple $(x,y)$ with $x \in \mathcal{X}$, $y \in \mathcal{Y}$.
\end{definition}

In what follows, we will sometimes refer to the marginal probability $f_X$ of $X$ as $\mathbb{P}(X)$, to $f_X(x)$ as $\mathbb{P}(X = x)$, to joint probability $f_{XY}$ of $X$ and $Y$ as $\mathbb{P}(X,Y)$, and to $f_{XY}(x,y)$ as $\mathbb{P}(X=x,Y=y)$ as the only random variables we will be dealing with will be discrete.
Notice that the notation $(E = e)$ is also often overloaded to signify an assignment of values to a set of random variables $E$; in this case what is meant is that every variable in the set $E = {X_1,...,X_k}$ assumes a certain value from its own domain. 
We will denote sets in bold so $\boldsymbol{E} = \boldsymbol{e}$ stands to mean that every variable $E$ in the set of random variables $\boldsymbol{E}$ assumes a value $e$ from its own domain. 
Finally, recall that the set of values that $X$ can take is denoted by the cursive $\mathcal{X}$.

\subsection{Probability interpretations} \label{subsec:probability-interpretations}
There are two main views through which to interpret the probability of an event: the \textit{frequentist} and the \textit{subjectivist/Bayesian} one.

The former views the probability of an event as the expected ratio of times it would occur over a great number of trials.
That is, the probability of an event is seen as the \textit{limiting frequency} of a repeatable event.
So, for example, the probability of observing heads when tossing a coin is said to be $0.5$ because over repeated throws heads was observed half the time.

The other view is the subjectivist or \textit{Bayesian} one (from the 18th century mathematician Thomas Bayes) in which probabilities are instead viewed as the \textit{subjective} degree of belief attributable to the manifestation of an event.
In this interpretation, stating that a coin has probability of heads of $0.5$ simply means that the person making the claim personally believes that the chances of seeing heads or tails are the same.
This is useful in that it enables the characterisation of certain events that haven't yet come about or that are liable to happen only once or a small number of times (that is, they are not repeatable).

Philosophically, Bayesian inference assigns a probability to a hypothesis (a \textit{prior}) while the frequentist method tests a raw hypothesis empirically before assigning it any probability.
As Bayesian inference naturally embraces and deals with uncertainty, it is an enormously useful tool to model and reason about the real, stochastic world we live in.

From the Bayesian point of view, we would consider the probability of a state of a random variable as simply representing the subjective degree of belief we would have over a set of outcomes we believed could manifest themselves.

\subsection{Conditional Probabilities} \label{subsec:conditional-probabilities}
\begin{definition}[Conditional Probability] \label{def:conditional-probability}
	The conditional probability mass function of random variable $Y$ given $X=x$, $x \in \mathcal{X}$, $y \in \mathcal{Y}$ is:
\begin{equation*}
\mathbb{P}(Y=y \mid X=x) = \frac{\mathbb{P}(X=x,Y=y)}{\mathbb{P}(X=x)} \,.
\end{equation*}
To be defined, it must be that $\mathbb{P}(X=x) > 0$.
\end{definition}

Definition \ref{def:conditional-probability} can easily be manipulated in order to obtain another basic result, called the \textit{chain rule of conditional probabilities}:
\begin{equation} \label{eq:chainrule}
	\mathbb{P}(Y=y,X=x) = \mathbb{P}(Y=y \mid X=x) \mathbb{P}(X=x) \,.
\end{equation}
Equation \ref{eq:chainrule} can be generalised to any number of variables:
\begin{align} \label{eq:chainrule-multiple}
\begin{split}
	\mathbb{P}(X_1=x_1, \ldots , X_n=x_n ) = & \mathbb{P}(X_n=x_n \mid X_1=x_1, \ldots, X_{n-1}=x_{n-1}) \times \\
	&  \ldots   \\
	 &\times \mathbb{P}(X_1=x_1 \mid X_2=x_2 ) \; \times \\
	 &\times \mathbb{P}(X_1=x_1) \,.
\end{split}
\end{align}
Intuitively, it means that we can decompose joint probabilities as products of conditional probabilities.  

Another immediate, and crucial, consequence of Definition \ref{def:conditional-probability} is known as \textit{Bayes' Theorem}, which lets us calculate the revised probability of an event given new knowledge regarding another event.
\begin{theorem}[Bayes' Theorem] \label{th:bayes-theorem}
	Given random variables $X$, $Y$ and the events $X=x$, $Y=y$, $\mathbb{P}(Y=y) > 0$, it holds that:
	\begin{equation*}
		\mathbb{P}(X=x \mid Y=y)=\frac{\mathbb{P}(Y=y \mid X=x) \mathbb{P}(X=x)}{\mathbb{P}(Y=y)} \,.
	\end{equation*}
\end{theorem}
Intuitively, this is a process of \textit{belief revision} as the belief in event $X=x$ is revised by the new knowledge that $Y=y$. 

\subsection{Independence} \label{subsec:independence}
\begin{definition}[Random Variables Independence]
	Two random variables $X$ and $Y$ with domains $\mathcal{X}$ and $\mathcal{Y}$ are independent when their joint probability mass $\mathbb{P}(X,Y)$ is equal to the product of their probability densities:
	\begin{equation*}
		\mathbb{P}(X=x,Y=y) =  \mathbb{P}(X=x) \times \mathbb{P}(Y=y) \quad \forall x \in \mathcal{X}, \forall y \in \mathcal{Y} \,.
	\end{equation*}
\end{definition}
In the real world it is hard or even impossible - if we consider Nature being based on chaos theory when viewed at a fine-enough level - to find two such perfectly non-interacting events.
Thus, a more useful concept is that of \textit{conditional independence} where two previously dependent events become independent when conditioned on a third one

\begin{definition}[Random Variables Conditional Independence]
	Two random variables $X$ and $Y$ with domains $\mathcal{X}$ and $\mathcal{Y}$ are conditionally independent on a third random variable $Z$ with domain $\mathcal{Z}$ when their probability densities conditioned on $Z$ are independent.
	That is, when the joint probability mass function conditioned on $Z$ is equal to the product of the conditional probability mass functions:
	\begin{equation*}
		\mathbb{P}(X=x,Y=y \mid Z=z) = \mathbb{P}(X=x \mid Z=z) \times \mathbb{P}(Y=y \mid Z=z) \quad \forall x \in \mathcal{X}, \forall y \in \mathcal{Y}, \forall z \in \mathcal{Z} \,.
	\end{equation*}
\end{definition}
Intuitively, this means that knowing any value of $Z$ makes the probability distributions of $X$ and $Y$ independent.
\section{Information Theory} \label{sec:information-theory}
The birth of the field of \textit{information theory} is usually traced back to the seminal paper \enquote{A Mathematical Theory of Communication} (\cite{Shannon1949}) where Claude Shannon set the mathematical basis for the quantification of the amount of \textit{information} transmissible over a noisy channel. 
In his words \enquote{The fundamental problem of communication is that of reproducing at one point, either exactly or approximately, a message selected at another point.}
The concepts of field are broad enough to have influenced practically every other scientific discipline and deep enough to have enabled the \enquote{digital age}, for example by enabling the creation of ever more complicated coding schemes for the compression, reconstruction and obfuscation of digital data.

\subsection{Entropy} \label{subsec:entropy}
In classical mechanical statistics, entropy can be seen as a measure of the uncertainty, or randomness, of a physical system.  
This concept was reapplied by Shannon to measure the amount of randomness in a random variable.
\begin{definition}
	Given a random variable $X$ with probability distribution $\mathbb{P}(X)$, its entropy $\Eta(X)$ is defined as the expected amount of information content carried by $X$ (\cite{Schneider2005}):
\begin{equation} \label{eq:entropy}
	\Eta(X) = \mathbb{E}(I(X)) = \mathbb{E}(-\log (\mathbb{P}(X)) = -\sum_{i=1}^{n} \mathrm{P}\left(x_{i}\right) \log _{b} \mathrm{P}\left(x_{i}\right)
\end{equation}
\end{definition}
The base $b$ of the logarithm defines the unit of measure.  Shannon used $b=2$ as he was dealing with the transmission of digital, binary-coded data; in this case the unit of measure are $bits$.

The simplest example of how information entropy characterises a random variable $X$, is in imagining $X$ to model a coin and the task being to predict the probability of the outcome of a throw being heads.
If the coin is fair, we will not be any more surprised to see the outcome being heads than tails; the entropy is maximum as there is maximum uncertainty regarding the outcome.
However, if the coin is not fair and tails is more probable the we will be more surprised than not to see the outcome being heads.  
The entropy is sub-maximal because there is less uncertainty regarding the outcome: tails is more probable than heads.
If one of the outcomes is impossible, for example if the coin has two heads, then the entropy of the coin is $0$ as there is no uncertainty regarding the result of a toss.


\subsection{Normalised Entropy} \label{subsec:normalised-entropy}
Plain entropy is not a good choice when trying to characterise random variables with different cardinalities of their sample space.
Let us suppose that the objective is to find the variable with the least \enquote{entropic} distribution and we suppose that their values have all been generated by the same process, say Gaussian.
Simply calculating their entropies and ordering them according to this criterion will bias the selection process towards the variables with smallest cardinality.
This is because we supposed them to be distributed in the same way so there will naturally be less uncertainty when there are fewer possible outcomes.
This can easily be understood by imagining the distributions to all be random uniform.

To obviate to this problem we need to \textit{normalise} the entropy so that different-sized variables can be directly compared to each other.
To achieve this, we can look at a measure of \textit{normalised entropy} or \textit{efficiency}:
\begin{equation} \label{eq:normalisedentropy}
 	\eta(X)=-\sum_{i=1}^{n} \frac{p\left(x_{i}\right) \log _{b}\left(p\left(x_{i}\right)\right)}{\log _{b}(n)}
\end{equation}
From Eq. \ref{eq:normalisedentropy} it can be seen that $\eta(X) \in [0,1]$; it is thus normalised and comparable among distributions.
This ratio expresses the amount of entropy found in the distribution compared to the maximum possible entropy when using $n$ symbols, corresponding to the uniform distribution:
\begin{equation}
\mathrm{H}\left(\underbrace{\frac{1}{n}, \ldots, \frac{1}{n}}_{n}\right) = - \sum_{i=1}^n \frac{1}{n} \log _{b} \left( \frac{1}{n} \right) = -n \cdot \frac{1}{n} \log _{b} \left( \frac{1}{n} \right) = - \log _{b} \left( \frac{1}{n} \right) = \log _{b}(n) 
\end{equation}   
\section{Graph Theory} \label{sec:graph-theory}
Many problems in Machine Learning (ML) do not involve classification or prediction of single data points in isolation, but of sets of entities that may present a more, or less, complex relation with each other. 
Most real-world phenomena fit into the latter framework.
Graphs are one of the most powerful tools for the modelling of this class of problems, as their structure naturally captures the wide variety of relations that may exist between entities.
These range from the atomic structure of a molecule to a social network of friends.  
In both these examples graphs help in reasoning, visualising and making inferences and predictions.

\subsection{Graphs} \label{subsec:graphs}
\begin{definition}[Directed Graph]
	A directed graph is a tuple 
	\begin{equation*}
		\mathcal{G} = (\mathcal{V}, \mathcal{E}) \,,
	\end{equation*}
with $\mathcal{V} = \{ v_1 \ldots v_n \}$ the set of vertices/nodes and $\mathcal{E}\subseteq \mathcal{V} \times \mathcal{V}$ the set of edges.
\end{definition}
We will not be needing the subclass known as \textit{undirected graphs} that are characterised by $\mathcal{E}$ being a \textit{set} of unordered pairs, that is of sets of the form $\{x,y\}$, with $x,y \in \mathcal{V}$.

The class of graphs presently of interest are those where there can be at most a single directed edge between any pair of nodes in $\mathcal{V}$; that is, we are not considering \textit{multigraphs}.
We are also interested in enforcing that there be no \textit{cycles} in the graph so there can be no subset of edges in $\mathcal{E}$ that when followed from vertex $v_i$ eventually ends up in $v_i$ again.
A cycle is a \textit{walk} - a sequence of edges which joins a sequence of vertices - of nodes of the form $v_i, v_j, \cdots, v_i$ i.e., a walk where only the first and last vertex are repeated.
Thus we have also automatically excluded the special case of cycle called \textit{self-loop}: an edge from a node to itself. 
The resulting graph possessing only directed edges and no cycles is commonly called a \textit{directed acyclic graph}, or DAG for short.  
\begin{definition}[Directed Acyclic Graph] \label{def:dag}
	A directed acyclic graph is a graph where every edge is directed and there are no cycles.
\end{definition}

In a DAG we may qualify nodes based on their \enquote{relationship status}:
\begin{description}
	\item[children] The children of node $u$ are all nodes $k$ for which there is a \textit{directed edge} from $u$ to $k$
	\item[parents] The parents of node $u$ are all nodes $k$ for which there is a \textit{directed edge} from $k$ to $u$
	\item[descendants] The descendants of node $u$ are all nodes $k$ for which there is a \textit{directed path} i.e., a walk where all vertices are distinct, from $u$ to $k$
	\item[ancestors] The ancestors of $u$ are all nodes for which there is a directed path from $k$ to $u$
\end{description}

An example of a DAG, containing five nodes, is shown in Figure \ref{fig:bn-example-dag}.

\begin{figure}[htbp]
\centerline{\includegraphics[width=0.5\textwidth]{mathematical-background/images/bn-example-structure}}
\caption{Example DAG representing a subset of the data set used in this thesis}
\label{fig:bn-example-dag}
\end{figure}

Polytrees and trees will also be defined because these are a fundamental concept for the work carried out in this thesis.
\begin{definition}[Tree] \label{def:tree}
	A tree is an undirected graph where there is one and only one walk between every node.	
\end{definition}
\begin{definition}[Polytree] \label{def:polytree}
	A polytree is a DAG whose underlying undirected graph is a tree.
	That is, if the directionality of edges is removed from the DAG, the resulting object is a tree. 
\end{definition}

\subsection{D-separation} \label{subsec:d-separation}
\textit{Dependence-separation} or \textit{d-separation}, as the name entails, is a concept relating to the conditional dependence between variables and was first presented by \citet{Pearl1988}.
We define the notation $u \rightarrow v$ to signify that there is a \textit{trail} between $u$ and $v$ in the graph, with a trail $(u, \ldots ,v)$ being a walk where all edges are distinct.

$u$ and $v$ and a node $z \in Z$ may be arranged in the graph in one of the following four configurations, called \textit{v-structures} in this context:
\begin{itemize}
  \item \textit{chain}: $u \rightarrow z \rightarrow v$
  \item \textit{chain}: $u \leftarrow z \leftarrow v$
  \item \textit{fork}: $u \leftarrow z \rightarrow v$
  \item \textit{collider}: $u \rightarrow z \leftarrow v$
\end{itemize}

These v-structures may be \textit{closed} by the set $Z$, this happens when:
\begin{itemize}
  \item \textit{chain}: $u \rightarrow z \rightarrow v$ and $z \in Z$
  \item \textit{chain}: $u \leftarrow z \leftarrow v$ and $z \in Z$
  \item \textit{fork}: $u \leftarrow z \rightarrow v$ and $z \in Z$
  \item \textit{collider}: $u \rightarrow z \leftarrow v$ and $z \notin Z$ and no descendant $z'$ of $z$ i.e., $z'$ such that $z \rightarrow z'$ exists, is also in the set $Z$
\end{itemize}

We say that $u$ and $v$ are \textit{d-separated} by $Z$ if every v-structure they appear in, is closed by $Z$.
Conversely, if there is at least one \textit{open} v-structure $u$ and $v$ are \textit{d-connected}.

If $u$ and $v$ are d-connected, then knowing something about $u$ also tells us something new about $v$, and viceversa.
An intuition for this can be given by interpreting the paths \textit{causally}.
In the case of a \textit{chain}, $z$ is the cause of $v$ so knowing $z$ tells us everything we need to know about the value of $v$ (or of $u$, if the chain is reversed).
In a \textit{fork}, conditioning on the \textit{common cause} $z$ has the same effect: $z$ is sufficient to know $u$ and $v$.
This is also called the \enquote{Common Cause Principle} \citep{sober1988principle}.

A good intuition for when $u$ and $v$ are \textit{d-separated} was given by \citet{Pearl1988}: imagine that there are two independent causes for a car refusing to start ($z$): having no gas ($u$) and having a dead battery ($v$): $u \rightarrow z \leftarrow  v$.
Only knowing that the battery is charged gives no information about the car having fuel or not.
But if we now know that the battery is charged after observing that the car won't start, we know for sure that it must be out of fuel.
So knowing something about $u$ is informative about $v$, after conditioning on $z$.

\begin{definition}[D-Separation] \label{def:d-separation}
	Given vertices $u$ and $v$ and a set of vertices $Z$, then $u$ and $v$ are d-separated by $Z$ if:
	\begin{itemize}
		\item $Z \neq \emptyset$ and $u$ and $v$ are never part of a collider;
		\item $Z = \emptyset$ and $u$ and $v$ are always part of a collider.
	\end{itemize}
\end{definition}

The independencies between variables are encoded in the structure of the DAG so every probability distribution modelled by a BN that has the same connections between nodes, also necessarily has the same independencies regardless of the values of the variables.

A series of examples using the DAG presented in Figure \ref{fig:bn-example-dag} are shown in Figures \ref{fig:bn-separations-example-1}, \ref{fig:bn-separations-example-2}, \ref{fig:bn-separations-example-3}.
We can see how the network's topology and the nodes chosen to be in the observed set $Z$, define the resulting separations.
In all cases $u=  \text{eta arrotondata} $ and $Y=V \mysetminus u \mysetminus Z$; we are asking for the set of all nodes in the DAG that are d-separated from $u$, given evidence $Z$.
This can easily be answered by enumerating all the v-structures in the network and applying Definition \ref{def:d-separation}.
In the case shown in Figure \ref{fig:bn-separations-example-1} we see that the node \textbf{eta arrotondata} is separated from nodes \textbf{recettori estrogeni}, \textbf{differenziazione} and \textbf{pN} given the observed evidence \textbf{mut17q21}.
The reason for this is because \textbf{$\text{eta arrotondata} \leftarrow \text{mut17q21} \rightarrow \text{recettori estrogeni}$} is a \textit{fork} and thus the flow of information from the rest of the network is blocked.
The way in which changing the conditioning set $Z$ also changes the independencies, can clearly be seen by comparing Figures \ref{fig:bn-separations-example-2} and \ref{fig:bn-separations-example-3}.

\begin{figure}[htbp]
\centerline{\includegraphics[width=0.5\textwidth]{mathematical-background/images/bn-example-separations-1}}
\caption{D-Separations in a subset of the provided data set (see Section \ref{sec:data-set})}
\label{fig:bn-separations-example-1}
\end{figure}

\begin{figure}[htbp]
\centerline{\includegraphics[width=0.5\textwidth]{mathematical-background/images/bn-example-separations-2}}
\caption{D-Separations in a subset of the provided data set (see Section \ref{sec:data-set})}
\label{fig:bn-separations-example-2}
\end{figure}

\begin{figure}[htbp]
\centerline{\includegraphics[width=0.5\textwidth]{mathematical-background/images/bn-example-separations-3}}
\caption{D-Separations in a subset of the provided data set (see Section \ref{sec:data-set})}
\label{fig:bn-separations-example-3}
\end{figure}

\section{Bayesian Networks} \label{sec:bayesiannetworks}
\subsection{Bayesian Networks Definition}
Given a variable $X$ and a set of variables $\boldsymbol{Y} = \{Y_1, ..., Y_n\}$, a \textit{conditional probability table} (CPT) is a table whose columns are in one-to-one correspondence with one of all the possible combinations of values of the variables in $\boldsymbol{Y}$. 
Each column is a probability mass function over $X$, say $P(X \mid Y_1=y_1, \ldots ,Y_n=y_n)$, conditional on the tuple $(Y_1=y_1, \ldots ,Y_n=y_n)$.
An example CPT can be seen in Table \ref{tab:b-cpd}.

\begin{definition}[Bayesian Network] \label{def:bayesian-network}
	A Bayesian Network (BN) is a probabilistic graphical model represented by a DAG $\mathcal{G}$ whose vertices are in one-to-one correspondence with a set of random variables $\mathcal{U}$ and the edges model the dependencies among these.
	The probability distribution of each variable $U \in \mathcal{U}$ is given by a CPT whose entries depend only on its parents in the graph structure.
	
	The so-called Markov Condition states that every variable $U$ is independent of all nodes in the network, except its descendent $Desc(U)$, given its parents $Pa(U)$:
	\begin{equation*}
		\forall U \in \mathcal{U}:  ( U \perp \neg Desc(U) \mid Pa(U)) \,.
	\end{equation*}
\end{definition}
The Markov Condition (named after the 19th century mathematician Andrej Markov) defines a d-separation on the DAG where each node/variable $U$ is d-separated from all others given the conditioning set $Pa(U) = \boldsymbol{Z}$.

A BN model is basically a way of representing an explicit joint distribution of random variables $\mathbb{P}(U_1=u_1, \ldots ,U_n=u_n) $ in a compact way.
The compactness is achieved by leveraging the Markov Condition, that is the independencies that exist among the random variables, and the Chain Rule (see Equation \ref{eq:chainrule}) to rewrite the joint as:
\begin{equation}
	\mathbb{P}(U_1=u_1, \ldots ,U_n=u_n) = \prod_i \mathbb{P}(U_i=u_i \mid Pa(U_i))
\end{equation} 
A BN gives the flexibility to drop the many weak dependencies that are bound to exist between variables thus leading to an even simpler model.
A full probability table for a joint distribution of random variables obscures the independencies and requires an exponential number of entries for the representation.
A Bayesian Network on the other hand can represent the same distribution using only a linear number of parameters.
The way that Bayesian Networks can be used to reduce the storage requirements for uncertain information is by taking advantage of the conditional independencies embedded in the underlying distribution being modelled.
The power of BNs comes from the additional information encoded in their structure and this was first explicitly described in its entirety by \citet{Pearl1988}, who defined the concept of dependence separation (see Subsection \ref{subsec:d-separation}) and applied it to Bayesian Networks.
A classic example of BN has been shown in Figure \ref{fig:asia-bn}.

One nice characteristic of BNs is that they very naturally model the type of mixed causal and stochastic processes that we find in all of Nature.
Imagine we want to represent the process modelled by joint distribution $\mathbb{P}(U,V)$; using the chain rule for conditional probabilities (Equation \ref{eq:chainrule}) we can write this as $\mathbb{P}(V \mid U) \mathbb{P}(U)$.
A BN modelling this process would be composed of two nodes $U$ and $V$ with an edge from the former to the latter $U \rightarrow V$, $U$ is called the \enquote{parent} of $V$.  Each of these two nodes would have its own probability table, with $\mathbb{P}(U)$ representing the \textit{prior} distribution over $V$ and $\mathbb{P}(V \mid U)$ the \textit{conditional probability distribution} of $V$ given $U$.

We can now see why these types of models are named \textit{Bayesian} Networks: the inference process is based in a given prior distribution/belief and evolves through a parent $\rightarrow$ child relationship to constantly yield an updated \textit{posterior} belief.
The BN DAG encodes a generative sampling where each variable's value is determined stochastically by Nature, based on the value of its parents.
This process is also highly compatible with our view of causality and this is one of the reason that makes BNs highly interpretable.
The prior $\mathbb{P}(A)$ can be seen as the result of some stochastic process caused by a series of latent (unmodelled) variables while the posterior $\mathbb{P}(B \mid A)$ is stochastically, causally determined by $A$. 
As mentioned in the previous paragraphs, there are probably no truly ``prior'' distributions in the Universe, at the modelling scale we are usually interested in.
Only on arriving on the quantum particle level may we find ``pure'' stochastic, uncaused processes due to quantum collapse.

A good example of how BNs are well compatible with our notion of causality may be to imagine $A$ as the random variable modelling the predisposition to having a certain disease and $B$ the one indicating actually developing the symptoms for it.
\textit{First}, genetic and epigenetic factors such as the environment stochastically contributed to having the predisposition and \textit{then} the development of the symptoms was stochastically determined by the degree of predisposition.
Adding an extra time dimension certainly helps in dealing with this class of models.

If the example show in Figure \ref{fig:bn-example-dag} is taken as the underlying graph structure of a Bayesian Network, each node now represents a random variable with an associated \textit{Conditional Probability Table} (CPT), that defines its probability distribution based on that of its parents.
The distributions for \enquote{eta arrotondata} and \enquote{mut17q21} in the Bayesian Network in question are shown in Table \ref{tab:mut-cpd} and \ref{tab:eta-cpd}.
\enquote{Mut17q21} is a root node i.e., has no parents, in the DAG so its probability distribution is unconditional or \textit{marginal}.
\enquote{Eta arrotondata}, on the other hand, is a child of \enquote{mut17q21} so the probability of its values is conditional on that of its parent and is represented by a CPT.
For example, \enquote{eta arrotondata} takes on value \enquote{<40} $44\%$ of the time when \enquote{mut17q21} has value \enquote{mut}, but only $4\%$ of the time when \enquote{mut17q21} has value \enquote{unknown}.

\begin{table*}[htbp]
\centering
\caption{mut17q21 mass function}
\begin{tabularx}{0.5\textwidth}{ccX}
\toprule
 \multirow{2}{*}{\textbf{mut17q21}} & mut & 0.01  \\
 & unknown & 0.99 \\
\bottomrule
\end{tabularx}
\label{tab:mut-cpd}
\end{table*}

\begin{table*}[htbp]
\centering
\caption{eta arrotondata CPT}
\begin{tabularx}{0.5\textwidth}{ccXX}
\toprule
      & &  \multicolumn{2}{c}{\textbf{mut17q21}} \\
\cmidrule(lr){3-4}
 & & mut & unknown    \\ 
 \multirow{3}{*}{\textbf{eta arr.}}  & <40 & 0.42 & 0.04  \\
 & 40-50 & 0.42 & 0.17    \\
 & >50 & 0.15 & 0.78 \\
\bottomrule
\end{tabularx}
\label{tab:eta-cpd}
\end{table*}

Probabilistic graphical models such as Bayesian networks - as just defined - are often used to express expert knowledge about a particular domain and perform reasoning on that problem. 
Alternatively, the specification of the network can be automatically achieved from a sufficient amount of data about the variables under consideration for a particular reasoning task. 
In this thesis we focus on the case of Bayesian network learned from data, but the methods presented in Chapter \ref{chap:methodology} would apply also to a user-designed network, as would be the case in an Expert System. 
As the Bayesian Network formalism consists of both a qualitative element (the directed graph) and a quantitative one (the conditional probability tables), in the following sections we will detail how these two components can be obtained automatically from data.

\subsection{Learning Bayesian Networks Structure from Data} \label{subsec:learning-bn-structure} 
Learning a BN DAG from data is typically addressed as an optimisation task and is known as the \textit{Bayesian Network Structure Learning problem}. 
In many probabilistic models initialisation is fast but then fitting the data is slow (for example in \textit{k-means}).
For Bayesian Networks the converse is true: fitting is fast as only sums of the counts in the data are needed, but learning the correct graph structure can take super-exponential time - more precisely, the space of Bayesian Networks that have $|V|$ variables has size $2^{O(|V|^2)}$ \citep{berzan2012exploration} - in the number of variables and thus easily becomes an intractable problem.

Let us consider the specification of a BN over the variables $\boldsymbol{X}=(X_1,\ldots,X_n)$ and denote as $\boldsymbol{D}$ a data set of joint and complete observations of $\boldsymbol{X}$. 
A \textit{score function} is a map $f$ giving a score to any possible DAG $\mathcal{G}$ whose nodes are in correspondence with $\boldsymbol{X}$ as a function of the dataset $\boldsymbol{D}$. 
The resulting score $f(\mathcal{G},\boldsymbol{D})$ is a measure of how well a BN with graph $\mathcal{G}$ fits the dataset $\boldsymbol{D}$.
The simplest approach consists in using the likelihood (i.e., the probability assigned by the BN to the data) as a score. 
Yet, to prevent over-fitting additional terms penalised complex models are added.
Given the score, the problem is basically a search over the set $\Gamma$ of all the possible DAGs with $|V|$ nodes i.e., $\mathcal{G}^* = \underset{\mathcal{G} \in \Gamma}{\arg\max} f(\mathcal{G},\boldsymbol{D})$. 
Such a task is NP-hard but approximate search procedures to solve it efficiently can be defined.

The methods to solve this problem can be roughly categorised into three categories:
\begin{description}
	\item[Search and Score] This is the most na{\"i}ve method as it does a brute force search over all the possibile graph structure space - i.e., all DAGs with the same number of variables as the input data - and scores all these depending on some cost function.
	
	There are many cost functions that have been proposed over the years; for example, a Bayesian cost function represents the probability of the DAG $G$ given the data $\boldsymbol{D}$: $\mathbb{P}(G \mid \boldsymbol{D})$, an Information Theory one scores the fitness of a DAG by its ability to balance graph description length and data description length given the graph. 
	
		This process is super-exponential in the number of variables but through the use of dynamic programming and heuristic search algorithms it can become sub-exponential.
		Nonetheless, solving the exact problem is only feasible up to $\approx 30$ variables.
	\item[Constraint Learning] Methods of this type calculate some measure of correlation to identify the presence and direction of edges between nodes and are much less used than the other ones presented.
		A typical test is to iterate over all triplets while testing for conditional independencies.
		Thanks to the d-separation properties outlined in Subsection \ref{sec:bayesiannetworks}, this test is able to identify the correct edges.
		The algorithm is quadratic in time and in the number of vertices.
	\item[Approximations] Several heuristical approaches have been developed to be able to find good network structures in an efficient manner.
		Examples of these are:
		\begin{itemize}
		  \item Chow-Liu, that builds a tree approximation of the probability distribution.
		  \item Greedy Hill-Climbing, that adds/removes/flips an edge at a time.
		  \item Optimal Reinsertion, that iteratively calculates the optimal \textit{Markov blanket} (the subset of all nodes that are sufficient to determine the value of another subset) of an ever-smaller subset of nodes.
		\end{itemize}
\end{description}

\subsection{Learning Bayesian Networks Parameters from Data} \label{subsec:learning-bn-parameters}
Once the DAG structure is given, learning the CPTs from the data $\boldsymbol{D}$ can be accomplished by one of two approaches:
\begin{description}
	\item[Frequentist] The frequentist approach treats the parameters $\theta$ to be learned as unknown but fixed and attempts to find a $\theta^*$ that maximises the likelihood function $\mathbb{P}(\boldsymbol{D} \mid \theta)$.
		Given symbol $j$ in the CPT of variable $i$ conditioned on the parents having value $k$ and $N_{i j k}$ the count of times that this combination of symbols appears in the data $\boldsymbol{D}$ then the Maximum Likelihood Estimator $\hat{\theta}_{i j k}^{MLE}$ for the entry $j,k$ in the CPT of $i$ is given by:
	\begin{equation}
		\hat{\theta}_{i j k}^{MLE}=\frac{N_{i j k}}{\sum\limits_{j^{\prime}} N_{i j^{\prime} k}}
	\end{equation}
	\item[Bayesian] The Bayesian method instead treats $\theta$ as a random variable with a prior probability $\mathbb{P}(\theta \mid \alpha)$, with $\alpha$ virtual pseudo-count, and uses Bayes' Rule (see Theorem \ref{th:bayes-theorem}) and a likelihood $\mathbb{P}(\boldsymbol{D} \mid \theta)$ to calculate the posterior probability $\mathbb{P}( \theta \mid \boldsymbol{D}, \alpha)$.
		Given symbol $j$ in the CPT of variable $i$ conditioned on the parents having value $k$ and $N_{i j k}$ the count of times that this combination of symbols appears in the data $\boldsymbol{D}$ then the Maximum a Posteriori Estimate $\hat{\theta}_{i j k}^{MAP}$ for the entry $j,k$ in the CPT of $i$ is given by:
		\begin{equation}
			\hat{\theta}_{i j k}^{MAP}=\frac{N_{i j k}+\alpha_{i j k}}{\sum\limits_{j^{\prime}}\left(N_{i j^{\prime} k}+\alpha_{i j^{\prime} k}\right)}
		\end{equation}	
\end{description}

\subsection{Bayesian Networks Updating} \label{subsec:bnupdating}
All the types of inference presented are instances of \textit{diagnostic reasoning}, also known as \textit{abductive reasoning}. 
Abductive reasoning is a form of inference that starts from observed evidences and looks for the best/most simple \textit{explanation} for them.
Unlike \textit{deductive reasoning}, abductive reasoning is not based on logical necessity, but on inferences based on observations; thus it can not verify the conclusion with absolute certainty but only yield, at best, a highly probable explanation.
The direction of inference is reversed in abduction - this is why it is sometimes called \textit{retroduction} - as compared to deduction.
It would be a logical fallacy know as \enquote{affirming the consequent} to state that any explanation were certain, because there may be multiple possible explanations for the same observation.

The following examples may help to clarify the difference between the two inference processes:
\begin{description}
	\item[deductive reasoning] given \enquote{Every man is mortal}$(a_1)$ and \enquote{Diogenes is a man}$(a_2)$ it necessarily follows that \enquote{Diogenes is mortal}$(b)$: $(a_1) \wedge (a_2) \implies (b) $.
	\item[abductive reasoning] given that \enquote{Diogenes is mortal}$(b)$ it is very probable that \enquote{Diogenes is a man}$(a_2)$; this is not certain as it may also be that \enquote{Diogenes is a dog}$(a_3)$: $(b) \centernot\implies (a_2) \wedge (b) \centernot\implies (a_3) $.
\end{description}

Abductive reasoning can either be modelled as a conditional probability or a MAP query and is of fundamental importance in many important problems of machine learning including medical diagnosis, that is of particular interest in this thesis.
Let us define three sets of variables of interest in the BN: $\boldsymbol{U_q}$ the \textit{query variables}, $\boldsymbol{U_e}$ the \textit{evidence variables} and $\boldsymbol{U_m}$ all other variables.

We can thus define the conditional probability query as the updated probability of $\boldsymbol{U_q}$ based on the observation of the values of $\boldsymbol{U_e}$.

\begin{definition}[Conditional Probability Query]
	The conditional probability query $\mathbb{P}(\boldsymbol{U_q} \mid \boldsymbol{U_e}=\boldsymbol{e})$ for variables $\boldsymbol{U_q}$ given evidence $\boldsymbol{U_e}=\boldsymbol{e}$ is:
\begin{equation*} 
	\mathbb{P}(\boldsymbol{U_q} \mid \boldsymbol{U_e}=\boldsymbol{e}) = \frac{\mathbb{P}(\boldsymbol{U_q},\boldsymbol{U_e}=\boldsymbol{e})}{\mathbb{P}(\boldsymbol{U_e}=\boldsymbol{e})} = \frac{ \sum\limits_{\boldsymbol{U_m}} \mathbb{P}(\boldsymbol{U_q},\boldsymbol{U_e}=\boldsymbol{e},\boldsymbol{U_m}) }{ \sum\limits_{\boldsymbol{U_m, U_q}} \mathbb{P}(\boldsymbol{U_q},\boldsymbol{U_e}=\boldsymbol{e},\boldsymbol{U_m}) } = 
	\frac{ \sum\limits_{\boldsymbol{U_m}} \prod_i \mathbb{P}(U_i \mid Pa(U_i)) }{ \sum\limits_{\boldsymbol{U_m, U_q}} \prod_i \mathbb{P}(U_i \mid Pa(U_i)) } \,.
\end{equation*}
\end{definition}

Another common type of question we might ask a BN is the following: \enquote{given evidence $\boldsymbol{U_e}$ which is the most likely assignment of a subset of variables $\boldsymbol{U_q}$?}.
This is know as \textit{Maximum a posteriori (MAP)} inference and is a much harder problem that a conditional probability query.
The solution is given by solving an optimisation problem.
\begin{definition}[Maximum a Posteriori Query]  \label{def:map}
Given sets $\boldsymbol{U_q} \subseteq \mathcal{U}$ and $\boldsymbol{U_z} = \mathcal{U} \mysetminus \boldsymbol{U_e} \mysetminus \boldsymbol{U_q}$, the MAP query for $\boldsymbol{U_q}$, $\text{MAP }( \boldsymbol{U_q} \mid \boldsymbol{U_e}=\boldsymbol{e} )$, is:
	\begin{equation*}
		\text{MAP }( \boldsymbol{U_q} \mid \boldsymbol{U_e}=\boldsymbol{e} ) = \underset{\boldsymbol{q}}{\arg\max} \sum\limits_{\boldsymbol{z}} \mathbb{P}(\boldsymbol{U_q}=\boldsymbol{q}, \boldsymbol{U_z}=\boldsymbol{z} \mid \boldsymbol{U_e}=\boldsymbol{e}) \,.
	\end{equation*}
\end{definition}
The solution will be the assignment of values $\boldsymbol{z}$ to all variables in the set $\boldsymbol{U_z}$ that maximises their joint probability.

An important thing to note is that the greedy assignment where each variable picks its most likely value can be very different from the most likely joint assignment of all variables.
A simple example showing this is given by \citet[pag. 26]{koller2007}.
Suppose a Bayesian Network is composed of two nodes $U$ and $V$ with the former the parent of the latter: $U \rightarrow V$.
Assume their CPDs are represented by the CPTs shown in Tables \ref{tab:a-cpd} and \ref{tab:b-cpd}.
The most probable value for $U$ is $U=u_1$ and this constrains $V$ to choose equivalently between $V=v_0$ or $V=v_1$.
The probability of the assignment $(U=u_1,V=v_1)$ given by $( \underset{u}{\arg\max} \mathbb{P}(U=u), \underset{v}{\arg\max} \mathbb{P}(V=v)) )$ is $0.6 \times 0.5 = 0.3$.
On the other hand, the most likely joint assignment given by $\underset{u,v}{\arg\max} \mathbb{P}(U=u)\mathbb{P}(V=v)$ is $(U=u_0,V=v_1)$ and has probability $0.4 \times 0.9 = 0.36$.

\begin{table*}[htbp]
\centering
\caption{U CPT}
\begin{tabularx}{\textwidth/4}{ccX}
\toprule
 \multirow{2}{*}{\textbf{A}} & $u_0$ & 0.4  \\
 & $u_1$ & 0.6 \\
\bottomrule
\end{tabularx}
\label{tab:a-cpd}
\end{table*}

\begin{table*}[htbp]
\centering
\caption{V CPT}
\begin{tabularx}{\textwidth/3}{ccXX}
\toprule
       &  \multicolumn{3}{c}{\textbf{A}} \\
\cmidrule(lr){3-4}
 & & $u_0$ & $u_1$   \\ 
 \multirow{2}{*}{\textbf{B}}  & $v_0$ & 0.1 & 0.5  \\
 & $v_1$ & 0.9 & 0.5    \\
\bottomrule
\end{tabularx}
\label{tab:b-cpd}
\end{table*}

The MAP problem is hard to solve efficiently because it is part of the \textit{NP-hard} complexity class, as proved by \citet{Shimony1994}.
Calculating it in a brute-force way would mean elencating all the possible variable-value tuples and computing their joint probabilities; as these are exponential in the number of variables, the problem is evidently untractable.

This is true even in a Bayesian Network.  
Such a model may possess a linear number of parameters but the underlying distribution is still implicitly exponential in size.
Explicitly calculating the MAP defeats the very purpose of the BN, that is computational efficiency.
For this reason, there exist a host of approaches to optimising MAP: elimination algorithms, gradient methods, simulated annealing and other stochastic local searches, belief propagation and integer linear programming.

A special case of MAP is the \textit{Most probable explanation (MPE)} and is an easier problem to solve.
\begin{definition}[Most Probable Explanation Query]  \label{def:mpe}
 Given set $ \boldsymbol{U_q}= \mathcal{U} \mysetminus \boldsymbol{U_e}$, the MPE query for $\boldsymbol{U_q}$, $\text{MPE }( \boldsymbol{U_q}\mid \boldsymbol{U_e}=\boldsymbol{e} )$, is:
	\begin{equation*} 
		\text{MPE }( \boldsymbol{U_q} \mid \boldsymbol{U_e}=\boldsymbol{e} ) = \underset{\boldsymbol{q}}{\arg\max}\, \mathbb{P}(\boldsymbol{U_q}=\boldsymbol{q} , \boldsymbol{U_e}=\boldsymbol{e}) \,.
	\end{equation*}
\end{definition}
The solution will be the assignment of values $\boldsymbol{u_q}$ to all variables in the set $\boldsymbol{U_q}$ that maximises their joint probability.

An intuition for why the MPE is easier to solve can be given by comparing Definition \ref{def:map} with Definition \ref{def:mpe}; unlike MPE, MAP presents both a summation and a maximisation and as such is part conditional probability query, part MPE query.
All algorithms for the computation of MAP obviously apply to MPE too, but there exist efficient approximate algorithms for MPE that do not generalise to MAP such as Loopy Belief Propagation \citep{Pearl1988} and Stochastic Local Search \citep{Kask1999}.
\section{Summary}

%%%%
%%%% METHODOLOGY
%%%%

\chapter{Methodology}\label{chap:methodology}
 The inspiration for the work carried out in this thesis was the paper \enquote{Explaining the Most Probable Explanation} by \cite{Butz2018}, that has been presented in detail in Sec. \ref{sec:explaining-mpe}.
 This paper proposed a system that would build a Bayesian Network modelling a medical data set and, through the interaction with a medical expert, distill an explanation tree.
 This tree, deemed to represent the solution to the MPE query, could then be used to generate a natural language explanation that the authors claim would lead to the extraction of extra knowledge from the original data set.  

 The driving hypothesis of the paper was that Bayesian Networks and the solution to the MPE problem would be a powerful tool in helping medical experts gain insights into data.
 Unfortunately, the paper did not provide any indication that a such a system had ever been built and any validation of the method was left for future work.
 As of the finalisation of this thesis (\today), there has been no work done in substantiating the conclusions by \cite{Butz2018}. \todo{controllare altri autori}
 As discussed in Chap. \ref{chap:introduction} and Chap. \ref{chap:literaturereview} there is an ever greater need for Machine Learning models and systems to be explainable, especially in mission-critical domains as is healthcare.
 Current machine learning systems are for the most part opaque and there is confusion regarding even what would constitute a good explanation of their working.
 
 For these reason, I believe that building a proof-of-concept system whose logic was inspired by the method presented in the aforementioned paper and validating it with real medical experts would be an important step forwards in the direction of answering the following questions:
 Can the method of the paper be corroborated?
 Are Bayesian Networks a good ML model to bootstrap an explanation from?
 How good an explanation does the proposed method give, as validated by a domain expert?
 What improvements are there to be made?
 \todo{migliorare dopo aver fatto literature review perche avro piu idee}


% !TEX root = thesis-thomas-tiotto.tex

\section{Data set} \label{sec:data-set}
As anticipated in Chap. \ref{chap:introduction} the work carried out in this thesis had a certain degree of collaboration with a third party, \cite{istitutocantonalepresentazione}.

\subsection{Istituto Cantonale di Patologia}
Istituto Cantonale di Patologia (ICP) is an institute based in Locarno that is specialised in the histological analysis of tissue samples received from private patients, clinics and hospitals, mainly in support of cancer diagnosis.
Its Laboratory of Medical Diagnostics supports pathologists in the diagnosis of neoplastic diseases through the application of cytogenetic techniques; that is, the focus is on understanding how chromosomes relate to cell behaviour, particularly during mitosis and meiosis.
One of the techniques used is Fluorescence in Situ Hybridization (FISH) that is able to localise the presence or absence of specific DNA sequences in chromosomes.
These tests are aimed at identifying the precise profile of the cancer cells and thus inform the clinician on the best treatment for the specific patient.

In addition to its clinical support activities, the Istituto also carries out scientific research aimed at better understanding certain types of cancers at a basic level.
In the last ten years, the ICP has published more than 200 peer-reviewed papers and more than 100 works in non-peer reviewed journals and is active at a national and international level.

\subsection{Motivation}
My first contact with the ICP was during a meeting with Dr. Vittoria Martin (\cite{martin2012}), molecular citogenetist, in date 28/01/2019.
The institute had expressed interest in bringing machine learning into their workflow in order to both augment their profiling capabilities for patients and to be able to extract new knowledge from their existing data.
This knowledge-extraction may lead towards the confirmation of current scientific theories or may be the first step towards the formulation of novel ones.

\todo[inline]{chiedere a Vittoria cosa si aspettavano/aspettano}

My interest in collaborating with the Istituto stemmed from the desire to apply the methods described in Sec. \ref{sec:novel-contributions} to a real-world case.
Being the theoretical work being carried out in this thesis an expert-driven MPE approximation, collaboration with the institute has also provided the opportunity to implement a proof of concept using real histological data.
The doctors and researchers of the Istituto have been able to validate, from an Explainable AI and clinical relevance points of view, the model software that I have developed.
That is to say, they have validated the capacity of the developed software both in its capacity to support clinical decision making and to surface clarifying explanations of the data set and in its adherence to established medical literature.

\todo[inline]{da confermare con Vittoria piu avanti}

\subsection{Provided Data Set}
The data set I was provided with was created by \textit{Registro Tumori Ticino} (Locarno, Ticino) in order to highlight possible new relations between clinical, histopathological and molecular features, as well as to potentially discover novel biomarkers for the progression of the disease 
It consists of the histological records over 38 recorded variables of 3218 breast cancer patients who have been diagnosed between the years 2005 and 2014 within the Ticino canton of Switzerland.
The data set had been pre-processed by collaborators of IDSIA in agreement with the ICT with 13 of the variables being dropped, because not considered relevant.
In particular, all variables relating to the patient post-treatment were discarded as well as those recording the diagnosis date.
In Tab. \ref{tab:datasetvariables} is a description of the measured variables, together with their clinical meaning.
The value distribution of the data set is shown in Tab. \ref{tab:datasetdistribution}.

The indications from Dr. Martin on how to further preprocess the data are shown in Tab. \ref{tab:datasetpreprocess}.
Note that some variable names were simplified.

\begin{table*}[htbp]
\caption{Data set variables}
\begin{tabularx}{\textwidth}{@{} l Y @{}}
\toprule 
\textbf{Variable} & Clinical meaning \\
\midrule 
\textbf{Codice globale} & Unique patient identifier \\
\textbf{mut17q21} & Mutation of chromosome 17 \\
\textbf{loss 17} & Loss of chromosome 17 \\
\textbf{et\`a arrotondata} & The age of the patient at diagnosis \\
\textbf{Lateralit\`a} & The affected breast \\
\textbf{Situ SUBGROUP MZ} & The primary site code of the tumour \\
\textbf{Morfologia SUBGROUP MZ} & The morphology classification of the tumour \\
\textbf{pT SUBGROUP MZ} & Primary tumour in the TNM classification for breast cancer \\
\textbf{pN SUBGROUP MZ} & Pathologic in the TNM classification for breast cancer \\
\textbf{M 8.2.96} & Distant metastasis in the TNM classification for breast cancer \\
\textbf{Differenziazione} & Tumour grade \\
\textbf{Recettori estrogeni percento 1.1.2003} & Expression of estrogen receptors \\
\textbf{Recettori progestinici percento 1.1.2003} & Expression of progestin receptors \\
\textbf{c erbB 2  cod percento 1.1.2003} & ErbB2 marker expression \\
\textbf{Ki67 cod percento} & Tumoural proliferation index \\
\textbf{FISHRatio} & FISH analysis result \\
\bottomrule
\end{tabularx}
\label{tab:datasetvariables}
\end{table*}

\begin{table*}[htbp]
\caption{Data set distribution before pre-processing}
\begin{tabularx}{\textwidth}{@{} l X c @{}}
\toprule 
\textbf{Variable} & Unique values & Distribution \\
\midrule 
\textbf{mut17q21} & 2 & \includegraphics[width=0.2\textwidth, height=10mm]{methodology/images/mut17q21}  \\
\textbf{loss 17} & 3 & \includegraphics[width=0.2\textwidth, height=10mm]{methodology/images/loss_17}\\
\textbf{eta arrotondata} & 74 & \includegraphics[width=0.2\textwidth, height=10mm]{methodology/images/eta_arrotondata}\\
\textbf{ateralit\`a} & 3 & \includegraphics[width=0.2\textwidth, height=10mm]{methodology/images/lateralita} \\
\textbf{situ} & 5 & \includegraphics[width=0.2\textwidth, height=10mm]{methodology/images/situ} \\
\textbf{morfologia} & 5 & \includegraphics[width=0.2\textwidth, height=10mm]{methodology/images/morfologia} \\
\textbf{pT} & 23 & \includegraphics[width=0.2\textwidth, height=10mm]{methodology/images/pt} \\
\textbf{pN} & 6 & \includegraphics[width=0.2\textwidth, height=10mm]{methodology/images/pn} \\
\textbf{M} & 3 & \includegraphics[width=0.2\textwidth, height=10mm]{methodology/images/m} \\
\textbf{differenziazione} & 5 & \includegraphics[width=0.2\textwidth, height=10mm]{methodology/images/differenziazione}  \\
\textbf{recettori estrogeni} & 40 & \includegraphics[width=0.2\textwidth, height=10mm]{methodology/images/recettori_estrogeni} \\
\textbf{recettori progestinici} & 40 & \includegraphics[width=0.2\textwidth, height=10mm]{methodology/images/recettori_progestinici}\\
\textbf{c erbB 2} & 4 & \includegraphics[width=0.2\textwidth, height=10mm]{methodology/images/c_erb_2}\\
\textbf{Ki67} & 52 & \includegraphics[width=0.2\textwidth, height=10mm]{methodology/images/ki67}\\
\textbf{FISH} & 5 & \includegraphics[width=0.2\textwidth, height=10mm]{methodology/images/fish}\\
\bottomrule
\end{tabularx}
\label{tab:datasetdistribution}
\end{table*}

\begin{table*}[htbp]
\caption{Data set preprocessing steps}
\begin{tabularx}{\textwidth}{@{} l Y @{}}
\toprule 
\textbf{Variable} & Action \\
\midrule 
\textbf{Codice globale} & Remove variable \\
\textbf{mut17q21} & Remove blanks \\
\textbf{loss 17} & Remove blanks \\
\textbf{eta arrotondata} & Bin into \enquote{$< 40$}, \enquote{$40-50$}, \enquote{$\geq 50$} \\
\textbf{lateralita} & Remove blanks and \enquote{sconosciuta} \\
\textbf{situ} & Remove blanks \\ \addlinespace
\textbf{morfologia} & Remove blanks and \enquote{unuseful} if performance on classification is subpar \\ \addlinespace
\textbf{pT} & Remove blanks and \enquote{unuseful}  \\
\textbf{pN} & Remove blanks and bin into \enquote{0} and \enquote{$\neq0$}\\
\textbf{M} & Remove blanks \\ 
\textbf{differenziazione} & Remove blanks and \enquote{Sconosciuto o non applicabile} \\ \addlinespace
\textbf{recettori estrogeni} & Remove blanks and bin into \enquote{negativo} if $\leq 10$,
		\enquote{debolmente positivo} if $\leq 50$, 
		\enquote{fortemente positivo} if $> 50$ \\ \addlinespace
\textbf{recettori progestinici} & Remove blanks and bin into \enquote{negativo} if $\leq 10$, 
		\enquote{debolmente positivo} if $\leq 50$, 
		\enquote{fortemente positivo} if $> 50$ \\ \addlinespace
\textbf{c erbB 2} & Remove blanks \\ 
\textbf{ki67} & Remove blanks and bin into \enquote{<14}, 
		\enquote{14-20}, \enquote{20-30}, \enquote{>30} \\ 
\textbf{FISH} & Remove blanks \\
\bottomrule
\end{tabularx}
\label{tab:datasetpreprocess}
\end{table*}


% !TEX root = thesis-thomas-tiotto.tex

\section{Methods} \label{sec:methods}

\subsection{Libraries}\label{subsec:libraries}
The system developed in this thesis was coded in Python and as such made use of an array of standard and less-know packages.
The most significant for the development of the system are Pomegranate and Pgmpy, that are both packages implementing probabilistic graphical models.

\subsubsection{Probabilistic Graphical Models Packages}
\textit{Pomegranate}\footnote{\url{https://Pomegranate.readthedocs.io/en/latest/}} is an open-source probabilistic models package for Python.
Its core philosophy is that every probabilistic model, from Hidden Markov to Bayesian Network, can be seen as a probability distribution and, as such, can be flexibly composed into hierarchical mixture models \citep{Schreiber2017}.
The package implements Bayesian Network as well as many other probabilistic models but currently only supports Discrete Bayesian Networks, so the random variable of each node must have a categorical distribution.
This is not an issue as the provided data set (see Section \ref{sec:data-set}) was already composed of only categorical variables.
Also, working with discrete entities should make explainability easier as the number of possible variable values at hand can be reduced at will; this should in turn reduce the cognitive load requested from the user.

Pomegranate was chosen among others for its good implementation of Bayesian Networks and its performance.
The package is written in Cython and natively supports multi-core parallelism and out-of-core learning.
Network \textit{structure learning from data} is claimed to be particularly efficient, thanks to a novel c\textit{onstraint learning} (see Subsection \ref{subsec:learning-bn-structure}) method that implements prior knowledge into the graph selection process \citep{schreiber_noble_2017}.
The claim made by the paper is that this innovative graph selection process should possess the speed of a heuristic approach, while still yielding a far better quality estimate of the correct graph structure.

\textit{Structure learning} from data is achieved using the \texttt{from\_samples} method of the \\ \texttt{BayesianNetwork} class, with the default algorithm being the novel one described by \citet{schreiber_noble_2017}.
The \textit{probability} of a sample is calculated using the \texttt{probability} function; the \texttt{predict\_proba} function is used to return the probability of each variable in the model given some evidence.
\textit{Predictions} (described in detail in Subsection \ref{subsec:bnupdating}) are run by passing an object a matrix with \texttt{None} as placeholders for missing values to the \texttt{predict} function.
\textit{Fitting} is done thought the \texttt{fit} function that uses MLE estimates to update each node's distribution in the model based on the input data.

A \texttt{BayesianNetwork} object can also be displayed graphically by calling its \texttt{plot} function.
The output is a \texttt{DOT} file that is generated using the PyGraphviz package\footnote{\url{https://pygraphviz.github.io}}, a Python interface to the famous Graphviz\footnote{\url{https://www.graphviz.org}} graph visualisation software.
An example of such an output is shown in Figure \ref{fig:Pomegranate_graph_example}.

\begin{figure}[htbp]
\centerline{\includegraphics[width=0.8\textwidth]{methodology/images/Pomegranate_example}}
\caption{Example output of \texttt{plot} \citep{Pomegranatetutorial}}
\label{fig:Pomegranate_graph_example}
\end{figure}

\textit{Pgmpy}\footnote{\url{http://pgmpy.org}} is, like Pomegranate, another recent probabilistic graphical model package for Python.
Unlike Pomegranate, it natively implements various exact and approximate inference algorithms (see Subsection \ref{subsec:bnupdating}), like variable elimination, belief propagation and max-product linear programming.

The reason that two different probabilistic graphical model libraries were used, is because there is currently no Python package that offers all the needed functionality.
Pomegranate implements a novel structure learning algorithm but is severely lacking in functionality in many other areas.
Pgmpy, on the other hand, has a very good API as regards inference.

\subsubsection{External Solvers}
\textit{DAOOPT}\footnote{\url{https://github.com/lotten/daoopt}} is an open-source implementation of the sequential AND/OR Branch-and-Bound algorithm proposed by \citet{Marinescu2006}.
Search-based algorithms traverse the model's space and are much more efficient in their use of memory, compared to inference-based algorithms such as variable elimination.

DAOOPT builds an AND/OR search space to generate an AND/OR graph that takes advantage of information encoded in the graphical model, namely its independencies.
The DAOOPT implementation found at \citet{daoopt}, is an exact solver for finding an MPE solution in Bayesian Networks.
The software is written in C++ and accessible through a command-line interface; the only required parameter is a \texttt{.uai} file representing a Markov Random Field or a Bayesian Network but in most cases an optional \texttt{.uai.evid} file will also be given, containing the observed evidences.

The \texttt{.uai} file format is a simple text file used to represent problem instances.
Such a file is composed of:
\begin{itemize}
  \item \textit{Preamble}: containing the type of the network (MARKOV or BAYES), the number and cardinality of variables and the cliques, that in the BAYES case are simply the variables appearing in each Conditional Probability Table (CPT).
  \item \textit{Function tables}: containing the actual definition of the CPTs i.e. the values of each node give its parents or, in the case of root nodes, the marginal probabilities.
\end{itemize}

The \texttt{.uai.evid} is a very simple file containing the number of variables in the evidence set followed by the index of each variable and its observed value.
In both formats the variables and their values are represented only by a numerical index, starting from $0$, with the ordering being defined in the preamble of the \texttt{.uai} and maintained consistent throughout the \texttt{.uai} and \texttt{.uai.evid}.

Following, is the \texttt{.uai} representing the network shown in Figure \ref{fig:bn-example-dag}, that has been the running example throughout the last chapters.
Lines starting with \texttt{c} are interpreted as comments; these are misinterpreted by DAOOPT and are thus removed when running it, but are here shown for clarity and because they are part of the official \texttt{.uai} format.
The file starts by stating that the model is a Bayesian Network composed of 5 random variables; these will then be referenced by an ordinal index starting at 0.
The first variable (index 0) is of cardinality 3, the second (index 1) is of cardinality 2 and so on.
We can then see the definition of the cliques or more precisely, as the model is BAYES  and not MARKOV, of the CPTs; there are 5 of these, each one associated to one of the five variables just stated.
The first CPT involves 2 random variables: the first (0) and the second (1); the second CPT involves only one variable (1) and this tells us that variable 1 is a root node in the BN's DAG.
The ordering is such that the child node is the last in the definition of each CPT's nodes so, for example, in the first CPT we find that variable 1 is the child of variable 0.
Finally, we have a complete definition of the function tables/CPTs.
The tables are printed so that each row corresponds to the conditional probability value of the child node and increasing rows correspond to increasing enumeration of the parents' states, in the order given when defining the variables involved in the CPTs.
The first table corresponds to \textbf{eta arrotondata}'s CPT shown in Table \ref{tab:eta-cpd}.
It contains 6 elements as it involves variables 0 (\textbf{mut17q21}) and 1 (\textbf{eta arrotondata)} that are of cardinality 2 and 3, respectively.
So, each row corresponds to the probability distribution of the three states of variable 1, given each of the two states of variable 0.

\begin{minipage}{\linewidth}
\begin{framed}
\begin{verbatim}
c
c Bayesian Network exported from Pomegranate - Thomas Tiotto (2019)
c

BAYES
5
3 2 3 3 2 

c
c Cliques
c

5
2 0 1 
1 1 
2 2 1 
3 3 2 4 
1 4 

c
c CPTs
c

6
 0.42105263157894735 0.42105263157894735 0.15789473684210523 
 0.043798177995795384 0.17063770147161877 0.7855641205325858 

2
 0.006613296206056387 0.9933867037939436 

6
 0.6842105263157895 0.0 0.3157894736842105 
 0.1373510861948143 0.021723896285914507 0.8409250175192712 

18
 0.004385964912280701 0.2412280701754386 0.7543859649122807 
 0.022598870056497175 0.11864406779661016 0.8587570621468926 
 0.10344827586206899 0.41379310344827586 0.4827586206896552 
 0.2121212121212121 0.45454545454545453 0.3333333333333333 
 0.14094488188976378 0.6362204724409449 0.22283464566929131 
 0.289612676056338 0.5677816901408451 0.1426056338028169 

2
 0.5315001740341107 0.4684998259658893 
\end{verbatim}
\end{framed}
\end{minipage}

The following is an example of a randomly generated \texttt{.uai.evid} evidence file that simply states that the evidence set has cardinality 2 and contains variable 4 (in the ordering given in the \texttt{.uai}) in its state 1 and variable 3 in state 2.

\begin{framed}
\begin{verbatim}
 2
  4 1
  3 2
\end{verbatim}
\end{framed}

Both the \texttt{.uai} and the \texttt{.uai.evid} were generated by the custom functions presented in Subsection \ref{subsec:algorithms} under the \textbf{MPE} header.
These are able to export a \texttt{Pomegranate} model and randomly generated evidence to the correct input format for DAOOPT.

\subsubsection{Standard Packages}
\textit{pandas}\footnote{\url{https://pandas.pydata.org/about.html}} is an extremely widely-used open-source Python library that provides data structures and methods to support data analysis.
The package excels in the manipulation of tabular data in the form of \texttt{DataFrame}, that is the analogous of R's \texttt{data.frame}.
A \texttt{DataFrame} can be seen as a \enquote{general 2D, size-mutable structure with potentially heterogeneously-typed columns}.
The syntax for slicing is very close to R's, as are many other functionalities; this is because one of Pandas' explicit goals was to offer all of CRAN's functionalities and to be easily approachable by anyone already knowing the other language.

Pandas was the default choice for this thesis' implementation because it is the \textit{de facto} standard in data analysis applications when using Python.
Its flexibility in reading Excel spreadsheets (the format of the provided data set, see Section \ref{sec:data-set}) and in then manipulating the data confirmed that this was a good choice.
Note that the additional \texttt{xlrd} package is needed to read files in the Excel format.

\textit{Scikit-learn}\footnote{\url{http://scikit-learn.github.io/stable}} aims at providing a unified API for basic machine learning; it does not include advanced paradigms such as Reinforcement Learning or graphical models for structured learning.
The latter omission was the reason that lead to select Pomegranate as the basis for the implementation of the system.

What is included are a stack supervised and unsupervised ML tools to prepare data sets, define machine learning models ranging from spectral analysis-based to ensemble methods to clustering and multiple evaluation and model selection utilities.

\textit{NumPy}\footnote{\url{http://numpy.org}} is another \textit{de facto} standard package when doing scientific computing with Python.
Most scientific packages (including Pandas, Scikit-learn and TensorFlow) depend on NumPy for low-level operations; this is because NumPy provides a fast implementation of n-dimensional array objects together with powerful manipulation functions.
In addition to this, NumPy implements linear algebra operations, Fourier Transform and random number generation.

The closest parallel to NumPy - as R was for Pandas, is MATLAB.

\textit{NetworkX}\footnote{\url{https://networkx.github.io}} is another widely-used package; it is specialised in the creation and manipulation of graph-structured data.
The main use for this package was in building the \enquote{knowledge base} structure that  the dialogue with the expert is based on.

\subsection{Algorithms} \label{subsec:algorithms}
This subsection is concerned with presenting algorithms and methods that were adapted and used for this thesis, but that were not part of the original work.

\subsubsection{Model Construction}
The data was given in \texttt{.xlsx} format and was imported using Panda's \texttt{read\_excel} function that returned a \texttt{DataFrame} object.
The imported data was then preprocessed by dropping unwanted records and binning the remaining ones following the instructions outlined in Table \ref{tab:datasetpreprocess}.
The actual BN representation is learned at runtime by calling the \texttt{from\_samples} method of Pomegranate's \texttt{BayesianNetwork} to solve the structure learning problem (defined in Subsection \ref{subsec:learning-bn-structure}).

The binned data was codified into integer representations before being passed to Pomegranate's structure learning algorithm.
Thus the network's state names are in natural language but the internal representation of the values of each random variable is an integer number.
A dictionary object is used to translate one representation into the other when interacting with the user.

\subsubsection{D-separation}
A na{\"i}ve implementation to check for d-separation between nodes $X$ and $Y$, according to Definition \ref{def:d-separation}, would have a complexity in the order of the number of trails between $X$ and $Y$; this would lead to an exponential, in the size of the graph's vertices set, running time.
Luckily, \citet{koller2007} present a linear time algorithm to solve the problem, whose pseudocode is shown in Algorithm \ref{alg:koller-d-separation}.

\begin{algorithm}[htp!]
	\caption{reachable procedure by \citet{koller2007}}
	\label{alg:koller-d-separation}
	\begin{algorithmic}[1]
		\State $\mathcal{G}$ BN graph
		\State $X$ source variable
		\State $\boldsymbol{Z}$ observations
		\State
		\State $\boldsymbol{L}= \boldsymbol{Z}$ \Comment{Phase 1}
		\State $\boldsymbol{A} = \emptyset$
		\While{$L \neq \emptyset$}
			\State Select some $Y$ from $\boldsymbol{L}$
			\State $\boldsymbol{L}=\boldsymbol{L} \mysetminus \{Y\}$
			\If{$Y \notin \boldsymbol{A}$}
				\State $\boldsymbol{L} = \boldsymbol{L} \cup Pa(Y)$
			\EndIf
			\State $\boldsymbol{A}=\boldsymbol{A} \cup \{Y\}$
		\EndWhile
		\State
		\State $\boldsymbol{A} = \{(X, \uparrow)\} $ \Comment{Phase 2}
		\State $\boldsymbol{V} = \emptyset$
		\State $R = \emptyset$
		\While{$\boldsymbol{L} \neq \emptyset$}
			\State Select some $(Y,d)$ from $\boldsymbol{L}$
			\State $\boldsymbol{L} = \boldsymbol{L} \mysetminus \{(Y,d)\}$
			\If{$(Y,d) \notin \boldsymbol{V}$}
				\If{$Y \notin \boldsymbol{Z}$}
					\State $R = R \cup \{Y\}$
				\EndIf
				\State $\boldsymbol{V} = \boldsymbol{V} \cup \{(Y,d)\}$
				\If{$d=\uparrow$ and $y \notin \boldsymbol{Z}$}
					\For{each $Z \in Pa(Y)$}
						\State $\boldsymbol{L} = \boldsymbol{L} \cup \{(Z,\uparrow)\}$
					\EndFor
					\For{each $Z \in Ch(Y)$}
						\State $\boldsymbol{L} = \boldsymbol{L} \cup \{(Z,\downarrow)\}$
					\EndFor
				\ElsIf{$d=\downarrow$}
					\If{$Y \notin \boldsymbol{Z}$}
						\For{each $Z \in Ch(Y)$}
							\State $\boldsymbol{L} = \boldsymbol{L} \cup \{(Z,\downarrow)\}$
						\EndFor
					\EndIf
					\If{$Y \in \boldsymbol{A}$}
						\For{each $Z \in Pa(Y)$}
							\State $\boldsymbol{L} = \boldsymbol{L} \cup \{(Z,\uparrow)\}$
						\EndFor
					\EndIf
				\EndIf
			\EndIf
		\EndWhile
		\State \textbf{return} R
	\end{algorithmic}
\end{algorithm}

The \texttt{reachable} procedure, as defined in the book, takes as input the DAG representing the Bayesian Network $\mathcal{G}$, a source variable $X$ and a set of observed variables $\boldsymbol{Z}$; on exit it returns the set of variables $R$ that are reachable from $X$.
The procedure runs in two phases, traversing the graph twice: first bottom-up from leaves to roots, then vice-versa.
During the first stage, the algorithm finds all nodes $\boldsymbol{A}$ that are ancestors of the evidence set $\boldsymbol{Z}$.
During the second phase, the procedure distinguishes the direction it visits each node in order to determine if it is traversable or not.
Any node $Y$ that is not in the evidence set is marked as reachable; if it is being visited in direction \enquote{up} ($(Y,\uparrow)$) it can be traversed as the v-structure (see Subsection \ref{subsec:d-separation} for a primer) is a \textit{chain}.
All the parents of $Y$ are marked to be visited in the \enquote{up} direction (i.e. from below) and the converse is done for $Y$'s children.
If $Y$ is being visited in the \enquote{down} ($(Y,\downarrow)$) direction its children are again added to be visited in the \enquote{down} direction, because $Y$ is traversable.
Additionally, if $Y$ happened to be in the set $\boldsymbol{A}$, found in the first step, then $Y$'s parents are marked to be visited in the \enquote{up} direction because the \textit{collider} is active and $Y$ can be traversed (a collider is open if and only if the central node or any of its descendants are observed).

The implementation in this thesis follows the pseudocode of the book very closely but the procedure \texttt{d-separated}, instead of finding all nodes $R$ that are d-connected to the input $X$, only tests if a given target $Y$ is d-separated from $X$ or not, as is shown in Algorithm \ref{alg:d-separation}.
This gives some extra flexibility in how the function can be used.
To find the set $S$ of all nodes d-separated from $X$, the \texttt{d-separated} is iterated to test over all nodes $V$ in the BN.

\begin{algorithm}[htp!]
	\caption{d-separation algorithm}
	\label{alg:d-separation}
	\begin{algorithmic}[1]
		\State $separated\_list = \emptyset$
		\For{target $Y \in V$} 
			\State append $d-separated(X, Y, E)$ to $separated\_list$ \Comment{will return true or false}
		\EndFor
	\end{algorithmic}
\end{algorithm}

\subsubsection{MPE}
The solution to the Most Probable Explanation problem (MPE) (Definition \ref{def:mpe}) is planned to be found by using DAOOPT (described in Subsection \ref{subsec:libraries} under the \textbf{DAOOPT} header) as an external solver.
The latest version of DAOOPT was downloaded from the official repository \citep{daoopt} and compiled into an executable.
DAOOPT only offers a command line interface so some extra work is needed in order to integrate it with the Python-based application under development.
The connection is provided by first writing to stable storage a \texttt{Pomegranate.uai} containing the model definition and a \texttt{Pomegranate.uai.uai.evid} with the chosen evidence.
These files are then fed to DAOOPT by using Python's \texttt{subprocess} module, by running the following command in a background shell:
\begin{verbatim}
	./daoopt -f Pomegranate.uai -e Pomegranate.uai.evid
\end{verbatim}
The shell output is captured and also written to stable storage, in order for the solution to be parsed from it.

To exemplify the process, we return to the example used while presenting DAOOPT in its Subsection in \ref{subsec:libraries}.
Given the \texttt{.uai} representing the BN and the \texttt{.uai.evid} random evidence:
\begin{framed}
\begin{verbatim}
 2
  4 1
  3 2
\end{verbatim}
\end{framed}

DAOOPT would give the following output:

\begin{minipage}{\textwidth}
	\begin{framed}
\begin{verbatim}
	--- Starting search ---
[0] u 3 4 -1.3581 5 2 1 2 2 1
[0] Cache statistics: . . . .

--------- Search done ---------
Problem name:  Pomegranate
OR nodes:      3
AND nodes:     4
OR processed:  3
AND processed: 8
Leaf nodes:    2
Pruned nodes:  4
Deadend nodes: 1
Time elapsed:  0 seconds
Preprocessing: 0 seconds
-------------------------------
-1.3581 (0.0438433)

p 2 1 2
l 2 1 6
s -1.3581 5 2 1 2 2 1
\end{verbatim}	
\end{framed}
\end{minipage}
The end of the final line is the one of interest, as it is the assignment of values to the variables that solves the MPE problem.
The \texttt{5 2 1 2 2 1} string is to be interpreted as meaning:
\begin{itemize}
  \item there are 5 variables in the solution
  \item the variable indexed by 0 (in the ordering given in the preable of the \texttt{.uai}) is assigned its second value (the ordering is inferred by the CPTs defined in the \texttt{.uai}) in the MPE solution
  \item variable 1 is assigned its second value
  \item variable 2 is assigned its third value
  \item variable 3 is assigned its third value
  \item variable 4 is assigned its second value
\end{itemize}
Variables 3 and 4 are constrained to assume the value specified in the input \texttt{.uai.evid}; in this case 1 and 2, respectively.

All the functionality relating to solving the MPE with DAOOPT is encapsulated in the \texttt{daoopt\_solver} function that given the input \texttt{.uai} files, returns the MPE solution.

%\subsubsection{Other Machine Learning Methods}
%In order to have a benchmark for the classification performance of the Bayesian Network, a series of tests were implemented with the aim of finding the best performing algorithm on the data set.
%Given that the performance of Machine Learning algorithms is heavily dependent on the input classes, a process of \textit{exhaustive variable elimination} was used, in order to identify the most relevant features for the predictions.
%Each input subset was scored using the following ML algorithms, in order to find the best performing one:
%\begin{itemize}
%  \item \textit{linear regression}: this method assumes that the relationship between the dependent variable $y$ and the regressors $x$ is linear i.e., that $y$ can be written as a linear combination of $x$'s components: $y = \beta_0 + \beta_1 x_1 + \ldots \beta_n x_n$.
%  \item \textit{logistic regression}: is used in lieu of Linear Regression when the values of the variables are categorical; it assumes that the relationship between the regressors $x$ and the log-odds of $y$ are linear i.e. $\log _{b} \frac{p}{1-p}=\beta_{0}+\beta_{1} x_{1}+ \ldots + \beta_{n} x_{n}$ with $p$ the probability of the event of interest.
%  \item \textit{linear discriminant analysis}: LDA is related to Principal Component Analysis (PCA) in that it attempts to find a linear equation modelling the data but LDA explicitly tries to express the difference between the data classes.
%  \item \textit{decision tree}: a decision tree is built using a recursive, greedy algorithm that continually splits the dataset into two.  The variable along which to bisect is the one that yields the lowest accuracy loss in the resulting split.
%  \item \textit{na{\"i}ve Bayes}: a na{\"i}ve bayes classifier is a conditional probability model that given features $x_1 \ldots x_n$, attempts to assign a probability to each of the possible outcomes $O_{k}$ of interest by using Bayes' Theorem (Definition \ref{th:bayes-theorem}): $\mathbb{P}\left(O_{k} | x \right)=\frac{\mathbb{P}\left(O_{k}\right) \mathbb{P}\left(x| O_{k}\right)}{\mathbb{P}(x)}$.  The method is called \enquote{na{\"i}ve} because of the strong (ofter unrealistic) assumption that all the features $x_1 \ldots x_n$ are independent.
%  \item \textit{k-nearest neighbours}: the algorithm is non-parametric with the output class depending on the predominant class among the $k$ nearest neighbours (according to some distance metric) of the input vector $x$.
%  \item \textit{support vector}: a support vector machine (SVM) is an algorithm that attempts to find the set of \textit{best-separating hyperplanes} between classes of objects, seen as points in a high-dimensional space.  Such a hyperplane is the one that has maximum distance from the closest representatives of each class.
%  \item \textit{random forest}: this is an ensemble method that aims to correct decision trees' tendency to overfit the data.  A multitude of decision trees is constructed and the final classification output is the class that appears most often in the intermediate step.
%  \item \textit{AdaBoost}: short for \textit{adaptive boosting}; this meta-learning, ensemble algorithm combines a series of \textit{weak classifiers}, that may only be slight better than a random guess, through a weighted sum into a \textit{strong classifier}.  It is a meta-learning algorithm because the weak classifiers are revised over a series of iterations in order to improve their performance on previously misclassified instances.
%  \end{itemize}




\section{Novel contributions}\label{sec:novel-contributions}
 \todo{tengo qui o sposto nel cap 4?}
\subsection{Theory}

\subsection{Algorithms}
\subsubsection{\enquote{pseudo-MPE}}
\subsubsection{alternative explanation branches}
\subsubsection{\enquote{pseudo-MPE} from random evidence}
\section{Validation Methodology} \label{sec:validation}
Having direct access to expert pathologists has not only helped in guiding research into the theoretical explainability properties of the system but also enabled their \textit{application-grounded evaluation} (see Section \ref{sec:evaluation-of-explainability}).
There are two main validation points of view to be addressed: the clinical (Subsection \ref{subsec:clinical-validation-methodology}) and the explainability (Subsection \ref{subsec:explainability-validation}), with the results of the latter depending on part on those of the former.
 
\subsection{Clinical Validation} \label{subsec:clinical-validation-methodology}
A validation of the methods carried out in this thesis in their adherence to established clinical literature is of paramount importance.
A failure on the Bayesian network's part in capturing the true relationships between the variables would hamper it in being able to give any meaningful representation of them.
For the experts to even start to trust the system or to be able to make sense of its outputs, it is vital that there be as little cognitive dissonance between their basic beliefs and expectations and those that they see represented in the system.

For this reason, the initial validation phase with the ICP concentrated on the clinical aspect.
The methodology chosen to clinically validate the system was for the ICP to formulate a series of natural language queries; each one of these questions was annotated with the queried variable and its value, together with the values of any evidence variables.
The experts included the expected reply to the queries together with its likelihood, based on the latest medical literature and their personal, knowledge-based expertise.
These questions can be abstracted as:
\begin{center}
\enquote{Given that the value of $var_1$ is $a_1$ and $\ldots$ and the value of $var_n$ is $a_n$, what is the probability that $var_{n+1}$ takes value $a_{n+1}$?}.	
\end{center}

The natural language questions formulated by the ICP can be classified along two axes:
\begin{itemize}
  \item based on their intended purpose: \textit{validation} vs. \textit{research}.
  The former questions' replies are known from established clinical literature and are the queries that will actually be used to validate the system from a clinical point of view.
  The latter are queries that don't have a definite clinical answer but that are nonetheless extremely interesting in helping to understand the types of questions a domain expert may want to ask the system.
  \item based on the way they may be answered: by a \textit{conditional probability query} (Definition \ref{def:conditional-probability}), a \textit{d-separation query} (Definition \ref{def:d-separation}) or an \textit{MPE query} (Definition \ref{def:mpe}).
\end{itemize}
The complete series of thirty questions has been organised according to the second criterion.
Appendixes \ref{app:conditionalprobability1} and \ref{app:conditionalprobability2} present fourteen questions that can be answered by conditional probability queries.
Appendix \ref{app:dseparation} shows a series of eight natural language questions that can be answered by running a d-separation query.
Appendix \ref{app:conditionalanddseparation} presents five questions that could be answered by a conditional probability query but also, at a higher level, by a d-separation query.
This is because what is being asked, is basically wether changing the value of the evidence variable has an influence on that of the target variable.
This could be answered by running multiple conditional probability queries and comparing the resulting target variable values or, more simply, by checking if the target and evidence variables are d-connected or not.
The first method would give a finer grained answer as it would also \textit{quantify} the magnitude of the effect of one variable on the other; checking for d-separation would only give a \textit{qualitative} answer, which may nonetheless be sufficient. 
Finally, Appendix \ref{app:mpe} shows three questions that are naturally mapped onto a query of the MPE type.

Most importantly at this stage, all questions can be implemented on the proof of concept system and consequently this shows a good coverage on the tool's part of the use cases that can be imagined by a domain expert.
If the system can, in principle, answer every question imagined by the expert then this is an indication that it conforms to her \textit{worldview} and thus could be well positioned to interact fruitfully with her.

The questions marked as \textit{validation} will be posed to the system, in autonomy, by the ICP's representatives, who will then compare the outputs with the result they would have expected, based on established medical literature and their expertise.
The columns containing the experts' expected results and their comments have been omitted from the natural language questions shown in Appendix \ref{app:natural-language-questions} and included directly in the discussion of the results in Subsection \ref{subsec:clinical-validation-results}, alongside the system's outputs.
If the system's outputs conform to the experts' preconceived ideas in a high number of cases (as confirmed by the experts themselves) then the system can be said to have been \textit{clinically validated}.
This is important because the enabling condition for the user to trust the predictions made by the software is that these shouldn't be in strong discordance with her existing beliefs.
Not having a strong \textit{cognitive dissonance} is a \textit{necessary} - but not sufficient - condition to enable trust and therefore explainability.

\subsection{Explainability Validation} \label{subsec:explainability-validation}
In general, there is strong resistance to novelty in the field of medicine, both for ethical reasons and because of the need for clinicians to be conservative in attending to established best practices in the field.
Any tool that is too onerous in terms of time and cognitive load is liable to remain underutilised.
In this field, \textit{a tool must therefore only be the means by which a question is answered}, not itself become a question; the methods developed in this thesis aim to conform to this objective, barring the experimental nature of the software and the consequent lack of refinement of its interface.
The need for a comprehensible and efficient tool is especially present because the goal of a pathologist is to arrive at a diagnosis, containing the elements useful to define prognosis and therapeutical approach, in the briefest time possible.
The main reasons are ethical, since for a patient waiting for a report is extenuating, and clinical, because a timely diagnosis is the first factor at the base of life expectancy.
Obviously, the highest possible accuracy is always strived for.
The clinical field and that of biomedicine are forced to embrace uncertainty, as this is an integral part of their practice.
Consequently, any tool able to support in comprehension and decision-making is automatically useful, once it has been clinically validated; in other words, even though a specific system may not be decisive or applicable to all reviewed cases, it will nonetheless be taken into account.

Thus, a system validated in terms of its adherence to clinical literature could then also meaningfully be validated from an explainability point of view.
The main question to be addressed is its capacity to relate to the expert user.
Is the system able to engender the user's trust?
In doing so, is she able to extract more knowledge from existing data when using the system than not?
Especially in cases where there may be a dearth of data, can the expert maximise the benefit from the available information?
Does the user subjectively feel that the system may positively impact her work?
These are all hard questions to answer, as there is a very high degree of subjectivity involved.
Thus to attempt to answer them, the chosen methods were borrowed from the social sciences.

In an earlier stage, the experts were introduced to the system in prototype form and instructed on the use cases it offered.
This process would enable the collection of feedback on the functionalities of the system and help in shaping its subsequent design.

The finalised system was, in a later phase (early August 2019), provided to the experts at the ICP for use in their daily work.
To quantify the performance of the system, as perceived by its users in a real setting over an extended period of time, a follow-up was done after three weeks by way of an \enquote{explainability evaluation questionnaire}, designed to test the gaps identified in Chapter \ref{chap:literature-review}.
The full questionnaire can be found in Appendix \ref{app:questionnaire}.

The \enquote{explainability evaluation questionnaire} presents five sections:
\begin{itemize}
  \item \textit{confidence}: aimed at assessing wether the use of the system incremented the confidence the clinician felt in making her decisions;
  \item \textit{features}: to understand in more detail which interaction modes were perceived as most useful and the subjective reasons for this.
  Of particular interest is the understanding of the perceived quality of the dialogical interaction modes and of the \enquote{pseudo-MPE} query;
  \item \textit{time}: questions focusing on the the temporal element, mainly the time needed to understand various explanations offered by the system.
  This element is often overlooked in the relevant xAI literature (see Section \ref{sec:explainability-in-bayesian-networks});
  \item \textit{tool}: general questions regarding the use of tool and if any important use-case was felt to be missing;
  \item \textit{clinical}: investigating if the tool was clinically relevant in day-to-day work.
  Unlike the clinical validation presented in Subsection \ref{subsec:clinical-validation-methodology}, these questions investigate \textit{a posteriori} the use of the tool and as such should provide a broader evaluation of its clinical relevance;
  \item \textit{satisfaction}: simple question asking to rate the general satisfaction with the proof of concept system.
\end{itemize}

As discussed throughout Chapter \ref{chap:literature-review} and summarised in Section \ref{sec:literature-review-summary}, one of the main gaps in the field of explainable AI is the absence of real-world validation of the - supposedly - explainable models.
The objective of the questionnaire is to act as an \textit{application-grounded evaluation}, in the taxonomy proposed by \citet{doshi2017towards} and presented in Section \ref{sec:evaluation-of-explainability}, and thus provide what is considered the gold standard for the evaluation of a machine learning system.
Also included, since it is almost always neglected in literature, is a focus on the \textit{temporal element} of the explanations that was noted as important by \citet{gilpin2018explaining}.
Of particular interest is evaluating the Bayesian network - underlying the tool's capabilities - in its capacity to surface cogent explanations for the target user; the questionnaire inflects the questions in order to identify which particular characteristics of the system and BN were perceived by the user as the most useful in order to gain an understanding of the underlying data set.
As noted in Section \ref{sec:explainability-in-bayesian-networks}, by acknowledging the psychological characteristics of an explanation identified by \citet{miller2018explanation}, explanations have various essential characteristics that seem to also be inherent in BNs; the questionnaire thus seeks to understand if these are actually present and perceived as useful, in the sense of enabling explainability, by the domain experts.

The questionnaire is not the only source of the results relating to the \textit{application-grounded evaluation} of the developed system; similarly to \citep{stumpf2009interacting} in their \enquote{think-aloud experiment}, many results and details throughout Chapter \ref{chap:results} will be the outcome of observing and listening to the expert users while they were engaging with the system.
We refer to these as \enquote{informal explainability evaluation results} contrasting them with the \enquote{formal explainability evaluation results} that will be the outcomes of the questionnaire.
\section{Summary}


%%%%
%%%% RESULTS
%%%%

\chapter{Results}\label{chap:results}
\todo{complessita algoritmi}

%%%%
%%%% CONCLUSIONS
%%%%

\chapter{Conclusions}\label{chap:conclusions}


\section{Future developments}\label{sec:future-developments}



\appendix %optional, use only if you have an appendix


\backmatter

\chapter{Glossary} %optional

%\bibliographystyle{alpha}
%\bibliographystyle{dcu}
\bibliographystyle{plainnat}
\bibliography{biblio}

%\cleardoublepage
%\theindex %optional, use only if you have an index, must use
	  %\makeindex in the preamble


\end{document}
