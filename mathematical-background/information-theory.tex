\section{Information Theory} \label{sec:information-theory}
The birth of the field of \textit{information theory} is usually traced back to the seminal paper \enquote{A Mathematical Theory of Communication} (\cite{Shannon1949}) where Claude Shannon set the mathematical basis for the quantification of the amount of \textit{information} transmissible over a noisy channel. 
In his words \enquote{The fundamental problem of communication is that of reproducing at one point, either exactly or approximately, a message selected at another point.}
The concepts of field are broad enough to have influenced practically every other scientific discipline and deep enough to have enabled the \enquote{digital age}, for example by enabling the creation of ever more complicated coding schemes for the compression, reconstruction and obfuscation of digital data.

\subsection{Entropy} \label{subsec:entropy}
In classical mechanical statistics, entropy can be seen as a measure of the uncertainty, or randomness, of a physical system.  
This concept was reapplied by Shannon to measure the amount of randomness in a random variable.
\begin{definition}
	Given a random variable $X$ with probability distribution $\mathbb{P}(X)$, its entropy $\Eta(X)$ is defined as the expected amount of information content carried by $X$ (\cite{Schneider2005}):
\begin{equation} \label{eq:entropy}
	\Eta(X) = \mathbb{E}(I(X)) = \mathbb{E}(-\log (\mathbb{P}(X)) = -\sum_{i=1}^{n} \mathrm{P}\left(x_{i}\right) \log _{b} \mathrm{P}\left(x_{i}\right)
\end{equation}
\end{definition}
The base $b$ of the logarithm defines the unit of measure.  Shannon used $b=2$ as he was dealing with the transmission of digital, binary-coded data; in this case the unit of measure are $bits$.

The simplest example of how information entropy characterises a random variable $X$, is in imagining $X$ to model a coin and the task being to predict the probability of the outcome of a throw being heads.
If the coin is fair, we will not be any more surprised to see the outcome being heads than tails; the entropy is maximum as there is maximum uncertainty regarding the outcome.
However, if the coin is not fair and tails is more probable the we will be more surprised than not to see the outcome being heads.  
The entropy is sub-maximal because there is less uncertainty regarding the outcome: tails is more probable than heads.
If one of the outcomes is impossible, for example if the coin has two heads, then the entropy of the coin is $0$ as there is no uncertainty regarding the result of a toss.


\subsection{Normalised Entropy} \label{subsec:normalised-entropy}
Plain entropy is not a good choice when trying to characterise random variables with different cardinalities of their sample space.
Let us suppose that the objective is to find the variable with the least \enquote{entropic} distribution and we suppose that their values have all been generated by the same process, say Gaussian.
Simply calculating their entropies and ordering them according to this criterion will bias the selection process towards the variables with smallest cardinality.
This is because we supposed them to be distributed in the same way so there will naturally be less uncertainty when there are fewer possible outcomes.
This can easily be understood by imagining the distributions to all be random uniform.

To obviate to this problem we need to \textit{normalise} the entropy so that different-sized variables can be directly compared to each other.
To achieve this, we can look at a measure of \textit{normalised entropy} or \textit{efficiency}:
\begin{equation} \label{eq:normalisedentropy}
 	\eta(X)=-\sum_{i=1}^{n} \frac{p\left(x_{i}\right) \log _{b}\left(p\left(x_{i}\right)\right)}{\log _{b}(n)}
\end{equation}
From Eq. \ref{eq:normalisedentropy} it can be seen that $\eta(X) \in [0,1]$; it is thus normalised and comparable among distributions.
This ratio expresses the amount of entropy found in the distribution compared to the maximum possible entropy when using $n$ symbols, corresponding to the uniform distribution:
\begin{equation}
\mathrm{H}\left(\underbrace{\frac{1}{n}, \ldots, \frac{1}{n}}_{n}\right) = - \sum_{i=1}^n \frac{1}{n} \log _{b} \left( \frac{1}{n} \right) = -n \cdot \frac{1}{n} \log _{b} \left( \frac{1}{n} \right) = - \log _{b} \left( \frac{1}{n} \right) = \log _{b}(n) 
\end{equation}   

\subsection{Mutual Information} \label{subsec:mutualinformation}
Another way of characterising the interrelatedness of two variables is through the concept of \textit{mutual information}, as defined in \cite{Cover2006}, that is closely linked to entropy (Eq. \ref{eq:entropy}).
\begin{definition}
	The mutual information of two random variables $X$ and $Y$ is given by:
	\begin{equation} \label{eq:mutual-information}
		I(X,Y) = \sum_{x \in \mathcal{X}} \sum_{y \in \mathcal{Y}} p_{XY}(x,y) \log \left( \frac{p_{XY}(x, y)}{p_{X}(x) p_{Y}(y)} \right)
	\end{equation}
\end{definition}
NB: In Information Theory, the convention is that $0 \log(0) = 0$.

$I_{XY}$, intuitively, measures the amount of information that $X$ and $Y$ share that can also be seen as the degree to which one variable is informative of the other.
If $X$ and $Y$ are independent then they share no mutual information and knowing one of the two gives no information about the other.
This can be immediately understood by rewriting Eq. \ref{eq:mutual-information} as:
\begin{equation}
	I(X,Y) = H(X) - H(X|Y) = H(Y) - H(Y|X)
\end{equation}
The mutual information I(X,Y) is the reduction in uncertainty of one of the variables given the knowledge of the other.
if $X$ and $Y$ are perfectly correlated ($\rho_{XY}= \pm 1$) then they both convey the same amount of information and $I_{XY}$ is equal to the entropy $\Eta(X) = \Eta(Y)$.

\subsection{Hamming Distance} \label{subsec:hamming-distance}
The \textit{Hamming Distance} is a widely-used distance measure that quantifies the similarity of strings.
\begin{definition}
	The Hamming Distance $d_H(x,y)$ between two vectors $x$ and $y$ is given by:
	\begin{equation}
		d_H(x,y) = \sum_i \Gamma(x_i, y_i)
	\end{equation}
	with $\Gamma(i,j)$ defined as:
	\begin{equation}
		\Gamma(i,j) = 
		\begin{cases}
			0 \quad i \neq j \\
			1 \quad i = j	
		\end{cases}
	\end{equation}
\end{definition}
Given strings of characters, or more in general vectors over some field, of equal length, their Hamming Distance is the number of positions where they differ.
It can be seen as the number of substitutions needed to transform one into the other.

This is a valid distance measure because:
\begin{itemize}
  \item it is non-negative: $d_H(x,y) \geq 0 \quad \forall x, y$
  \item it fulfils the identity of indiscernibles: $x = y \Rightarrow d_H(x,y)=0$
  \item it respects the \textit{Triangle Inequality}: $d_H(x,y) \leq d_h(x,z) + d_H(z,y) $
\end{itemize}

For example, strings $x=01234$ and $y=15244$ have $d_H(x,y)=2$, as they differ in two positions.

\subsection{Jaccard Distance} \label{subsec:jaccard-distance}
The \textit{Jaccard Distance} is a popular metric to measure the similarity of sets.
\begin{definition}
	The Jaccard Similarity Coefficient $J(A,B)$ of two sets $A$ and $B$ is given by:
	\begin{equation}
		J(A,B) = \frac{|A \cap B |}{| A \cup B |}
	\end{equation}
\end{definition}
\begin{definition}
	The Jaccard Distance $d_J(A,B)$ between two sets $A$ and $B$ is given by:
	\begin{equation}
		d_J(A,B) = 1 - J(A,B)
	\end{equation}
\end{definition}

This is a valid distance measure because:
\begin{itemize}
  \item it is non-negative: $d_J(A,B) \geq 0 \quad \forall A, B$
  \item it fulfils the identity of indiscernibles: $A = B \Rightarrow d_J(A,B)=0$
  \item it respects the \textit{Triangle Inequality}: $d_J(A,B) \leq d_J(A,C) + d_J(C,B) $
\end{itemize}

For example, the sets $A = \{ 0, 2, 3, 4, 1 \}$ and $B = \{ 1, 3, 5 \}$ have $d_J(A,B) = \frac{2}{6}$, as their intersection is of cardinality 2 and their union of cardinality 6.
